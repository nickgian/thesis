\chapter{Control-Flow Integrity}\label{ch:cfi}

\ch{The beginning of this chapter is very rough. Jumping from one thing
  to the other without any logical connection.}
\nick{The structure you proposed for this chapter is not clear to me.
Lose the related work title completely?
Or keep the related work section and break the rest into
3 subsections (fine-grained,coarse-grained,verification)?}


Restricting the control-flow of a program in some way has been proven
as an effective technique to mitigate a wide range of attacks.
For example non-executable data (NXD)
can be considered as a form of (very) coarse-grained \CFI where control-flow is
not allowed to reach any memory region that holds non-executable data. Another
popular mitigation technique is to protect return addresses on the stack,
thus restricting the control-flow on returns.

% Moreover it is common that security properties are enforced dynamically by code
% that is statically injected to the program (\EG Inlined Reference Monitors (IRM)
% \cite{Erlingsson04} follow that approach), thus some form of \CFI is required in
% order to ensure that these checks are not circumvented.

\section{Related Work}

\subsection{Balancing between performance and security}\label{sec:security_cfi}

\ch{Before bashing Abadi too much, you should remark the huge
  distinction between their formalization and property, in which they
  consider fine-grained CFI (or at least a generic property that can
  be instantiated to both fine-grained and coarse-grained?), and their
  implementation in which they enforce a very coarse grained thing.
  Alternatively, make sure you mention very explicitly (e.g. in a
  footnote or a parenthesis) that all this is about their
  implementation, not their theory which is discussed later on.}

Abadi \ETAL~\cite{abadi2005} first proposed a technique to enforce
\CFI based on Inlined Reference Monitors (IRMs). In particular, the
method they described (and to some extent formalized) marked all valid
targets of \emph{indirect} control transfers with a unique identifier
and injected checks before all indirect jumps (including return
instructions). However due to high runtime overhead, their actual
implementation assumed that any two destinations are equivalent, in
the sense that they share the same identifier, if the CFG contains
edges from the same set of sources, which significantly reduced the
precision of the CFG. The authors also note that a 2-ID approach where
one identifier is used for calls and another for returns could provide
adequate security in many cases.

The work of Abadi \ETAL sparked interest of researchers who tried to improve
some of the weaknesses of the initial implementation, usually by choosing
between performance against precision and vice-versa.

Bletsch \ETAL \cite{Bletsch:2011:MCA:2076732.2076783} followed the work
of Abadi \ETAL, but changed their checking mechanism to perform the check
after the control flow transfer has occurred which, as the authors claim,
reduced the cache pressure and resulted in better performance. Precision remains
the same with the implementation of Abadi \ETAL.

Zhang \ETAL \cite{Zhang2013} proposed \emph{Compact Control Flow Integrity
and Randomization} (CCFIR), a new efficient way to enforce coarse-grained \CFI.
CCFIR collects all valid targets of indirect control-transfers and stores them
in a random order, in a protected section called ``Springboard section''.
Indirect control-transfers are only allowed to addresses that are in the
Springboard. Their implementation uses a 3-ID approach where one identifier is
used for calls and the two other identifiers are for returns, separating them
between returns to sensitive and non-sensitive functions. Their implementation
also supports interaction between protected and un-protected modules, which
makes it an attractive solution to coarse-grained \CFI.

The security of the above coarse-grained techniques is evaluated in
\cite{outofcontrol_ieeesp2014} where the authors demonstrate
code-reuse attacks against binaries protected by coarse-grained
\CFI. These attacks illustrate the need for fine-grained \CFI which
however incurs a high runtime-overhead penalty making deployment of
such a mechanism unlikely.

\ch{There is more recent related work (not formal, but fine-grained):
\cite{NiuT14} which you should definitely also mention here.}
\nick{will do so, need to have a look at the paper first}

\paragraph{Standard assumptions for effective \CFI}\label{sec:cfi_assumptions}

Most -if not all- \CFI implementations also come with a set of assumptions under
which \CFI holds. Two standard assumptions for all mechanisms that attempt to
enforce \CFI are:
\begin{itemize}
\item Non-Executable Data (\emph{NXD}), a security
mechanism that disallows execution of data.
\item Non-Writable Code (\emph{NWC}). Changing the code of a
program would allow an attacker to circumvent dynamic checks.
\end{itemize}

Both assumptions are fairly standard for modern computers and are enforced
through hardware or software. In some cases \emph{NXD} can be lifted, but
additional security risks and complexity is not worth the minor advantages
offered by such an action.

Many implementations that attempt to do fine-grained \CFI also require that
identifiers used to mark nodes in the CFG are unique.

\subsection{Coarse-grained CFI Micro-Policy}\label{sec:cfi_coarse}

\ch{Turn this into a paragraph that comes when talking about
  coarse-grained CFI (and by the way we can encode it as a
  micro-policy)? This is clearly not related work, so it shouldn't be
  a subsection. Maybe a better way to structure this Section is:
  Coarse-grained CFI (including this Micro-policy),
  Fine-grained CFI,
  Verification of CFI?}

%\ch{consider moving to appendix, or related work section}

We can use the PUMP to implement the coarse-grained \CFI mechanisms described
earlier. Suppose we want to implement 1-ID \CFI, we tag all indirect flow
destinations and sources with a tag \MARK{} and the rest of the instructions as
\UNMARK. Executing instructions that are sources of indirect flows, propagates
their instruction tag to the \pc. We then have to check that the tag on the
destination matches the tag on the tag on the \pc.

\begin{figure}[htb!]
\infrule[Mark]{
  \ii{op} \in \lbrace \ii{Jump}, \ii{Jal} \rbrace
  }{
  \frule{\ii{op}}{}{\MARK}{}{}{} {\MARK}{-}
  }
\bigskip

\infrule[Check]{
  \textit{op} \not \in \lbrace \textit{Jump},\textit{Jal} \rbrace
  }{
  \frule{\ii{op}}{\MARK}{\MARK}{}{}{} {\UNMARK}{-}
  }
\bigskip

\infrule[NoCheck]{
  \textit{op} \not \in \lbrace \textit{Jump},\textit{Jal} \rbrace
  }{
  \frule{\ii{op}}{\UNMARK}{\UNMARK}{}{}{} {\UNMARK}{-}
  }
\caption{Rules enforcing coarse-grained \CFI, \NXD and \NWC}
\end{figure}

Rule \emph{Mark} is used in the case the opcode is Jump or Jal (the
only indirect jumps in the RISC machine we examine) and propagates the
\MARKname tag on the tag of the new \pc. Rule \emph{Check} applies
when the tag on the \pc is set to \MARKname and corresponds to a legal
destination and rule \emph{NoCheck} corresponds to any instruction
that is not a jump source or target.

We do not further study this coarse-grained approach as we consider it
ineffective since attacks against it has already been demonstrated in
\cite{outofcontrol_ieeesp2014}. Instead we are going to focus on
implementing and formalizing a fine-grained \CFI micro-policy.


\subsection{Formal verification of Control-Flow Integrity}\label{sec:cfi_verif}

In \cite{AbadiBEL09} Abadi \ETAL extended their original paper, with
-among other things- a more detailed formal study of \CFI. Their
formalization regarded a much simpler machine than the x86 omitting
all the complexity of modern systems. The machine has a few
instructions, a separate data memory and instruction memory which by
the operational semantics of the machine are non-executable and
non-writable respectively (enforcing \NXD and \NWC by construction),
and a small set of registers. Moreover, their attacker model permits
arbitrary changes to the data memory, arbitrary changes to all the
registers but a few distinguished ones that are used during the
dynamic checks and no changes to the instruction memory.  The authors
proof that under some assumptions every step respects the control-flow
graph even in the presence of an attacker as powerful as the one
described above. Their formal study served as a guideline for the
implementation, but as it is done on paper their proofs cannot be
machine checked. Furthermore, their formalization omits less
interesting but important details such as instruction encoding and
decoding which as shown in \cite{MorrisettTTTG12} are far from trivial
for the x86.

Machine-checked formal verification efforts include \cite{ZhaoLSR11},
which is a SFI formalization for the ARM architecture that also
enforces \CFI. Their formalization was developed using the HOL
theorem prover and a program logic framework they created. However
their benchmarks report a 240\% runtime overhead. The authors of
\cite{CriswellDA14} claim partial proofs for a \CFI enforcement
mechanism focused on the kernel of an operating system. Their runtime
overhead can also reach 100\%.

\section{Micro-Policy for Fine-Grained Control-Flow Integrity}
\label{sec:cfi_fine}

The PUMP hardware allows us to avoid taking the difficult decision between
performance and security. As shown in \cite{pump_asplos2015}, we can enforce a
\emph{fine-grained} \CFI policy with an average runtime overhead of less than 3\%
(maximum overhead of less than 10\%), on the SPEC2006 benchmarks.
\nick{again here, what should I say about overheads and what should I cite?}

We follow the standard approach and require both \NXD and \NWC in
order to correctly enforce \CFI. We designed a composed micro-policy
that enforces \NXD, \NWC and \CFI. We considered designs that lifted
the \NXD and \NWC restrictions but we rejected them, as there did not
seem to be any considerable advantages (\IE{ compatibility with
  self-modifying programs, JIT compilers, \ETC}). Moreover unlike
other \CFI enforcement mechanisms we do not have to rely on the CPU or
the operating system to enforce \NXD and \NWC, therefore lifting these
restrictions would not reduce our assumptions and consequently would
not increase our confidence in the robustness of our approach.

% Allowing the code of the program to change, would in
%practice require for the CFG to change as well, which unless done in
%a controlled, ``safe'' way, would invalidate the enforcement of \CFI.

%The micro-policy we implemented and studied is a composition of a fine-grained
%\CFI micro-policy and the \NWC, \NXD micro-policies explained above.

Our approach uses unique identifiers to tag the contents of the memory that
correspond to sources and potential destinations of indirect flows according to
a binary relation (on the identifiers) $\mathcal{CFG}$.

%\ch{I have more intuition about addresses than about identifiers}
We use the set of tags \TAGS{\DATA,\INSTR{\ii{id}}, \INSTR{\bot}}
where \id is a unique identifier (\IE used to tag the contents of only
one location in the memory). One simple way to achieve this is to use
the address of the instruction as it's \id, for example an instruction
stored at address 100 would be tagged \INSTR{100}. This is the
approach we take in our development. Adapting the rules from
\ref{sec:nwc_nxd}, we shall use \DATAname to tag all contents in
memory that are considered non-executable data, \INSTR{\ii{id}} to tag
all contents in memory that are considered executable instructions and
are sources or targets of indirect control flows and \INSTR{$\bot$} to
tag all other instructions. The rules to enforce \NWC and \NXD are
intuitively the same and only change to account for the splitting of
the \INSTRname tag.

\nick{if/when we move coarse-grained to appendix, then we don't follow
the same idea...}\ch{for now let's try to incorporate the coarse-grained
  micro-policy in 3.1.1 and have an explicit reference to that subsubsection
  here}

We follow the same idea as with coarse-grained \CFI\ch{from 3.1.1},
propagating the
instruction tag of instructions that are sources of indirect flows to
the tag on the \pc of the next state and upon execution of the next
instruction, checking that the tag on the \pc and on the instruction
are in some relation. In the case of coarse-grained \CFI we required
that they match but for fine-grained \CFI we require that they are in
the \CFG relation.

% \begin{figure}[!htpb]
% \begin{align}
%  & (Jump/Jal, -,-, -, -, -) \rightarrow (- ,-)
%  \text{ if (n,m) } \in \CFGm \label{fine_rule1} \tag{1} \\
%  & (Jump/Jal, \DATA, \INSTR{m}, -, -, -) \rightarrow (\INSTR{m},-)
%  \text{ if (n,m) } \in \CFGm \label{fine_rule2} \tag{2} \\
%  & (Store, \DATA, \INSTR{\_}, -, -, \DATA) \rightarrow (\DATA,\DATA)
%  \label{fine_rule3} \tag{3} \\
%  & (Store, \INSTR{n}, \INSTR{m}, -, -, \DATA) \rightarrow (\DATA,\DATA)
%  \text{ if (n,m) } \in \CFGm \label{fine_rule4} \tag{4} \\
%  & (Store, -, -, -, -, \INSTR{\_}) \rightarrow \varnothing
%   \label{fine_rule5} \tag{5} \\
%  & (-, \INSTR{n}, \INSTR{m}, -, -, -) \rightarrow (\DATA,-)
%  \text{ if (n,m) } \in \CFGm \label{fine_rule6} \tag{6} \\
%  & (-, \DATA, \INSTR{m}, -, -, -) \rightarrow (\DATA,-) \label{fine_rule7}
%  \tag{7}\\
%  & (-, -, -, -, -, -) \rightarrow \varnothing \label{fine_rule8}
%  \tag{8}
% \end{align}
% \caption{Rules enforcing fine-grained \CFI, \NXD and \NWC}
% \end{figure}

\begin{figure}[htb!]
\infrule[Flow/Check]{
  \ii{op} \in \lbrace \ii{Jump}, \ii{Jal} \rbrace \andalso
  (src,dst) \in \CFGm
  }{
  \frule{\ii{op}}{\INSTR{src}}{\INSTR{dst}}{}{}{} {\INSTR{dst}}{-}
  }
\bigskip

\infrule[Flow/NoCheck]{
  op \in \lbrace \textit{Jump},\textit{Jal} \rbrace
  }{
  \frule{\ii{op}}{\DATA}{\INSTR{dst}}{}{}{} {\INSTR{dst}}{-}
  }
\bigskip

\infrule[Store/Check]{
  (src,dst) \in \CFGm
  }{
  \frule{\ii{Store}}{\INSTR{src}}{\INSTR{dst}}{}{}{\DATA} {\DATA}{\DATA}
  }
\bigskip

\infrule[Store/NoCheck]{
  \textit{ti} \in \lbrace \INSTR{dst}, \INSTR{\bot} \rbrace
  }{
  \frule{\ii{Store}}{\DATA}{\textit{ti}}{}{}{\DATA} {\DATA}{\DATA}
  }
\bigskip

\infrule[Rest/Check]{
  \textit{opcode} \not \in \lbrace \textit{Jump}, \textit{Jal}, \textit{Store}
  \rbrace \andalso (src,dst) \in \CFGm
  }{
  \frule{\textit{opcode}}{\INSTR{src}}{\INSTR{dst}}{}{}{} {\DATA}{-}
  }
\bigskip

\infrule[Rest/NoCheck]{
  \textit{opcode} \not \in \lbrace \textit{Jump}, \textit{Jal}, \textit{Store}
  \rbrace \andalso
  \textit{ti} \in \lbrace \INSTR{dst}, \INSTR{\bot} \rbrace
  }{
  \frule{\textit{opcode}}{\DATA}{\textit{ti}}{}{}{} {\DATA}{-}
  }
\caption{Rules enforcing fine-grained \CFI\strikeout{, \NXD and \NWC}}
\end{figure}

We note in the above rules that the tag on the \PCname is \DATAname when
no check for a control-flow violation is required and \INSTR{\textit{src}} where
\textit{src} is some id, when an indirect flow instruction was executed and a
check for a control-flow violation is required. An important observation is that
the rules above allow for one control-flow violation to occur, but disallow the
next step and therefore the machine will certainly halt after a violation.

If the PUMP hardware fetched the tag on the memory address the machine is
jumping to and passed it as an argument to input vector, as it does in the
case of a Store instruction, we would be able to enforce \CFI with no violations
at all. \TODO{It can't do that for efficiency reasons?}






