% DOCUMENT FORMAT ==============================================================

\newif\ifdraft  \drafttrue
\interfootnotelinepenalty=10000



\documentclass{ntua-thesis} % a4paper,11pt,twoside,titlepage already set
% \pagestyle{plain} % pagestyle already set
% \usepackage[margin=2.5cm]{geometry} % margins already set


% PACKAGE SETTINGS =============================================================

\usepackage[cm-default]{fontspec}
\usepackage{amsmath}
\usepackage{amssymb}
\usepackage{amsthm}
\usepackage{thmtools}
\usepackage{listings}
\usepackage{graphicx}
\usepackage{xunicode}
\usepackage{xltxtra}
\usepackage{xspace}
\usepackage{url}
\usepackage{color}
\usepackage{bytefield}
\usepackage[format=hang,textformat=simple]{caption}
\usepackage{wrapfig}
\usepackage{float}
\usepackage{bcprules}
\restylefloat{table}
\restylefloat{figure}
\usepackage[color]{coqdoc}
\usepackage{tabularx}
\newcolumntype{m}[1]{>{$}#1<{$}}
\usepackage[%
    xetex,%
    unicode,%
    hyperfootnotes=true,%
    colorlinks=true,%
    pdfpagemode=UseOutlines,%
    pdfstartview=FitH,%
    linkcolor=dkblue,%
    citecolor=blue,%
    urlcolor=magenta,%
    pdftitle={},%
    pdfauthor={Nick Giannarakis},%
    pdfsubject={},%
    pdfkeywords={}%
]{hyperref}

\setromanfont[Mapping=tex-text]{CMU Concrete Roman}
\setsansfont[Mapping=tex-text]{CMU Sans Serif}
\setmonofont[Mapping=tex-text]{CMU Typewriter Text}
\setmainfont[Mapping=tex-text]{CMU Serif}

% CMU Concrete Roman looks nice for headings, but not for text!
% \setromanfont[Mapping=tex-text]{CMU Concrete Roman}
% \setsansfont[Mapping=tex-text]{CMU Concrete Roman}
% \setmonofont[Mapping=tex-text]{CMU Concrete Roman}
% \setmainfont[Mapping=tex-text]{CMU Concrete Roman}


% CUSTOM COMMANDS ==============================================================

\newcommand{\flink}[1]{\footnote{\url{#1}}}
\newcommand{\subscript}[1]{\ensuremath{{\textrm{#1}}}}
\newenvironment{fulltable}[3]{
    \def\tempcaption{#2}
    \def\templabel{#3}
    \begin{table}[hbtp]
    \begin{center}
    \begin{tabular}[c]{#1}
}{
    \end{tabular}
    \end{center}
    \caption{\tempcaption\label{\templabel}}
    \end{table}
}
%macro που δίνει το μέγιστο επιτρεπτό μέγεθος σε μια εικόνα χωρίς να
%παραβιάζει τα όρια του LaTeX
\makeatletter
\def\maxwidth{%
  \ifdim\Gin@nat@width>\linewidth
  \linewidth
  \else
  \Gin@nat@width
  \fi
}

% Generic commands

\declaretheorem[numberwithin=chapter]{theorem}
\declaretheorem[sibling=theorem]{definition}
\declaretheorem[sibling=theorem]{lemma}

% \makeatletter
% \AtBeginDocument{%
%   \renewcommand*{\thetable}{\arabic{chapter}.\arabic{table}}
%   \renewcommand*{\thefigure}{\arabic{chapter}.\arabic{figure}}
%   \renewcommand*{\thelstlisting}{\arabic{chapter}.\arabic{lstlisting}}
%   \let\c@table\c@theorem
%   \let\c@table\c@figure
%   \let\c@table\c@lstlisting
%   \let\thetable\thefigure
%   \let\thetable\thelstlisting
%   \let\c@figure\c@lstlisting
%   \let\c@figure\c@theorem
%   \let\c@lstlisting\c@theorem
%   \let\thefigure\thelstlisting
%   \let\ftype@lstlisting\ftype@figure\ftype@table
%   % give the floats the same precedence
% }
% \makeatother

\makeatletter
  \renewcommand*{\thetable}{\arabic{chapter}.\arabic{table}}
  \renewcommand*{\thefigure}{\arabic{chapter}.\arabic{figure}}
\makeatother

\makeatletter
\AtBeginDocument{%
  \renewcommand*{\thelstlisting}{\arabic{chapter}.\arabic{lstlisting}}
}
\makeatother

\usepackage{cleveref}

\newcommand*{\EG}{e.g.,\xspace}
\newcommand*{\IE}{i.e.,\xspace}
\newcommand*{\ETAL}{{\em et al.}\xspace}
\newcommand*{\ETC}{etc.\xspace}

\newcommand{\TODO}[1]{(\textcolor{red}{\textbf{TODO:}}~#1)}

\definecolor{dkblue}{rgb}{0,0.1,0.5}
\definecolor{bluegreen}{rgb}{0,0.5,0.5}
\definecolor{dkgreen}{rgb}{0,0.6,0}
\definecolor{dkred}{rgb}{0.6,0,0}
\definecolor{dkpurple}{rgb}{0.7,0,0.4}
\definecolor{purple}{rgb}{0.7,0,0.7}
\definecolor{olive}{rgb}{0.5, 0.5, 0.0}
\definecolor{teal}{rgb}{0.0,0.5,0.5}



\newcommand{\ch}[1]{\ifdraft{\color{dkblue}\em (CH: #1)}\fi}
\newcommand{\nick}[1]{\ifdraft{\color{dkred}\em (NG: #1)}\fi}
\newcommand{\chfoot}[1]{\ifdraft\footnote{\color{dkblue}\em (CH: #1)}\fi}
\newcommand{\chrev}[1]{\ifdraft{\color{dkblue}#1}\else#1\fi}
\newcommand{\chlater}[1]{\iflater\ifdraft{\color{dkblue}\em (CH: #1)}\fi\fi}
\newcommand{\bcp}[1]{\ifdraft{\color{blue}\em (BCP: #1)}\fi}
\newcommand{\bcpfoot}[1]{\ifdraft\footnote{\color{blue}\em (BCP: #1)}\fi}
\newcommand{\bcprev}[1]{\ifdraft{\color{blue}\em #1}\else #1\fi}
\newcommand{\bcprevision}[1]{\ifdraft{\color{blue}\em (BCP: #1)}\fi}
\newcommand{\bcplater}[1]{\ifdraft\iflater{\color{blue}\em (BCP: #1)}\fi\fi}
\newcommand{\amd}[1]{\ifdraft{\color{purple}\em (AMD: #1)}\fi}
\newcommand{\amdrevision}[1]{\ifdraft{\color{purple} #1}\fi}
\newcommand{\ud}[1]{\ifdraft{\color{cyan}[UD -- #1]}\fi}
\newcommand{\udrev}[1]{\ifdraft{\color{cyan}#1}\else #1\fi}
\newcommand{\aaa}[1]{\ifdraft{\color{dkgreen}\em (AAA: #1)}\fi}

\newcommand*{\ii}[1]{\textit{#1}}

% Micro-policies related commands
\newcommand{\mem}[0]{\ii{mem}\xspace}
\newcommand{\imem}[0]{\ii{im}\xspace}
\newcommand{\dmem}[0]{\ii{dm}\xspace}
\newcommand{\reg}[0]{\ii{reg}\xspace}
\newcommand{\ok}[0]{\ii{ok}\xspace}
\newcommand{\rd}[2]{#1[#2]}
\newcommand{\upd}[3]{#1[#2{\leftarrow}#3]}
\newcommand{\pc}[0]{\ii{pc}\xspace}
\newcommand{\epc}[0]{\ii{epc}\xspace}
\newcommand{\PCname}[0]{\ii{pc}\xspace}
\newcommand{\EPCname}[0]{\ii{epc}\xspace}
\newcommand{\ra}[0]{\ii{ra}}
\newcommand{\tpc}[0]{t_\ii{pc}}
\newcommand{\tra}[0]{t_\ii{ra}}
\newcommand{\ti}[0]{t_\ii{i}}
\newcommand{\targ}[1]{t_\ii{#1}}
\newcommand{\extra}[0]{\ii{int}}
\newcommand{\TRANSFER}{\ii{transfer}\xspace}
\newcommand{\EXTRASTATES}{\ii{EX}}
\newcommand{\cache}{\ii{cache}\xspace}
\newcommand{\step}[2]{\ensuremath{#1\to#2}}
\newcommand{\stepn}[2]{\ensuremath{#1\to_n#2}}
\newcommand{\stepa}[3]{\ensuremath{#1\to_a^{#3}#2}}
\newcommand{\longstep}[2]{\begin{array}[t]{l@{\;}l}&#1\\\to&#2\end{array}}

\newcommand{\old}{_{\mathit{old}}}
 \newcommand{\trapaddr}{\ii{trapaddr}}
 \newcommand{\handler}[8]{\ii{transfer}(#1,#2,#3,#4,#5,#6) = (#7,#8)}
\newcommand{\rulelookup}[8]{%
   \cache \vdash (#1,#2,#3,#4,#5,#6) \mapsto (#7,#8)}
\newcommand{\rulelookupshort}[2]{%
   \cache \vdash #1 \mapsto #2}
\newcommand{\rulelookupmissshort}[1]{%
   \cache \vdash #1 \uparrow}
\newcommand{\rulelookupmiss}[6]{%
   \cache \vdash (#1,#2,#3,#4,#5,#6) \not \mapsto}
\newcommand{\savestate}[7]{%
  #7 [0..5 \leftarrow ({#1},{#2},{#3},{#4},{#5},{#6})]}
\newcommand{\savestateshort}[2]{%
  #2 [0..5 \leftarrow {#1}]}

\newcommand{\astat}[4]{\ensuremath{(#1,#2,#3,#4)}}
\newcommand{\acfistat}[5]{\ensuremath{(#1,#2,#3,#4,#5)}}
\newcommand{\cstat}[5]{\ensuremath{(#1,#2,#3,#4,#5)}}
\newcommand{\stat}[5]{\ensuremath{(#1,#2,#3,#4,#5)}}

\newcommand{\DATA}[0]{\ii{Data}}
\newcommand{\INSTRname}[0]{\ii{Code}\xspace}
\newcommand{\INSTR}[1]{\INSTRname~#1}
\newcommand{\DATAname}[0]{\ii{Data}\xspace}

\newcommand{\USERname}[0]{\ii{User}\xspace}
\newcommand{\USER}[1]{\USERname~#1}
\newcommand{\MONITOR}[0]{\ii{Monitor}\xspace}
\newcommand{\ENTRYname}[0]{\ii{Entry}\xspace}
\newcommand{\ENTRY}[1]{\ENTRYname~#1}
\newcommand{\enc}[1]{\ii{enc}~#1}
\newcommand{\atom}[2]{#1@#2}

\newcommand{\CFI}[0]{\emph{CFI}\xspace}
\newcommand{\CFG}[0]{$\mathcal{CFG}$\xspace}
\newcommand{\CFGm}[0]{\mathcal{CFG}\xspace}
\newcommand{\J}[0]{$\mathcal{J}$\xspace}
\newcommand{\Jm}[0]{\mathcal{J}\xspace}
\newcommand{\SUCC}[1]{$\mathcal{SUCC_\CFGm^{#1}}$\xspace}
\newcommand{\SUCCm}[1]{\mathcal{SUCC_\CFGm^{#1}}\xspace}

\newcommand{\INITIAL}[1]{\textit{initial~#1}\xspace}

\newcommand{\frule}[8]{#1:\{\ifthenelse{\equal{#2}{}}{}{\mathit{PC}{=}#2\ifthenelse{\equal{#3}{}}{}{,}}
                           \ifthenelse{\equal{#3}{}}{}{\mathit{CI}{=}#3}\ifthenelse{\equal{#4}{}}{}{,}
                           \ifthenelse{\equal{#4}{}}{}{\mathit{OP}_1{=}#4}\ifthenelse{\equal{#5}{}}{}{,}
                           \ifthenelse{\equal{#5}{}}{}{\mathit{OP}_2{=}#5}\ifthenelse{\equal{#6}{}}{}{,}
                           \ifthenelse{\equal{#6}{}}{}{\mathit{MR}{=}#6}
                           \} \to \{\mathit{PC}'{=}#7, \mathit{RES}{=}#8\}}

\newcommand{\NXD}[0]{\emph{NXD}\xspace}
\newcommand{\NWC}[0]{\emph{NWC}\xspace}

\newcommand{\MARK}[0]{\emph{Marked}}
\newcommand{\UNMARK}[0]{\emph{Unmarked}}

\newcommand{\MARKname}[0]{\emph{Marked} \xspace}
\newcommand{\UNMARKname}[0]{\emph{Unmarked} \xspace}

\newcommand{\TAGS}[1]{$\mathcal{T} = \lbrace #1 \rbrace$}
\newcommand{\id}[0]{\ii{id}\xspace}
\newcommand{\word}[0]{\ii{word}\xspace}

\newcommand{\init}[1]{\ensuremath{\mathit{Init}(#1)}}
\newcommand{\initf}[0]{\ensuremath{\mathit{Init}}}
\newcommand{\Initf}[1]{\ensuremath{\initf(#1)}}

%TIKZ

% Mysterious example found on the TeX Stack Exchange to draw a diagram
% with a reflexive-transitive closure.
\usepackage{tikz}
\usepgflibrary{plotmarks}
\usetikzlibrary{matrix,arrows,decorations.pathmorphing,positioning,backgrounds,fit}

\tikzset{snake it/.style={decorate, decoration=snake}}

\usetikzlibrary{calc}



% CODE HIGHLIGHTING ============================================================

% CODE HIGHLIGHTING ============================================================

\usepackage{lstcoq}
\definecolor{ForestGreen}{RGB}{0, 155, 85}
\definecolor{Purple}{RGB}{153, 71, 155}
\definecolor{BrickRed}{RGB}{182, 50, 28}
\definecolor{darkgreen}{cmyk}{0.7, 0, 1, 0.5}

\lstset{%
    numbers=none,
    numberstyle=\tiny\color[rgb]{0.5,0.5,0.5},
    basicstyle=\ttfamily\footnotesize,
    basewidth=0.59em,
    keywordstyle=[3]{},
    commentstyle=\itshape\footnotesize,
    tabsize=4,
    frame=single,
    frameround=tttt,
    showstringspaces=false,
    breaklines=false,
    captionpos=b,
    aboveskip=\bigskipamount,
    belowskip=\bigskipamount,
    escapechar=^,
    keywordstyle=\color[rgb]{0,0,1},
    commentstyle=\color[rgb]{0.133,0.545,0.133},
    stringstyle=\color[rgb]{0.627,0.126,0.941}
}

\newcommand\code[1]{{\tt\small #1}}
\newcommand\regex{\coqe{regex}}
\newcommand\dk{\coqe{decide_kleene}}

% Style options:
% numberstyle,basicstyle,identifierstyle,commentstyle,stringstyle
% keywordstyle=[1]{},keywordstyle=[2]{},directivestyle
% \small\tiny\footnotesize\itshape\ttfamily\bf
\lstdefinestyle{coq_style}{%
  language=Coq, float=htb!
}
\lstnewenvironment{Coq}[2]{%
  \nopagebreak
  \lstset{style=coq_style,label={#1},caption={#2}}
}{}

\newcommand{\includecode}[4][Coq]{%
  \nopagebreak
  \lstinputlisting[label={#2},caption={#3},style={#1_style}]{#4}
}
\newcommand{\lstheader}[2]{%
  \begin{lstlisting}[label={#1},caption={#2},style=coq_style]
}

% DOCUMENT INFORMATION =========================================================

%\title{}
\title{Formally Verified Tag-Based Enforcement of\\Control Flow Integrity}
% Monitor, Mechanism -- too low level
% Formally Verified Tag-Based Control Flow Integrity Enforcement
\author{Νικόλαος Γιανναράκης}
\thesis[του]{Νικόλαου Γιανναράκη}
\presenting{11}{9}{2014}
\supervisor[Αν. Καθηγητής ]{Νικόλαος Παπασπύρου} % the space is necessary
\epitropiF[Αν. Καθηγητής ]{Κωστής Σαγώνας}
\epitropiS[Αν. Καθηγητής  ]{Ιώαννης Σμαραγδάκης}
\department{Σχολή Ηλεκτρολόγων Μηχανικών και Μηχανικών Υπολογιστών}
\division{Τομέας Τεχνολογίας Πληροφορικής και Υπολογιστών}
\lab{Εργαστήριο Τεχνολογίας Λογισμικού}


% MAIN DOCUMENT ================================================================

\begin{document}

% \frontmatter
\maketitle
\def\templen{\parindent}
\setlength{\parindent}{0pt}
\setlength{\parskip}{1.5ex plus 0.5ex minus 0.2ex}
\chapter*{Περίληψη}
\label{absgr}
\addcontentsline{toc}{chapter}{Περίληψη}
  Μια ευρεία γκάμα επιθέσεων λογισμικού προσπαθούν να ανακτήσουν τον
  έλεγχο ροής του προγραμμάτος με σκοπό να τροποποιήσουν τη
  συμπεριφορά του. Η Ακεραιότητα Ελέγχου-Ροής είναι μία αποτελεσματική
  πολιτική ασφαλείας, που μπορεί να αποτρέψει όλες τις επιθέσεις που
  επιχειρούν να παρακάμψουν την αρχική ροή ελέγχου του προγράμματος. 

  Σε αυτή τη διπλωματική εργασία, χρησιμοποιούμε το εργαλείο
  διαδραστικών αποδείξεων Coq για να αιτιολογήσουμε τυπικά την
  ορθότητα και την αποτελεσματικότητα ενός δυναμικού ελεγκτή που
  επιβάλλει Ακεραιότητα Ελέγχου-Ροής, βασιζόμενος σε ένα καινοτόμο
  μηχανισμό ασφαλείας που χρησιμοποιεί λογισμικί και υλικό.
  Συγκεκριμένα, αποδεικνύομε οτι ο μηχανισμός επιβάλλει Ακεραιότητα
  Ελέγχου-Ροής ακόμα και υπό την παρουσία ενός ισχυρού κακόβουλου
  χρήστη. Επιπλέον αποδεικνύουμε μέσω εκκαθάρισης ότι ένα μηχάνημα στο
  οποίο τρέχει ο δυναμικός ελεγκτής για την Ακεραιότητα Ελέγχου-Ροής,
  επακριβώς εξομοιώνει όλες τις συμπεριφορές ενός αφηρημένου
  μηχανήματος που έχει Ακεραιότητα Ελέγχου-Ροής εκ κατασκευής.

\begin{keywords}
ροή-ελέγχου, ασφάλεια, επαλήθευση, αρχιτεκτονικές με ετικέτες
\end{keywords}


\chapter*{Abstract}
%% English Abstract
\label{Abstract}
\addcontentsline{toc}{chapter}{Abstract}

  A wide-range of software attacks attempt to hijack the control-flow
  of the program in order to alter its behavior. Control-Flow
  Integrity is an effective security policy, able to thwart all
  attacks that attempt to circumvent the original control-flow of a
  program.

  In this thesis, we use the Coq proof assistant to formally reason
  about the correctness and the effectiveness of a dynamic monitor
  enforcing \CFI, based on a novel software-hardware security
  mechanism. In particular, we prove that the mechanism enforces \CFI
  even in the presence of a powerful attacker. Furthermore, we prove
  by refinement that a machine running the dynamic monitor for \CFI,
  precisely emulates all behaviors of an abstract machine that has
  \CFI by construction.

\begin{keywordseng}
control-flow, security, verification, tagged architectures
\end{keywordseng}




%%Greek Acknowledgements
\begin{acknowledgements}

Θα ήθελα να ευχαριστήσω τον C\u{a}t\u{a}lin Hri\c{t}cu για την εμπιστοσύνη που μου έδειξε, την ευκαιρία να εργαστώ σε
ένα κορυφαίο ερευνητικό κέντρο και την καθοδήγηση του κατα την εκπόνηση αυτής της διπλωματικής εργασίας. 

Θα ήθελα επίσης να ευχαριστήσω τους καθηγητές μου κ.Νίκο Παπασπύρου και κ.Κωστή Σαγώνα για τη διδασκαλία τους μέσω
της οποίας μου μετέφεραν το ενδιαφέρον τους για τις γλώσσες προγραμματισμού αλλά και τη βοήθεια που μου
προσέφεραν όποτε τη χρειάστηκα στη μέχρι τώρα ακαδημαϊκή μου πορεία.

Τέλος, θα ήθελα να ευχαριστήσω την οικογένεια μου και τη σύντροφο μου
Ζωή Παρασκευοπούλου για την αστείρευτη τους στήριξη και αγάπη.
\end{acknowledgements}

\setlength{\parindent}{\templen}
\tableofcontents
\listoffigures
\renewcommand{\lstlistlistingname}{List of Listings}
\lstlistoflistings
\renewcommand{\listtheoremname}{List of theorems and definitions}
\listoftheorems[ignoreall,show={definition,lemma,theorem}]


% \mainmatter
% moved these two commands here so that they don't influence the toc
\setlength{\parindent}{12pt}
\setlength{\parskip}{0.5pt}
\chapter{Introduction}\label{ch:introduction}

\section{Motivation}\label{sec:motivation}

Computer hardware and software continuously grow in size and complexity and as a
result ensuring the absence of exploitable behaviors is becoming increasingly 
difficult. In the era when \FEEDBACK{where?}computer systems are used extensively to
carry important information (e.g. credit card numbers, national security
documents), it has been widely accepted that security of these systems is a
priority. Researchers have identified a number of potential vulnerabilities
which arise from the violation of known but in-practice unenforceable safety
and security policies.
 
So far, computer security has been delegated mostly to software, while the 
hardware is being almost completely controlled by the software.
High-level languages are becoming more widely used,
due to features such as strong type systems with type inference
and automatic memory
management, making programming less error prone and reducing the number of
exploitable bugs. Furthermore, in order to strengthen the security of computing
systems a variety of low-level
mitigation techniques \TODO{reference some? stack canaries, ASLR, $W \oplus X$}
% http://prosecco.gforge.inria.fr/personal/hritcu/talks/05-control-hijacking-defenses.pdf
% Úlfar Erlingsson: Low-Level Software Security: Attacks and Defenses. FOSAD 2007: 92-134
have been
proposed, however these are mostly ad-hoc solutions designed to prevent specific
known attacks, rather than enforcing a security policy 
by preventing a well-defined class of attacks,
thus making it hard to reason about their
effectiveness. In fact most of these mitigation techniques can be circumvented
by attackers, \TODO{reference; Overcoming CFI; Eternal War in Memory}
which has lead to a continuous ``chase''
between attackers and security researchers.

One common attack technique is to 
exploit some low-level vulnerability such as a
buffer overflow to redirect the control flow to 
attacker injected code. This attack
can be stopped by a simple protection scheme known as $W \oplus X$, which
enforces that a memory page is either executable or writable but not both.
Unfortunately, clever attack techniques can bypass $W \oplus X$. In
particular, attackers have been using code-reuse attacks
(e.g. return/jump - oriented programming) that allows them to chain together
existing pieces of code to achieve malicious behavior without directly
introducing new code.
Abadi~\ETAL\cite{abadi2005}
introduced a property called Control Flow Integrity (CFI),
which provides effective protection against control-flow hijacking attacks.
CFI enforces that any execution of a program will
respect a statically computed control flow graph (CFG).
\FEEDBACK{missing references throughout}

The main contribution of this thesis is the formalization and
verification of a dynamic monitor for CFI, based on a generic
hardware-software security mechanism.
%
We provide a precise attacker model and prove in Coq that the monitor
enforces a variant of the CFI property proposed by
Abadi~\ETAL\cite{abadi2005}.
%
To obtain this result we prove refinement between a concrete
machine running a monitor satisfying our Coq specification
and an abstract machine having CFI by construction.
%
We conclude the proof using a novel generic result stating that under
certain assumptions CFI is preserved by refinement.
\FEEDBACK{Is there anything missing here?}

\section{Thesis Outline}\label{sec:outline}
Map
1. Intro
2a. Safety and Security Policies
2b. Micropolicies
3. CFI description
4. CFI formalization
5. Conclusions and Future work
6. Related work

Chapter 2 of this thesis briefly describes the basic requirements a security
policy must satisfy and puts into context the framework we utilize in order
to formalize and enforce a Control-Flow Integrity (CFI) policy.
Chapter 3 discusses the current state of research on Control-Flow Integrity 
and clarifies our goals and contributions to it.
Chapter 4 describes in detail the design of a fine-grained CFI policy and
how we used the framework from Chapter 2 in order to enforce the policy
and formally reason about it's security properties.
Chapter 5.. conclusions, future work?
Chapter 6.. related work and bibliography?
Appendix with code and/or step relations etc.?

\chapter{Micro-policies: A Framework for Verified, Hardware-Assisted Security Monitors}\label{ch:policies}
Currently the hardware provides very limited security mechanisms \TODO{name some;
  4 protection rings, page-level memory protection via virtual memory}, 
leaving most of the work to the software. This requires that the software 
performs various sanity-checks during an execution and that it carefully 
maintains various safety and security invariants, a tedious and error-prone task
that results in high runtime performance overheads.  

Many potentially effective mitigation techniques are not deployed because of the
performance overhead they incur. Another requirement for deployment of a 
protection mechanism is the compatibility with existing executables and 
the degree of intervention required by a human. 
Usually even making slight changes to a code and redistributing has high cost
and the protection mechanism is likely to see very low adoption. 

The lack of efficient and effective generic ways to enforce security policies, 
forces programmers to protect their own code, a task which is not trivial even 
for the small and simply programs. As a result most, if not all, software 
carries weaknesses which can be exploited by an attacker. ``Safe'' languages, 
automate some of the checks required and eases the work of the programmer,
for example by implementing array bounds checking or by disallowing 
pointer-arithmetic. However these solutions only reduce the chance of 
introducing exploitable bugs in a program and do not enforce stricter, 
more effective policies such as Control Flow Integrity
or complete Memory Safety (spatial/temporal protection for heap and stack). 
In addition, we still need effective and efficient protection mechanisms for a 
plethora of software written in unsafe languages such as C.

\FEEDBACK{More text introducing micro-policies:
instruction-level security monitoring mechanisms based
on fine-grained metadata tags (PUMP is just a way of implementing this 
  efficiently)
}

\FEEDBACK{Can the main idea of the CFI micro-policy be introduced here
  already? See grant proposal.}

Our key motivating insight is that a wide range of policies can
be dynamically enforced by tagging all data with {\em metadata tags}
describing its
provenance or security restrictions (\EG ``this is an instruction,'' ``this
came from the network,'' ``this is secret,'' ``this is sealed with key
$k$''), propagating the metadata as instructions are executed, and checking
that policy rules about the metadata are enforced throughout the
computation.  We use the term {\em micro-policies} for such
mechanisms---i.e., for instruction-level security-monitoring mechanisms
based on fine-grained metadata.  Our goal is to develop a generic framework
for defining, reasoning about, and implementing micro-policies.

More precisely, a micro-policy consists of
\begin{enumerate}
\item a set of {\em metadata tags} that are used to tag each piece of data in
the memory and registers (including the PC)
\item a {\em transfer function} that, given the current opcode and the tags
on the current instruction, the PC, and the operands of the instruction,
specifies how the PC and the instruction's result should be tagged in the
next machine state
\item a description of how to annotate
  the {\em initial state} of a process with tags
\item for some micro-policies, a set of {\em monitor services} that can be
invoked by user code.
\end{enumerate}

\ch{should at least mention that this can be done efficiently (and
  maybe reference PUMP section)}

\section{Control-Flow Integrity Micro-Policy}

\ch{Make it clearer that this is informal and
  you will return to the formalization later on}

\ch{Could break up NXD+NWC from CFI; could keep only the NXD+NWC
  part here, and have the CFI part in next chapter?}

We begin with a micro-policy targeting control-flow hijacking attacks,
in which an attacker exploits a low-level vulnerability (e.g. a buffer
or integer overflow) to gain full control of a target program~\cite{
  ShellcoderHandbook, Szekeres2013, Smashing1996, SkyLined, PincusB04,
  Sotirov07, DanielHM08, AfekS07, Dobrovitski03}.
%
As a first line of defense, we can use tags to make code non-writable
(NWC) and data non-executable (NXD), preventing the injection and
execution of an attacker payload.
%
This useful defense appears in various forms in existing systems.
However, it does not prevent code-reuse attacks~\cite{Newsham1997,
  SolarDesigner1997, McDonald1999, Shacham07, Checkoway2010,
  Buchanan2008, SnowMDDLS13, outofcontrol_ieeesp2014} such as return- or
jump-oriented programming~\cite{Shacham07, Checkoway2010}, where the
attacker chains together existing code snippets (``gadgets'') to induce
arbitrary malicious behavior.
%
We therefore use tags to enforce fine-grained {\em control-flow integrity
  (CFI)}~\cite{AbadiBEL09, ZhaoLSR11, Zhang2013, CriswellDA14, NiuT14,
  ZhaoLSR11, CriswellDA14} on top of basic NWC and NXD protection.
%
This ensures that all indirect control flows (computed jumps) adhere
to a fixed control flow graph (CFG).

\newcommand{\CODEname}[0]{\ii{Code}}
\newcommand{\CODE}[1]{\CODEname~#1}

\makeatletter
\newdimen\OPCODEwidth
\OPCODEwidth .5in
\newdimen\RULEwidth
\newcommand{\TRUE}{\text{\tt true}}
\newcommand{\RULE}[9]{
\gdef\RULEARROW{\ifthenelse{\equal{#7}{}}{\Rightarrow}{\rightarrow}}  % HACK!
\gdef\RULEINPUT{(#2,#3,#4,#5,#6)}
\gdef\RULEOUTPUT{(#8,#9)}
\gdef\RULECOND{%
  \ifthenelse{\equal{#7}{}}{}%
             {\ifthenelse{\equal{#7}{\TRUE}}{}%
                         {\mathrm{\;if\;}#7}}%
}
&& \hspace*{-18.5em}
  \hbox to 1in {
      \hbox to \OPCODEwidth {\ifx&#1&\else$#1$\ : \fi}
      % See how big it is on one line
      \setbox \@tempboxa \hbox{$\RULEINPUT \RULEARROW \RULEOUTPUT \RULECOND$}
      \RULEwidth \wd\@tempboxa
      % Does it fit?
      \ifdim \RULEwidth < 2.4in
        % Use it
        \box\@tempboxa
      \else\ifdim \RULEwidth < 4.8in
        % Put it on two lines
        $ \hspace*{-1.3em}
        \begin{array}[t]{@{}l@{\ }l}
            & \RULEINPUT \\
            \RULEARROW
            & \RULEOUTPUT\RULECOND
        \end{array}
       $
      \else
        % Put it on three lines
        $ \hspace*{-1.3em}
        \begin{array}[t]{@{}l@{}l}
            & \ \RULEINPUT \\
            \RULEARROW
            & \ \RULEOUTPUT \\
            & \RULECOND
        \end{array}
       $
     \fi\fi
  }
}

\newcommand{\RULEWITHPREMISE}[9]{
\typicallabel{}
  \infrule{#7}{\ii{#1} : (#2, #3, #4, #5, #6) \to (#8,#9)}
\typicallabel{MkKey}
}

We use tags to distinguish between code and data.
%
Tags on memory and the PC are drawn from the set
%
$
\DATA \;|\; \CODE{\ii{addr}} \;|\; \CODE{\bot}
$
(registers are always tagged $\DATA$).
%
To simplify the CFG conformance checks, instructions that are the
source or target of indirect control flows are tagged with
$\CODE{\ii{addr}}$, where $\ii{addr}$ is the address of that
instruction in memory.
%
For example, a $\ii{Jump}$ instruction stored at address $500$ is
tagged $\CODE{500}$.
%
All other instructions are tagged $\CODE{\bot}$.
%
\aaa{Actually, we can't use the instruction's address on the tag if we
  are to have the same number of bits on words and tags. Maybe change
  to ``id''?}

We write transfer functions as a collection of {\em
  symbolic rules}~\cite{popl2015, pump_hasp2014}.
% \ch{Should we remove all references to \cite{pump_ccs2014}?}
% BCP: yes
%
(The PUMP hardware uses a lower-level {\em concrete rule} format, 
described in \autoref{implementation}.)
%
Each symbolic rule has the form
%
% \begin{eqnarray*}
% \RULE
%   {\mathit{opcode}}
%   {\mathit{PC}}{\mathit{CI}}{\mathit{OP_1}}{\mathit{OP_2}}{\mathit{OP_3}}
%   {\TRUE}
%   {\mathit{PC'}}{\mathit{R'}}
% \end{eqnarray*}
%
\newcommand{\INLINERULE}[8]{\mathit{#1}
\mathrel{:}\allowbreak
 ({\mathit{#2}},\allowbreak{\mathit{#3}},\allowbreak{\mathit{#4}},\allowbreak{\mathit{#5}},\allowbreak{\mathit{#6}})
 \rightarrow\allowbreak ({\mathit{#7}},\allowbreak{\mathit{#8}})
}%
``$\INLINERULE{opcode}{PC}{CI}{OP_1}{OP_2}{OP_3}{PC'}{R'}$,''
%
which says that the rule matches on the given {\it opcode} together
with the metadata tags on the program counter ($\mathit{PC}$), the
current instruction ($\mathit{CI}$), and on up to three operands
($\mathit{OP_1}$ to $\mathit{OP_3}$).
%
\aaa{Eventually, it would be nice to have variadic rules with a number
  of operands that depends on the opcode. The POPL symbolic machine is
  soon going to support this new format}%
If the rule applies, the right-hand side determines how to update the
tags on the PC ($\mathit{PC'}$) and on the result of the operation
($\mathit{R'}$).
%
We write ``$-$'' to indicate input or output fields that are ignored
(``wildcard'').
%
\chrev{All instructions that are not explicitly allowed by the
  symbolic rules are disallowed.}%
\aaa{We should choose only one of $-$ or $\_$ for our wildcard and use
  it consistently (cf. the ``Store'' rule below)}%

The CFI transfer function enforces that only memory locations tagged
$\DATA$ can be modified (NWC) and only instructions fetched from
locations tagged $\CODEname$ can be executed (NXD).
%
The symbolic rule for the $\ii{Store}$ instruction illustrates both
these points:
%
\begin{eqnarray*}
\INLINERULE
  {\ii{Store}}
  {\DATA}{\CODE{\text{\textunderscore}\,}}{-}{-}{\DATA}
  {\DATA}{-}
\end{eqnarray*}
%
It requires the fetched $\ii{Store}$ instruction to be tagged
$\CODEname$ and the written location to be tagged $\DATA$.
%
This rule only applies when the PC is also tagged $\DATA$, which is
the case when the $\ii{Store}$ instruction was reached by direct
control flow (not a computed jump).
%
The rule preserves the $\DATA$ tag on the PC, since $\ii{Store}$ is
not a computed jump.
%
Performing a computed jump (\EG using \ii{Jal}, a
jump-and-link instruction) requires that the current instruction be
tagged $\CODE{\ii{src}}$ for some address $\ii{src}$.
%
\begin{eqnarray*}
\INLINERULE
  {\ii{Jal}}
  {\DATA}{\CODE{\ii{src}}}{-}{-}{-}
  {\CODE{\ii{src}}}{-}
\end{eqnarray*}
%
This rule copies $\CODE{\ii{src}}$ to the PC tag to indicate
that a jump from \ii{src} has just occurred.
%
Only on the next instruction do we have enough information about the
destination in the tags to check that the jump is indeed allowed by
the CFG.
%
For this we add a second rule for \ii{Store}, dealing with the case
where it is the target of a jump and thus the PC is tagged
$\CODE{\ii{src}}$.
%
\RULEWITHPREMISE
  {\ii{Store}}
  {\CODE{\ii{src}}}{\CODE{\ii{tgt}}}{-}{-}{\DATA}
  {(\ii{src},\ii{tgt}) \in \ii{CFG}}
  {\DATA}{-}
%
\aaa{Maybe we could discuss here a little bit why we verify the jump
  on the next instruction, as opposed to when the jump is
  performed. This might get some people confused, since this is not
  very natural and fundamentally driven by our current design of the
  PUMP. Even Nick wanted to know if we couldn't do it differently.}%
  The premise of this rule ensures that the source and target of the
  just-performed jump are allowed by the CFG.
%
  We add a similar rule for each instruction, including jumps (since
  the target of a computed jump can itself be another computed
  jump):
%
\RULEWITHPREMISE
{\ii{Jal}}
{\CODE{\ii{src}}}{\CODE{\ii{tgt-src}}}{-}{-}{-}
{(\ii{src},\ii{tgt-src}) \in \ii{CFG}}
{\CODE{\ii{tgt-src}}}{-}

This micro-policy enforces fine-grained CFI~\cite{NiuT14,
  outofcontrol_ieeesp2014, CriswellDA14}, not coarse-grained
approximations~\cite{AbadiBEL09, Zhang2013} that are potentially vulnerable
to attack~\cite{outofcontrol_ieeesp2014}.  Indeed, we recently
proved~\cite{popl2015} that this micro-policy enforces a variant of the CFI
property introduced by Abadi~\ETAL\cite{AbadiBEL09}, ensuring that all
indirect control flows adhere to a fixed CFG.
%
Recent simulations of an optimized PUMP architecture~\cite{pump_asplos2015}
show that the CFI policy can be enforced with around 3\% average runtime
overhead.


\section{Formalization and Verification of Micro-Policies}\label{sec:micropolicies}

The software components that can be changed to enforce a security policy
are collectively called a micro-policy.\FEEDBACK{Should be quite useless once
  micro-policies are defined at the beginning of chapter}
Unsurprisingly, designing a security policy, reasoning about it's effectiveness 
against potential attackers and encoding it as a micro-policy can become a 
complex task. Azevedo \ETAL \cite{pump_popl2015} built a generic framework for
defining micro-policies on top of a simple machine modeling a RISC processor 
augmented with the PUMP hardware (referred to as concrete machine), formalized
this framework in Coq and used it to define and formally verify micro-policies
for dynamic sealing, control-flow integrity, memory safety, compartmentalization
and protecting the monitor code itself.
\FEEDBACK{maybe I should mention the word monitor at some point earlier}

The framework offers a high-level machine, called the symbolic machine, that
abstracts away from unnecessary implementation details and can be used as an 
interface to the concrete machine, simplifying the work of the micro-policy 
designer. Additionally the symbolic machine is used to simplify correctness 
proofs. To instantiate the symbolic machine, the micro-policy designer needs to
provide a set of symbolic tags which will be used to tag the various values of
the machine, a transfer function that monitors program execution and determines
how tags are propagated in each step and optionally a set of monitor services 
that are partial functions from machine states to machine states and can be used
to control the monitor's behavior dynamically.

In order to implement the micro-policy at the concrete machine level, one needs
to additionally provide machine code that implements the transfer function, an
encoding of tags to words and machine code for any monitor services that the
micro-policy may use. The relation between the symbolic and the concrete machine
is formally defined as a two-way refinement (forward and backward). This is a 
generic refinement proof, parameterized by the encoding of the symbolic tags to
words and a proof of correctness of the monitor code for a micro-policy.
The designer of a micro-policy can use this two-way refinement simply by
providing these two parameters.

\subsection{Correctness of micro-policies}\label{sec:verification}

For each micro-policy an abstract machine which serves as a specification to the
invariants the policy designer wants to enforce is defined. The abstract machine 
is ``correct'' by construction, meaning that it's designed to respect those 
invariants. Using the symbolic machine as an intermediate step to simplify the
proofs, by proving a refinement between the symbolic and the abstract machine 
and by utilizing the the generic refinement between the symbolic and the
concrete machine, we can prove a refinement between the abstract and
the concrete machine, thus showing that every valid step for the concrete
machine is also a valid step for the abstract machine. 
\FEEDBACK{say smth about steps and refinement earlier..}

All the machines introduced in the original paper by Azevedo \ETAL 
\cite{pump_popl2015}, 
as well as this thesis, have a similar structure. In particular, they share a
common RISC-based instruction set (with a few - uninteresting for the scope of
this thesis - exceptions) and they have a fixed number of general-purpose
registers, along with a pc register. Of course the abstract machine defined
by the policy designer can differ in various ways, but more similarities with
the symbolic machine implies easier proofs of correctness.

\FEEDBACK{Introduce the (names of the) various machines and
  how they relate to each-other. Nice diagram?}

\subsection{Symbolic Machine}\label{sec:symbolic}

As mentioned above, the symbolic machine enables us to abstract away from 
various low-level details of the concrete machine. We can express and reason
about policies in terms of mathematical objects written in Gallina rather than
machine code and the corresponding proofs for the concrete machine comes for 
free under some assumptions. The symbolic machine follows the structure of the 
basic machine but it's augmented to better match a PUMP architecture. 
Specifically the symbolic machine is parameterized by the following:
\begin{itemize}
\item A set of symbolic tags, used to tag the contents of the memory, the
registers and the pc.
\item A partial function \TRANSFER, that on every step checks whether the
step is allowed according to opcode of the instruction executed and the tags on
it. In the case it's allowed it returns a tag for the new pc and a tag for any 
resulting data from executing the instruction.
\item A partial function \textit{get\_service}, mapping addresses to 
\textit{symbolic monitor services}. In the symbolic machine, monitor services
are represented as a tuple of a partial function on machine states and a
symbolic tag.
\item An internal machine state with an initial value, that can be used by 
monitor services.
\end{itemize}

The states of the symbolic machine consists of a memory, registers, a \pc 
register and an internal state.
The memory  and register contents, as well as the \pc, are all tagged with a
symbolic tag \textit{t}. We name their contents \textit{symbolic atoms} referred
to with the notation \atom{\ii{w}}{\ii{t}}, where \ii{w} is the value (word) and
\ii{t} is the tag.

At each step, a record named \emph{mvector} is formed. It consists of the 
current opcode, the tag on the \pc, the tag on the current instruction and 
optionally up to three tags depending on the opcode of the instruction.
The \emph{mvector} is passed to the transfer function
which decides whether the step violated the policy enforced by the \TRANSFER
function and in this case halts the machine, or if no violation occurred returns
a tag for the new \pc and a tag for any results the instruction execution 
produced.

We write, in form of inference rules, the stepping relation for the Store and 
Jump instructions, in order to demonstrate the above mechanism. The complete
definition of the stepping relation can be found at \TODO{cite appendix}

\begin{figure}[!htpb]
\infrule[Store]{
  \mem[\pc] = \atom{i}{\ti} \andalso \ii{decode}~i = \ii{Store}~r_p~r_s  \\
  \rd{\reg}{r_p} {=} \atom{w_p}{t_p} \andalso
  \rd{\reg}{r_s} {=} \atom{w_s}{t_s} \andalso
  \rd{\mem}{w_p} {=} \atom{w\old}{t\old} \\
  \handler{\ii{Store}}{\tpc}{\ti}{t_p}{t_s}{t\old} {\tpc'}{t_d'} \\
  \mem' = \upd{\mem}{w_p}{w_s@t_d'}
  }{\step{\astat{\mem}{\reg}{\atom{\pc}{\tpc}}{int}}
  {\astat{\mem'}{\reg}{\atom{\pc + 1}{\tpc'}}{int}}
  }
\bigskip

\infrule[Jump]{
  \mem[\pc] = \atom{i}{\ti} \andalso \ii{decode}~i = \ii{Jump}~r \andalso
  \rd{\reg}{r} {=} \atom{w}{t_w} \\
  \handler{\ii{Jump}}{\tpc}{\ti}{t_w}{-}{-} {\tpc'}{-}
  }{\step{\astat{\mem}{\reg}{\atom{\pc}{\tpc}}{int}}
  {\astat{\mem}{\reg}{\atom{w}{\tpc'}}{int}}
  }
\caption{Symbolic stepping relation for Store and Jump}
\end{figure}

Notice that when a store instruction executed, the tag on the memory location to
be overwritten is fetched, allowing the \TRANSFER function to know what kind of
data we are trying to overwrite.

\section{A Programmable Unit for Metadata Processing}\label{sec:pump}

\ch{Could consider moving this one level up (turn it into chapter)}

\subsection{Hardware Architecture}

The Programmable Unit for Metadata Processing (PUMP) architecture
\cite{pump_hasp2014}
allows us to efficiently implement a wide range of security policies 
\cite{pump_ccs2014} by associating metadata to the data being processed 
(e.g., this is an instruction, this is from the network, this is private),
propagating the metadata as instructions are executed and using a rules-based 
system to check invariants on the metadata in parallel with the main computation.
Abstractly, the tag propagation rules form a partial function from a set of 
input tags to a set of output tags
$$(opcode, tag_{pc},tag_{instr}, tag_{arg1}, tag_{arg2}, tag_{arg3})
\nrightarrow (tag_{pc'},tag_{result})$$
informally read as, ``if the next instruction to be executed is opcode, the 
current tag of the program counter is $pc_{tag}$, the current tag on the 
instruction location is $tag_{instr}$ and the tags on the operands of the 
instruction are $tag_{arg1}, tag_{arg2}$ and $tag_{arg3}$ then if execution of 
the instruction is allowed the tag on the program counter should be set
to $tag_{pc'}$ and any new data created by the instruction should be tagged 
$tag_{result}$''.

On the hardware level, the PUMP is an extension to a conventional RISC 
architecture. Every word of data in the machine - whether in memory 
or a register, is extended with a word-sized metadata tag.
These tags are not interpreted by hardware, instead the interpretation of the 
tags is left to the software, thus making it easy to implement new policies on 
the metadata. Since tags are word-sized, they can be pointers to complex 
data-structures of tags, such as tuples of tags, allowing for complex policies 
to be expressed and multiple orthogonal policies to be enforced in parallel.

The hardware undertakes the correct propagation of tags from operands to results 
according to the rules defined by the software. 
A hardware rule cache mapping sets of input tags to sets of output tags is used 
for common case efficiency. On each instruction dispatch, in parallel 
with the usual behavior of an instruction 
(\EG execution of an addition in the ALU), the hardware forms the set of input 
tags and a lookup is performed on the rule cache. If the lookup is successful
a set of output tags is returned and combined with the results of the normal 
execution of the  instruction a new state is produced. On the other hand, 
if the lookup failed, the hardware invokes a trusted piece of system software - 
the fault handler - which checks the input tags and decides whether the 
execution should be allowed or not. In the first case, the fault handler returns
a set of result tags, a pair of set of input and output tags is formed and
inserted into the rules cache, while the faulting instruction is restarted 
and will now hit the cache. Otherwise, execution of this instruction violated 
some rules of the enforced policy and execution should not continue normally 
(\EG should be halted).

As described in the original PUMP paper by Dehon \ETAL \cite{pump_hasp2014} and 
in more detail in the follow-up \cite{pump_ccs2014} a rich set of effective 
security policies can be efficiently implemented using the architecture 
mentioned above. In particular, implementations of dynamic typing, memory safety
for heap-based data, control flow integrity and taint tracking are described, 
evaluated against a specific threat model and benchmarked. The benchmarks are
done using a simulation of the described hardware and the two papers claim low
overhead (~10\% on average) for each of the policies named above.

Compared to other software solutions for enforcing security policies, the PUMP 
offers  significantly lower overhead, thanks to dedicated hardware assistance, 
while the fact that interpretation of the metadata is done by software offers 
flexibility with regard to the policies that can be implemented, compared to 
hardware solutions implementing a specific policy.

While the PUMP offers flexibility at a low runtime performance overhead, 
there are more overheads associated to such a mechanism. For example adding 
metadata to all the data in the machine, would result in a 100\% memory overhead.
In addition, the extra hardware and the rule cache along with potentially larger
memories could result into a 400\% overhead on energy usage. \cite{pump_ccs2014}
The authors claim that a careful and well-optimized implementation can reduce 
these numbers, resulting in a 50\% energy overhead.
%
\FEEDBACK{Cite ASPLOS instead of CCS; use optimized numbers}

\subsection{Concrete Machine}\label{sec:concrete}

The concrete machine is a model of the basic machine with PUMP hardware, 
in particular a rules \cache and a software \emph{miss handler}. 
The instruction set has been extended with four additional instructions that 
are meant to be used by monitor code only, a restriction enforced by the monitor
self-protection mechanism.

The states of the concrete machine consists of a memory, registers, a \pc 
register, an \epc register a special purpose register that holds the address of
the faulting instruction after a cache miss and a cache.
The cache works as a key-value store where a key is an \emph{input vector} that
contains an instruction opcode, the concrete tag of the current instruction,
the concrete tag of the pc and up to three operand tags, and a value is an 
\emph{output vector} which contain a tag for the new pc and a tag for any
results from the execution of the instruction. Intuitively a concrete tag is the
encoding into a word of a symbolic tag. 
Lifting this encoding relation to vectors, we get that a concrete vector is the
encoding of a symbolic vector (\emph{mvector}). 
In accordance \FEEDBACK{em this sucks? does this word even exist? think about smth else?} to the symbolic machine
the contents of the memory, the registers, the pc and the epc are concrete atoms
\atom{w}{t} where w is a word and t is the encoding of a tag into a word.

The stepping relation for the concrete machine is a bit more complicated than
the one for the symbolic machine. In particular, on each step the machine forms
the \emph{input vector} and looks it up in the cache. If the lookup succeeds 
then the instruction is allowed, a \emph{output vector} is returned by the
cache and the next state is tagged according to it. 
If the lookup fails, then the \emph{input vector} is saved in memory, the 
current \pc is stored in \epc and the machine traps to the \emph{miss handler}.
The above are demonstrated in the two example rules below:

\TODO{put example rules}

Addresses 0 to 5 are used to store the \emph{input vector} and 6 to 7 are used
by the miss handler to store the \emph{output vector}. As a side-note, cache 
eviction is not modeled (an infinite cache is assumed).

\subsection{Concrete Policy Monitor}\label{sec:concrete_policy}

Unlike the symbolic machine, where the user cannot cannot change the 
\TRANSFER function, enforcing a micro-policy on the concrete machine requires
that we are able to protect the memory of the policy monitor and that privileged
instructions are not executed by user code. This self-protection policy can be 
easily composed with another micro-policy and enforced by the infrastructure
described above. 

Using tags of the form, \USER{\ii{st}}, \ENTRY{\ii{st}}, \MONITOR we can 
distinguish between user memory, monitor memory and monitor services. 
In particular \USER{\ii{st}} is used to tag a user-level atom, where \ii{st} is
the word-encoding of a symbolic tag. \MONITOR is used to tag the monitor memory
and a few reserved registers. The \pc is tagged with \MONITOR when a monitor
execution takes place and \USER{\ii{st}} when user-code is executed. The tag
\ENTRY{\ii{st}} is used to tag the first instruction of a monitor service and 
serves as an indication that execution will continue under the privileged
\MONITOR mode. 

The miss handler is a composed policy monitor that protects itself from
\USERname code and that enforces a desired micro-policy.
One important thing to note is that the miss handler for the concrete machine
can take an arbitrary number of steps before deciding that no violation occurred
and returning to \USERname  mode, unlike the symbolic \TRANSFER function that
does not need to take any steps.


\chapter{Control-Flow Integrity}\label{ch:cfi}

Restricting the control-flow of a program in some way is a technique widely
spread among security researchers. For example non-executable data (NXD)
can be considered as a form of (very) coarse-grained \CFI where control-flow is
not allowed to reach any memory region that holds non-executable data. Other
mitigation techniques such as protecting return addresses on the stack enforce
a form of coarse-grained \CFI.

Moreover it is common that security properties are enforced dynamically by code
that is statically injected to the program (\EG Inlined Reference Monitors (IRM)
\cite{Erlingsson04} follow that approach), thus some form of \CFI is required in
order to ensure that these checks are not circumvented. 

\TODO{Think about title}
\section{Balancing between performance and security}\label{sec:security_cfi}

Abadi \ETAL first proposed a technique to enforce \CFI based on IRMs. 
In particular, they proposed to mark all valid targets of \emph{indirect} 
control transfers with a unique identifier and inject checks before all 
indirect jumps (including return instructions). However they assume that any 
two destinations are equivalent, in the sense that they share the same 
identifier, if the CFG contains edges from the same  set of sources, which may
significantly reduce the precision of the CFG. 
The authors also note that a 2-ID approach where one identifier is used for 
calls and another for returns could provide adequate security in many cases. 

The work of Abadi \ETAL sparked interest of researchers who tried to improve
some of the weaknesses of the initial implementation, usually by choosing 
between performance against precision and vice-versa.

Bletsch \ETAL \cite{Bletsch:2011:MCA:2076732.2076783} followed the work
of Abadi \ETAL, but changed their checking mechanism to perform the check
after the control flow transfer has occurred which, as the authors claim,
reduced the cache pressure and resulted in better performance. Precision remains
the same with the implementation of Abadi \ETAL.

Zhang \ETAL \cite{Zhang2013} proposed \emph{Compact Control Flow Integrity
and Randomization} (CCFIR), a new efficient way to enforce coarse-grained \CFI.
CCFIR collects all valid targets of indirect control-transfers and stores them
in a random order, in a protected section called ``Springboard section''. 
Indirect control-transfers are only allowed to addresses that are in the
Springboard. Their implementation uses a 3-ID approach where one identifier is
used for calls and the two other identifiers are for returns, separating them
between returns to sensitive and non-sensitive functions. Their implementation
also supports interaction between protected and un-protected modules, which
makes it an attractive solution to coarse-grained \CFI.

The above techniques are evaluated in \cite{outofcontrol_ieeesp2014} where
the authors demonstrate code-reuse attacks against binaries protected by
coarse-grained \CFI. These attacks illustrate the need for fine-grained
\CFI which however incurs a high runtime-overhead penalty making deployment
of such a mechanism unlikely.

\subsection{Standard assumptions for effective \CFI}\label{sec:cfi_assumptions}

Most -if not all- \CFI implementations also come with a set of assumptions under
which \CFI holds. Two standard assumptions for all mechanisms that attempt to
enforce \CFI are:
\begin{itemize}
\item \emph{NXD} is an abbreviation for Non-Executable Data, a security
mechanism that disallows execution of data. 
\item \emph{NWC} stands for Non-Writable Code. Changing the code of a
program would allow an attacker to circumvent dynamic checks.
\end{itemize}

Both assumptions are fairly standard for modern computers and are enforced
through hardware or software. In some cases \emph{NXD} can be lifted, but
additional security risks and complexity is not worth the minor advantages
offered by such an action.

Many implementations that attempt to do fine-grained \CFI also require that
identifiers used to mark nodes in the CFG are unique.

\section{Formal verification of Control-Flow Integrity}\label{sec:cfi_verif}

In \cite{AbadiBEL09} Abadi \ETAL extended their original paper, with
-among other things- a more detailed formal study of \CFI. Their formalization
regarded a much simpler machine than the x86 omitting all the complexity in
modern systems. The machine has a few instructions, a separate data memory and
instruction memory which by the operational semantics of the machine are 
non-executable and non-writable (enforcing \NXD and \NWC by construction), and a
small set of registers.
Moreover, their attacker model permits arbitrary changes to the data memory,
arbitrary changes to all the registers but a few distinguished ones that are
used during the dynamic checks and no changes to the instruction memory.
The authors proof that under some assumptions \CFI is preserved for every step
even in the presence of an attacker as powerful as the one described above. 
Their formal study served as a guideline for the implementation, but as it is 
done on paper their proofs cannot be machine checked. Furthermore, their 
formalization omits less interesting but important details such as instruction 
encoding and decoding which as shown in \cite{MorrisettTTTG12} are far from 
trivial for the x86.

Machine-checked formal verification efforts include \cite{ZhaoLSR11}, which is
a SFI formalization for the ARM architecture that also enforces \CFI.
Their formalization was developed using the HOL theorem prover and a program
logic framework they created. However their benchmarks report a 240\% runtime
overhead. The authors of \cite{CriswellDA14} claim partial proofs for a \CFI
enforcement mechanism focused on the kernel of an operating system. Their
runtime overhead can also reach 100\%.

\section{Control-Flow Integrity over PUMP}\label{sec:cfi_pump}

The PUMP hardware allows us to avoid taking the difficult decision between
performance and security. As shown in \cite{pump_ccs2014}, we can enforce a
\emph{fine-grained} \CFI policy with an average overhead of 8\%.
\TODO{Is this number right?}

In our design, we take the standard approach and claim \CFI under \NXD and
\NWC. We considered designs that lifted these assumptions but
we rejected them, for the time being, as there did not seem to be any
considerable advantage \IE{compatibility with self-modifying programs,
JIT compilers, \ETC} Allowing the code of the program to change, would in
practice require for the CFG to change as well, which unless done in
a controlled, ``safe'' way, would invalidate the enforcement of \CFI.
However, we do not have to rely on special hardware or software to enforce
\NXD and \NWC. We can achieve this easily and efficiently by
creating a separate micro-policy.

\subsection{Enforcing Non-Writable Code \& Non-Executable Data}
\label{sec:nwc_nxd}

Consider the set of tags \TAGS{\DATA,\INSTRname}. If we initially tag
all executable regions in memory as \INSTR{} and all non-executable as \DATAname
then we can enforce \NWC by two rules of the form

\begin{figure}[!htpb]
\begin{align*}
 & (Store, \_,\_, \_, \_,\INSTR) \rightarrow \varnothing \\
 & (Store, \_,\_, \_, \_,\DATA) \rightarrow (\_, \DATA)
\end{align*}
\caption{Rules enforcing \NWC}
\end{figure}

The \_ in the vectors, represent \textit{don't care} values. In the context of
the input vector their behavior is the same as \textit{don't care} values in
match expressions in ML languages. In the context of the output vector it just
captures the intuition that we will not really use the result tags, so anything
could be returned as a result tag (\IE \DATAname or we can copy-through tags
from the input vector).
Informally the above rules reads as ``If the current opcode is Store and the
content of the memory location we are trying to write is tagged \INSTR{}
then the memory write is not allowed. Otherwise if it is tagged \DATAname then
the write is permitted and the new value will also be tagged \DATA.''

We can enforce \NXD in a similar fashion
\begin{figure}[!htpb]
\begin{align*}
 & (-, -,\DATA, -, -, -) \rightarrow \varnothing \\
 & (-, -,\INSTR, -, -,-) \rightarrow (-, -)
\end{align*}
\caption{Rules enforcing \NXD}
\end{figure}

Informally the above rules reads as ``If the tag on the current instruction is
\DATAname then execution is not allowed. Otherwise if it is \INSTRname then
execution is allowed''.

\FEEDBACK{Used \_ and -, I think the second one looks better, opinions?}\\
\TODO{Perhaps explain what each tag means for each opcode earlier -- or maybe 
just in appendix?}\\
\FEEDBACK{These tuple-vectors make it hard for people not familiar with them
to remember what each field is, any better ways to represent them?}

\subsection{Enforcing Control-Flow Integrity}\label{sec:cfi_enforce}

\subsubsection{Coarse-grained Control-Flow Integrity}\label{sec:cfi_coarse}
We can use the PUMP to implement the coarse-grained \CFI mechanisms described
earlier. Suppose we want to implement 1-ID \CFI, we tag all indirect flow 
destinations and sources with a tag \MARK and the rest of the instructions as 
\UNMARK. Executing instructions that are sources of indirect flows, propagates
their instruction tag to the \pc. We then have to check that the tag on the
destination matches the tag on the tag on the \pc.

\begin{figure}[!htpb]
\begin{align}
 & (Jump/Jal, -,\MARK, -, -, -) \rightarrow (\MARK,-) \label{coarse_rule1} 
\tag{1} \\
 & (-, \MARK, \MARK, -, -, -) \rightarrow (\UNMARK, -) \label{coarse_rule2}
\tag{2} \\
 & (-, \MARK, \UNMARK, -, -, -) \rightarrow \varnothing \label{coarse_rule3} 
\tag{3}
\end{align}
\caption{Rules enforcing coarse-grained \CFI}
\end{figure}

\TODO{align all elements of the rules above}

Rule \ref{coarse_rule1} is used in the case the opcode is Jump or Jal (the only 
indirect jumps in the RISC machine we examine) and propagates the \MARKname tag on
the tag of the new \pc. Rule \ref{coarse_rule2} applies when the tag on the \pc
is set to \MARKname and corresponds to a legal destination and rule
\ref{coarse_rule3} corresponds to an illegal destination (\IE one that is 
tagged \UNMARK) and is not allowed.

We do not further study this coarse-grained approach as we consider it 
ineffective since attacks against it has already been demonstrated in 
\cite{outofcontrol_ieeesp2014}. Instead we are going to focus on implementing
and formalizing a fine-grained \CFI micro-policy.

\subsubsection{Fine-grained Control-Flow Integrity}\label{sec:cfi_fine}

The micro-policy we implemented and studied is a composition of a fine-grained
\CFI micro-policy and the \NWC, \NXD micro-policies explained above.

Our approach uses unique identifiers to tag the contents of the memory that 
correspond to sources and potential destinations of indirect flows according to
a binary relation (on the identifiers) $\mathcal{CFG}$.

Consider the set of tags 
\TAGS{\DATA,\INSTR{\ii{id}}, \INSTR{\bot}} where \ii{id} is a unique identifier 
(\IE used to tag the contents of only one location in the memory). 
Adapting the rules from \ref{sec:nwc_nxd}, we shall use \DATAname to tag all 
contents in memory that are considered non-executable data, \INSTR{\ii{id}}
to tag all contents in memory that are considered executable instructions and 
are sources or targets of indirect control flows and \INSTR{$\bot$} to tag all
other instructions.
The rules to enforce \NWC and \NXD are intuitively the same and only
change to account for the splitting of the \INSTRname tag.

We follow the same idea as with coarse-grained \CFI, propagating the instruction
tag of instructions that are sources of indirect flows to the tag on the \pc of
the next state and upon execution of the next instruction, checking that the tag
on the \pc and on the instruction are in some relation. In the case of
coarse-grained \CFI we required that they match but for fine-grained \CFI we
require that they are in the \CFG relation.

% \begin{figure}[!htpb]
% \begin{align}
%  & (Jump/Jal, -,-, -, -, -) \rightarrow (- ,-) 
%  \text{ if (n,m) } \in \CFGm \label{fine_rule1} \tag{1} \\
%  & (Jump/Jal, \DATA, \INSTR{m}, -, -, -) \rightarrow (\INSTR{m},-) 
%  \text{ if (n,m) } \in \CFGm \label{fine_rule2} \tag{2} \\
%  & (Store, \DATA, \INSTR{\_}, -, -, \DATA) \rightarrow (\DATA,\DATA) 
%  \label{fine_rule3} \tag{3} \\
%  & (Store, \INSTR{n}, \INSTR{m}, -, -, \DATA) \rightarrow (\DATA,\DATA)
%  \text{ if (n,m) } \in \CFGm \label{fine_rule4} \tag{4} \\
%  & (Store, -, -, -, -, \INSTR{\_}) \rightarrow \varnothing
%   \label{fine_rule5} \tag{5} \\
%  & (-, \INSTR{n}, \INSTR{m}, -, -, -) \rightarrow (\DATA,-)
%  \text{ if (n,m) } \in \CFGm \label{fine_rule6} \tag{6} \\
%  & (-, \DATA, \INSTR{m}, -, -, -) \rightarrow (\DATA,-) \label{fine_rule7} 
%  \tag{7}\\
%  & (-, -, -, -, -, -) \rightarrow \varnothing \label{fine_rule8} 
%  \tag{8}
% \end{align}
% \caption{Rules enforcing fine-grained \CFI, \NXD and \NWC}
% \end{figure}

\begin{figure}[!htpb]
\infrule[Flow/Check]{
  opcode \in \lbrace Jump, Jal \rbrace \andalso 
  (src,dst) \in \CFGm
  }{
  \frule{opcode}{\INSTR{src}}{\INSTR{dst}}{-}{-}{-} {\INSTR{dst}}{-}
  }
\bigskip

\infrule[Flow/NoCheck]{
  opcode \in \lbrace Jump, Jal \rbrace
  }{
  \frule{opcode}{\DATA}{\INSTR{dst}}{-}{-}{-} {\INSTR{dst}}{-}
  }
\bigskip

\infrule[Store/Check]{
  (src,dst) \in \CFGm
  }{
  \frule{Store}{\INSTR{src}}{\INSTR{dst}}{-}{-}{\DATA} {\DATA}{\DATA}
  }
\bigskip

\infrule[Store/NoCheck]{
  ti \in \lbrace \INSTR{dst}, \INSTR{\bot} \rbrace
  }{
  \frule{Store}{\DATA}{ti}{-}{-}{\DATA} {\DATA}{\DATA}
  }
\bigskip

\infrule[Rest/Check]{
  opcode \not \in \lbrace Jump, Jal, Store \rbrace \andalso
  (src,dst) \in \CFGm
  }{
  \frule{opcode}{\INSTR{src}}{\INSTR{dst}}{-}{-}{-} {\DATA}{-}
  }
\bigskip

\infrule[Rest/NoCheck]{
  opcode \not \in \lbrace Jump, Jal, Store \rbrace
  ti \in \lbrace \INSTR{dst}, \INSTR{\bot} \rbrace
  }{
  \frule{opcode}{\DATA}{ti}{-}{-}{-} {\DATA}{-}
  }
\caption{Rules enforcing fine-grained \CFI, \NXD and \NWC}
\end{figure}

We note in the above rules that the tag on the \PCname is \DATAname when
no check for a control-flow violation is required and \INSTR{\textit{src}} where
\textit{src} is some id, when an indirect flow instruction was executed and a
check for a control-flow violation is required. An important observation is that
the rules above allow for one control-flow violation to occur, but disallow the
next step and therefore the machine will certainly halt after a violation.

If the PUMP hardware fetched the tag on the memory address the machine is
jumping to and passed it as an argument to input vector, as it does in the
case of a Store instruction, we would be able to enforce \CFI with no violations
at all. \TODO{It can't do that for efficiency reasons?}







\chapter{Formally Verified Control-Flow Integrity Micro-Policy}\label{ch:verified_cfi}

Using the micro-policies framework described in \ref{sec:micropolicies} we 
proved that the concrete machine instantiated with a \CFI micro-policy like the
one described in \ref{sec:cfi_fine} \ii{simulates} an abstract machine that has
\CFI by construction.

Additionaly, we provide an attacker model for all the machines used and we prove
that a property capturing the notion of \CFI holds even when the attacker
tampers with the machine, similarly to what is proposed in \cite{abadi2005}, but
adapted to the setting of our machines.

\section{Expressing the control-flow through tags}\label{sec:cfi_tags}

Our approach for enforcing \CFI, as explained in \ref{sec:cfi_fine}, requires
that we encode the nodes in the control-flow graph in terms of identifiers,
which in turn are used to tag all sources and targets of indirect control-flows.

At this point we take a detour, to point out an important design point of 
the micro-policies framework and our \CFI micro-policy.
Throughout both developments, a heavily parametric and modular approach was 
taken. This parametric design is enabled by the use of the \emph{Section} and
\emph{Type Classes} mechanisms of Coq. As an example, the node identifiers,
along with a number of properties we require of them are expressed by the
following interface (defined in terms of a type class):

\begin{figure}[!htpb] 
\begin{coqdoccode}
\coqdocnoindent
\coqdockw{Context} \{\coqdocvar{t} : \coqdocvar{machine\_types}\}.\coqdoceol
\coqdocemptyline
\coqdocnoindent
\coqdockw{Class} \coqdocvar{cfi\_id} := \{\coqdoceol
\coqdocindent{1.00em}
\coqdocvar{id}         : \coqdocvar{eqType};\coqdoceol
\coqdocnoindent
\coqdoceol
\coqdocindent{1.00em}
\coqdocvar{word\_to\_id} : \coqdocvar{word} \coqdocvar{t} \ensuremath{\rightarrow} \coqdocvar{option} \coqdocvar{id};\coqdoceol
\coqdocindent{1.00em}
\coqdocvar{id\_to\_word} : \coqdocvar{id} \ensuremath{\rightarrow} \coqdocvar{word} \coqdocvar{t};\coqdoceol
\coqdocindent{1.00em}
\coqdoceol
\coqdocindent{1.00em}
\coqdocvar{id\_to\_wordK} : \coqdockw{\ensuremath{\forall}} \coqdocvar{x}, \coqdocvar{word\_to\_id} (\coqdocvar{id\_to\_word} \coqdocvar{x}) = \coqdocvar{Some} \coqdocvar{x};\coqdoceol
\coqdocindent{1.00em}
\coqdocvar{word\_to\_idK} : \coqdockw{\ensuremath{\forall}} \coqdocvar{w} \coqdocvar{x}, \coqdocvar{word\_to\_id} \coqdocvar{w} = \coqdocvar{Some} \coqdocvar{x} \ensuremath{\rightarrow} \coqdocvar{id\_to\_word} \coqdocvar{x} = \coqdocvar{w}\coqdoceol
\coqdocindent{14.50em}
\coqdoceol
\coqdocnoindent
\}.\coqdoceol
\end{coqdoccode}
\caption{Interface of node identifiers}
\label{fig:id_class}
\end{figure}

The \emph{Context} command on the top of the code above, allows us to
\textit{assume} that there exists an instance of this interface.
In fact, the \emph{machine\_types} argument is just another type class,
serving as a specification of the various types of the machine (\EG the word 
size). This approach allowed us to abstract away from insignificant details and
structure our proofs in a clean way.
In addition, we can easily instantiate a different machine with minimal changes
in our proofs and definitions (\EG instantiate the machine with a different word
size).

However, one drawback is that one wrong specification in a type class would
disallow us to instantiate it and would require that we go back and change
all parts that used this wrong specification (\EG in our case, the 
\emph{machine\_types} class was widely used). Therefore one should be careful
when doing heavy use of such mechanisms.

Returning to the identifiers, looking at the definition in \ref{fig:id_class},
we require that the type of the identifiers \id is an Eqtype (has decidable
boolean equality) and that there exists conversion functions between elements
of type \word and \id, satisfying some constraints. 

\QUESTION{Should I give some intuition, as to why word\_to\_id is partial? Is it
obvious?}

\section{A Theorem About Control-Flow Integrity}\label{sec:cfi_property}

Our formalization includes a definition of \CFI, similar to the one found in
\cite{abadi2005}, which we prove to be true of all our machines. The need for a
new definition arises from fundamental differences between our enforcement
mechanism on the concrete mechanism and the one used by Abadi \ETAL. In
particular, our enforcement-mechanism does not prevent a violation, instead
it can detect it after it has occurred by taking an arbitrary number of
``protected'' (monitor mode) steps before eventually bringing the machine to a
halt. This does not have any impact on the security effectiveness of our
mechanism, it does however lead to a more complex definition and therefore
more complex proofs.

As mentioned in \ref{sec:cfi_fine}, we check for violations of the control-flow
with respect to a binary relation (on the identifiers) \CFG which represents
the set of allowed (indirect) jumps. We can extend this relation to precisely
describe the control-flow of a program, by lifting \CFG to a relation 
\SUCC on machine states, that includes the set of allowed targets for the rest
of the instructions. 

The definition of \CFI is further parameterized by an attacker model. We
model the attacker as a step relation (\stepa{}{}). Intuitively the attacker is
allowed to change any \emph{user-level} data but not the code of the program and
the \pc, as well as the tags in the case of a tagged machine. 
This limitations ensures that an attacker cannot directly circumvent the monitor
protection mechanism and our user-level policies (\NWC , \NXD and \CFI). To 
account for attacker steps, the stepping relation is extended as the union of 
the normal step relation (\stepn{}{}), as defined by the machine semantics, and
the attacker step relation (\stepa{}{}), as defined by the attacker model.

\begin{figure}[ht]
\centering
\begin{minipage}[b]{0.25\linewidth}
\centering
\infrule[]{\stepn{s}{s'}
  }{\step{s}{s'}}
\label{fig:step_stepn}
\end{minipage}
\hspace{0.5cm}
\begin{minipage}[b]{0.15\linewidth}
\centering
\infrule[]{\stepa{s}{s'}
  }{\step{s}{s'}}
\label{fig:step_stepa}
\end{minipage}
\caption{Step relation definition}
\end{figure}

We define a predicate \INITIAL{s}, where s is a machine state, that
states that s is an initial state. We use this predicate to express some
invariants that are preserved through execution (\EG the initial tagging scheme
for the memory). Finally we define a stopping predicate on an execution trace
that states that the machine is coming to a halt with respect to normal steps.

Since we want to instantiate the above parameters in a different way for
each of our machines, it makes sense to wrap them in a type class which
we will instantiate for each machine to get the corresponding definition
of \CFI.

\begin{figure}[!htpb]
\begin{coqdoccode}
\coqdocnoindent
\coqdockw{Class} \coqdocvar{cfi\_machine} := \{\coqdoceol
\coqdocindent{1.00em}
\coqdocvar{state} : \coqdockw{Type};\coqdoceol
\coqdocindent{1.00em}
\coqdocvar{initial} : \coqdocvar{state} \ensuremath{\rightarrow} \coqdockw{Prop};\coqdoceol
\coqdocindent{1.00em}
\coqdoceol
\coqdocindent{1.00em}
\coqdocvar{step} : \coqdocvar{state} \ensuremath{\rightarrow} \coqdocvar{state} \ensuremath{\rightarrow} \coqdockw{Prop};\coqdoceol
\coqdocindent{1.00em}
\coqdocvar{step\_a} : \coqdocvar{state} \ensuremath{\rightarrow} \coqdocvar{state} \ensuremath{\rightarrow} \coqdockw{Prop};\coqdoceol
\coqdocnoindent
\coqdoceol
\coqdocindent{1.00em}
\coqdocvar{succ} : \coqdocvar{state} \ensuremath{\rightarrow} \coqdocvar{state} \ensuremath{\rightarrow} \coqdocvar{bool};\coqdoceol
\coqdocindent{1.00em}
\coqdocvar{stopping} : \coqdocvar{list} \coqdocvar{state} \ensuremath{\rightarrow} \coqdockw{Prop}\coqdoceol
\coqdocnoindent
\}.\coqdoceol
\end{coqdoccode}
\caption{Interface of a cfi\_machine}
\label{fig:cfi_machine}
\end{figure}

\QUESTION{Unsure about the caption on the above figure}

For a machine of type cfi\_machine we give the following definitions:

\begin{definition}\label{definition:traceHasCfi}
  We say that an execution trace $s_0 \to s_1 \to \ldots \to s_n$ {\em has CFI}
  if for all $ i \in [0,\ldots,n)$ if \stepn{s_i}{s_{i+1}} then
  $(s_i,s_{i+1}) \in$ \SUCC .
\end{definition}

\QUESTION{The word relation for succ and cfg is strange since they are
booleans, is it ok, or does it confuse you, making you believe they are Props?}

The above definition corresponds to the one found in \cite{abadi2005}, however
it is stronger in the sense that it requires that steps that are in the
intersection of normal and attacker steps respect the control-flow. If we did
not allow for any violations then the above definition would be enough, but
since our enforcement mechanism allows for one violation we have to resort to a
weaker definition.

\begin{definition}[CFI]\label{definition:CFI}
  We say that the machine
  $(\ii{State}, \ii{initial}, \to_n,\allowbreak \to_a, \ii{cfg}, \ii{stopping})$
  has \CFI with respect to the set of allowed indirect jumps \CFG
  if, for any execution starting from initial state $s_0$
  and producing a trace $s_0 \to \ldots \to s_n$, either
  \begin{enumerate}
  \item The whole trace has \CFI according to
    \autoref{definition:traceHasCfi}, or else
  \item There is some $i$ such that $s_i \to_n s_{i+1}$,
  and $(s_i, s_{i+1}) \not \in$ \SUCC, where
  the sub-traces $s_0 \to \ldots \to s_i$ and
  $s_{i+1} \to \ldots \to s_n$ both have CFI
  and the sub-trace $s_{i+1} \to \ldots \to s_n$ is stopping.
  \end{enumerate}
\end{definition}

\section{The Abstract Machine}\label{sec:abstract_cfi}

The abstract machine is based on the basic machine explained in \ref{sec:basic},
has \CFI, \NXD and \NWC by construction and will serve as a 
specification for the symbolic and eventually the concrete machine that
implement \CFI through the tag-based system explained in the previous chapter.

Unlike the symbolic and the concrete machine, this abstract machine splits the 
memory into two disjoint memories, an instruction memory and a data memory. The
instruction memory is fixed (non-writable) and the machine uses this memory to
fetch instructions to execute, so \NWC and \NXD are enforced by construction.

In addition the state of the machine includes an \ok bit, indicating 
whether a control-flow violation has occurred or not. The rest of the machine
state is completed by a set of registers and a \pc register. We use a 5-tuple
notation for the state \acfistat{\imem}{\dmem}{\reg}{\pc}{\ok}, where the first
field is the instruction memory, the second the data memory, the third the
registers, the fourth is the pc register and the fifth is the \ok bit.

\subsection{Operational semantics}\label{abstract_semantics}
Below is the step rule for the Store instruction, illustrating both \NWC and 
\NXD. Notice that the instruction is fetched by the instruction memory and
the store is done on the data memory.

\begin{figure}[!htpb]
\infrule[Store]{
  \imem[\pc] = i \andalso \ii{decode}~i = \ii{Store}~r_p~r_s \andalso
  \rd{\reg}{r_p} = p \\
  \rd{\reg}{r_s} = w \andalso
  \dmem' = \upd{\dmem}{p}{w}
  }{\step{\acfistat{\imem}{\dmem}{\reg}{\pc}{\ii{true}}}{
    \acfistat{\imem}{\dmem'}{\reg'}{\pc+1}{\ii{true}}}}
\caption{Step rule for Store instruction of abstract machine}
\end{figure}

In the above rule, the \ok bit is true for both the starting and the
resulting state. In fact, the machine can take a step only when the \ok
bit is set to true. In the above rule, the \ok bit is set to true in the
resulting state, indicating that no control-flow violation has happened, as
expected by the execution of a Store instruction. Control-flow violations in the
\NWC setting our machine is executing, can only occur from \emph{indirect} jump
instructions, in our case the Jump and Jal instructions. Upon execution of a
Jump or Jal instruction, a function \J, which represents the set of allowed
jumps, checks whether the change of control-flow is legal. If the jump is not 
allowed according to \J then the jump is taken but the \ok bit is set to false, 
which will halt the machine in the next step as it is only allowed to step when
the \ok bit is set to true.

\begin{figure}[!htpb]
\infrule[Jal]{
  \imem[\pc] = i \andalso \ii{decode}~i = \ii{Jal}~r \andalso
  \rd{\reg}{r} = \pc' \\
  \reg' = \upd{\reg}{\ra}{pc+1} \andalso
  \ii{ok} = (\pc,\pc') \in \Jm
  }{\step{\acfistat{\imem}{\dmem}{\reg}{\pc}{\ii{true}}}{
    \acfistat{\imem}{\dmem}{\reg'}{\pc'}{\ii{ok}}}}
\bigskip

\infrule[Jump]{
  \imem[\pc] = i \andalso \ii{decode}~i = \ii{Jump}~r \andalso
  \rd{\reg}{r} = \pc' \andalso
  \ii{ok} = (\pc,\pc') \in \Jm
  }{\step{\acfistat{\imem}{\dmem}{\reg}{\pc}{\ii{true}}}{
    \acfistat{\imem}{\dmem}{\reg'}{\pc'}{\ii{ok}}}}
\caption{Step rule for Jump and Jal instruction of abstract machine}
\end{figure}

As the abstract machine serves as a specification to a machine with \CFI, a more
intuitive definition of it would not include the \ok bit and would only allow
the Jump and Jal instructions to step if they do not violate the control-flow
graph. However, this abstract machine would not allow for any violations to
occur unlike our enforcement mechanism for the symbolic and the concrete machine
and would lead to more complex simulation proofs, therefore we do not favor it.

The abstract machine also allows for monitor services to be included, although
the \CFI enforcement mechanism does not require any. We assume that a monitor
service is a privileged action and that it's execution does not violate the
control-flow of the program. Execution of a monitor service is done simply by
jumping to it's address, there is no separate instruction. Th
As with all other instructions, execution of the monitor service is only allowed
if the \ok bit is set to true.

\begin{figure}[!htpb]
\infrule[Service]{
  \pc \not\in \ii{dom(\imem)} \andalso \pc \not\in \ii{dom(\dmem)} \andalso
  \ii{get\_service}~\pc = (addr,f) \\
  f~\acfistat{\imem}{\dmem}{\reg}{\pc}{\ii{true}}=
    \acfistat{\imem}{\dmem'}{\reg'}{\pc'}{\ii{true}}
  }{\stepn{\acfistat{\imem}{\dmem}{\reg}{\pc}{\ii{true}}}{
    \acfistat{\imem}{\dmem'}{\reg'}{\pc'}{\ii{true}}}}
\caption{Step rule for monitor services of abstract machine}
\end{figure}

\subsection{Proving \CFI for the abstract machine}\label{abstract_proof}

\subsubsection{Attacker model}\label{sec:abstract_attacker}

The attacker for the abstract machine is allowed to change the
contents of the data memory and the registers but not the rest of the state.

\begin{figure}[!htpb]
\infrule{
  \ii{dom}~\dmem = \ii{dom}~\dmem' \andalso
  \ii{dom}~\reg = \ii{dom}~\reg'
}{
  \acfistat{\imem}{\dmem}{\reg}{\pc}{\ok} \to_a^A
  \acfistat{\imem}{\dmem'}{\reg'}{\pc}{\ok}
}
\caption{Attacker model for the abstract machine}
\end{figure}

\subsubsection{Legal control-flows for the abstract machine}
\label{sec:abstract_flow}

Assuming a function \CFG that represents the set of allowed (indirect) jumps,
we can construct a function \SUCC for the abstract machine that represents
the set of allowed control-flows for all instructions.

Below we give a specification of the \SUCC function for the abstract machine,
in form of inference rules. A function is defined in the actual Coq development.

\begin{figure}[!htpb]
\infrule[IndirectFlows]{
  \imem[\pc] = i 
  \andalso \ii{decode}~i \in \lbrace \ii{Jal~r}, \ii{Jump~r} \rbrace
  \andalso (\pc,\pc') \in \CFGm
  }{
  (\acfistat{\imem}{\dmem}{\reg}{\pc}{\ok} ,
  \acfistat{\imem}{\dmem'}{\reg'}{\pc'}{\ok}) \in \SUCCm}
\bigskip

\infrule[ConditionalFlows]{
  \imem[\pc] = i 
  \andalso \ii{decode}~i = \ii{Bnz~r~imm}\\
   (\pc' = \pc + 1) \vee (\pc' = \pc + imm)
  }{
  (\acfistat{\imem}{\dmem}{\reg}{\pc}{\ok} ,
  \acfistat{\imem}{\dmem'}{\reg'}{\pc'}{\ok}) \in \SUCCm}
\bigskip

\infrule[NormalFlows]{
  \imem[\pc] = i 
  \andalso 
  \ii{decode}~i \not\in \lbrace \ii{Jal~r}, \ii{Jump~r}, \ii{Bnz~r~imm}, 
  \varnothing \rbrace
  \\ \pc' = \pc + 1
  }{
  (\acfistat{\imem}{\dmem}{\reg}{\pc}{\ok} ,
  \acfistat{\imem}{\dmem'}{\reg'}{\pc'}{\ok}) \in \SUCCm}
\bigskip

\infrule[ServiceFlows]{
  \imem[\pc] = \varnothing \andalso \dmem[\pc] = \varnothing\\
  \ii{get\_service}~\pc = (addr,f)
  }{
  (\acfistat{\imem}{\dmem}{\reg}{\pc}{\ok} ,
  \acfistat{\imem}{\dmem'}{\reg'}{\pc'}{\ok}) \in \SUCCm}
\caption{Legal control-flows for instructions of the abstract machine}
\end{figure}

Notice that a monitor service is allowed to return anywhere. As we mentioned
before, monitor services, execute in a protected environment and we assume them
to be secure.

\subsubsection{Stopping predicate for the abstract machine}
\label{sec:abstract_stopping}

Finally, we define what it means for the abstract machine to be ``stopping'' by
defining a predicate on execution traces:
\begin{enumerate}
\item All states in the trace are stuck with respect to normal steps (\stepn{}{})
\item All steps in the trace are attacker steps (\stepa{}{})
\end{enumerate}

\subsubsection{Abstract machine as a \CFI machine}\label{abstract_cfi}

Regarding initial states, we only require that the \ok bit is set to true.
We can now instantiate the class of the machines defined in 
\ref{fig:cfi_machine}, with the abstract machine and then prove the following
theorem.

\begin{thm}[Abstract CFI]\label{thm:CFIabstract}
The abstract machine has the \CFI property defined by \ref{definition:CFI}.
\end{thm}

\begin{proof}
The proof procceeds by induction on the execution trace.
\end{proof}
\QUESTION{Should I write some proofs informally?}


\section{The Symbolic  Machine}\label{sec:symbolic_cfi}

The symbolic machine was described in \ref{sec:symbolic}. Unlike the abstract
machine, the symbolic machine has one memory and distinction between data
and executable instructions is made through tags, in a similar fashion to what
was shown in \ref{nwc_nxd} and \ref{cfi_fine}. We instantiate the symbolic
machine, according to the aforementioned sections, with a set of tags
\TAGS{\DATA,\INSTR{\ii{id}}, \INSTR{\bot}}, where id now is drawn from the class
of identifiers \ref{fig:cfi_id}. We do not have any monitor services and
we do not need any internal state for this micro-policy therefore, only the
transfer function is left to implement.

\subsection{Transfer Function}\label{sec:transfer_fun}

We implement the \TRANSFER function based on the rules found in
\ref{sec:cfi_fine}, using Gallina to define a function mapping
input vectors (mvector) to output vectors (rvector).
\pagebreak

\begin{figure}[!htpb]
\begin{coqdoccode}
\coqdocnoindent
\coqdockw{Definition} \coqdocvar{cfi\_handler} \coqdocvar{umvec} :=\coqdoceol
\coqdocindent{1.00em}
\coqdockw{match} \coqdocvar{umvec} \coqdockw{with}\coqdoceol
\coqdocindent{1.00em}
\ensuremath{|} \coqdocvar{mkMVec}   \coqdocvar{JUMP}   (\coqdocvar{INSTR} (\coqdocvar{Some} \coqdocvar{n}))  (\coqdocvar{INSTR} (\coqdocvar{Some} \coqdocvar{m}))  \coqdocvar{\_}\coqdoceol
\coqdocindent{1.00em}
\ensuremath{|} \coqdocvar{mkMVec}   \coqdocvar{JAL}    (\coqdocvar{INSTR} (\coqdocvar{Some} \coqdocvar{n}))  (\coqdocvar{INSTR} (\coqdocvar{Some} \coqdocvar{m}))  \coqdocvar{\_}  \ensuremath{\Rightarrow}\coqdoceol
\coqdocindent{2.00em}
\coqdockw{if} \coqdocvar{cfg} \coqdocvar{n} \coqdocvar{m} \coqdockw{then} \coqdocvar{Some} (\coqdocvar{mkRVec} (\coqdocvar{INSTR} (\coqdocvar{Some} \coqdocvar{m})) \coqdocvar{DATA})\coqdoceol
\coqdocindent{2.00em}
\coqdockw{else} \coqdocvar{None}\coqdoceol
\coqdocindent{1.00em}
\ensuremath{|} \coqdocvar{mkMVec}   \coqdocvar{JUMP}   \coqdocvar{DATA}  (\coqdocvar{INSTR} (\coqdocvar{Some} \coqdocvar{n}))  \coqdocvar{\_}\coqdoceol
\coqdocindent{1.00em}
\ensuremath{|} \coqdocvar{mkMVec}   \coqdocvar{JAL}    \coqdocvar{DATA}  (\coqdocvar{INSTR} (\coqdocvar{Some} \coqdocvar{n}))  \coqdocvar{\_}   \ensuremath{\Rightarrow} \coqdoceol
\coqdocindent{2.00em}
\coqdocvar{Some} (\coqdocvar{mkRVec} (\coqdocvar{INSTR} (\coqdocvar{Some} \coqdocvar{n})) \coqdocvar{DATA})\coqdoceol
\coqdocindent{1.00em}
\ensuremath{|} \coqdocvar{mkMVec}   \coqdocvar{JUMP}   \coqdocvar{DATA}  (\coqdocvar{INSTR} \coqdocvar{None})  \coqdocvar{\_}\coqdoceol
\coqdocindent{1.00em}
\ensuremath{|} \coqdocvar{mkMVec}   \coqdocvar{JAL}    \coqdocvar{DATA}  (\coqdocvar{INSTR} \coqdocvar{None})  \coqdocvar{\_}  \ensuremath{\Rightarrow}\coqdoceol
\coqdocindent{2.00em}
\coqdocvar{None}\coqdoceol
\coqdocindent{1.00em}
\ensuremath{|} \coqdocvar{mkMVec}   \coqdocvar{STORE}  (\coqdocvar{INSTR} (\coqdocvar{Some} \coqdocvar{n}))  (\coqdocvar{INSTR} (\coqdocvar{Some} \coqdocvar{m}))  [\coqdocvar{\_} ; \coqdocvar{\_} ; \coqdocvar{DATA}]  \ensuremath{\Rightarrow}\coqdoceol
\coqdocindent{2.00em}
\coqdockw{if} \coqdocvar{cfg} \coqdocvar{n} \coqdocvar{m} \coqdockw{then} \coqdocvar{Some} (\coqdocvar{mkRVec} \coqdocvar{DATA} \coqdocvar{DATA}) \coqdockw{else} \coqdocvar{None}\coqdoceol
\coqdocindent{1.00em}
\ensuremath{|} \coqdocvar{mkMVec}   \coqdocvar{STORE}  \coqdocvar{DATA}  (\coqdocvar{INSTR} \coqdocvar{\_})  [\coqdocvar{\_} ; \coqdocvar{\_} ; \coqdocvar{DATA}]  \ensuremath{\Rightarrow} \coqdoceol
\coqdocindent{2.00em}
\coqdocvar{Some} (\coqdocvar{mkRVec} \coqdocvar{DATA} \coqdocvar{DATA})\coqdoceol
\coqdocindent{1.00em}
\ensuremath{|} \coqdocvar{mkMVec}   \coqdocvar{STORE}  \coqdocvar{\_}  \coqdocvar{\_}  \coqdocvar{\_}  \ensuremath{\Rightarrow} \coqdocvar{None}\coqdoceol
\coqdocindent{1.00em}
\ensuremath{|} \coqdocvar{mkMVec}    \coqdocvar{\_}    (\coqdocvar{INSTR} (\coqdocvar{Some} \coqdocvar{n}))  (\coqdocvar{INSTR} (\coqdocvar{Some} \coqdocvar{m}))  \coqdocvar{\_}  \ensuremath{\Rightarrow} \coqdoceol
\coqdocindent{2.00em}
\coqdockw{if} \coqdocvar{cfg} \coqdocvar{n} \coqdocvar{m} \coqdockw{then} \coqdocvar{Some} (\coqdocvar{mkRVec} \coqdocvar{DATA} \coqdocvar{DATA}) \coqdockw{else} \coqdocvar{None}\coqdoceol
\coqdocindent{1.00em}
\ensuremath{|} \coqdocvar{mkMVec}    \coqdocvar{\_}    \coqdocvar{DATA}  (\coqdocvar{INSTR} \coqdocvar{\_})  \coqdocvar{\_}  \ensuremath{\Rightarrow} \coqdoceol
\coqdocindent{2.00em}
\coqdocvar{Some} (\coqdocvar{mkRVec} \coqdocvar{DATA} \coqdocvar{DATA})\coqdoceol
\coqdocindent{1.00em}
\ensuremath{|} \coqdocvar{mkMVec} \coqdocvar{\_} \coqdocvar{\_} \coqdocvar{\_} \coqdocvar{\_} \ensuremath{\Rightarrow} \coqdocvar{None}\coqdoceol
\coqdocindent{1.00em}
\coqdockw{end}.\coqdoceol
\coqdocemptyline
\coqdocnoindent
\end{coqdoccode}
\caption{Transfer function for symbolic machine in Gallina}
\label{fig:transfer_coq}
\end{figure}


Although, the rules in \ref{sec:cfi_fine} were fairly simply, expressing them
using Gallina's pattern matching increased their size. We also experimented,
with different ways of writing the transfer function but we decided to stick
with the definition above as it's the most straightforward. It's worth to note
that bugs in the above definition were easily made apparent when proving
theorems involving the transfer function. In fact, an ``interesting'' experiment
was to re-define the above function in a different way and prove the two 
equivalent. It took two iterations before getting both functions to agree
and although for small definitions like the one above, testing or manually
reviewing the code will reveal most if not all bugs, the importance of formal
verification in software engineering and critical software is made obvious
even for definitions that may seem trivial at first. The correctness of the
transfer function will come from simulation proofs between the abstract
and the symbolic machine.

\subsection{Attacker model}\label{sec:symbolic_attacker}

Similar to the abstract attacker, the symbolic attacker can change all words
tagged as \DATAname but not the ones tagged as \INSTRname. This is expressed by
the following relations:


\begin{figure}[ht]
\centering
\begin{minipage}[b]{0.25\linewidth}
\centering
\infrule[AttackData]{\atom{w_1}{\DATA}
  }{\atom{w_2}{\DATA}}
\label{fig:Attack_{data}}
\end{minipage}
\hspace{0.5cm}
\begin{minipage}[b]{0.15\linewidth}
\centering
\infrule[AttackInstr]{\atom{w_1}{\INSTR{id}}
  }{\atom{w_1}{\INSTR{id}}}
\label{fig:Attack_{instr}}
\end{minipage}
\caption{Attacker capabilities}
\end{figure}

These attacker relations on symbolic atoms are extended to a relation on the
memory and the registers.

\begin{figure}[!htpb]
\infrule{
  \ii{dom}~\dmem = \ii{dom}~\dmem' \andalso
  \ii{dom}~\reg = \ii{dom}~\reg'
}{
  \acfistat{\imem}{\dmem}{\reg}{\pc}{\ok} \to_a^A
  \acfistat{\imem}{\dmem'}{\reg'}{\pc}{\ok}
}
\caption{Attacker model for the abstract machine}
\end{figure}

\chapter{Conclusions}
\label{ch:conclusion}
\section{Future Work}

In this thesis we formalized and verified a dynamic monitor for \CFI,
by providing a specification for it and evaluating its effectiveness
against potential attackers. There are many directions still left to
explore before we can consider our work done. Some of them include
writing the \CFI monitor code and verifying it, increasing precision
by enforcing call-stack protection, scaling to more complex
architectures (\EG ARM) and looking for ways to enforce \CFI-like policies on
self modifying programs.

\subsection{Writing and Verifying Monitor Code}

In this thesis, we described the \CFI micro-policy and reasoned about
its security properties by using a high-level specification of the
policy monitor, expressed in terms of a \TRANSFER function written in
Coq. In reality, when we leveraged the micro-policies framework we
\emph{assumed} the existence of machine code implementing the \CFI
policy monitor and its correctness as specified by the high-level
\TRANSFER function.

Although we have not written the machine code for the policy monitor -
and consequently not verified it - we consider the existence of
correct code implementing the policy monitor as a realistic
assumption. Azevedo \ETAL provided code for a dynamic sealing
micro-policy in \cite{popl2015}, although they did not verify it.
Furthermore in \cite{PicoCoq2013}, that can be considered as a predecessor
to the micro-policies project, machine code for an IFC
monitor was obtained using structured code generators and a verified
DSL compiler.

Arguably the code for a dynamic sealing monitor is simpler than the
code for a \CFI monitor, but even an efficient implementation of a
\CFI monitor would probably resemble a compiled switch statement/match
expression, for which there are plenty of resources on efficient
compilation strategies. One could even write the \CFI policy monitor
by hand, however we decided not to attempt this, as it seemed that
without verifying it, there was little added value considering the
amount of effort required. Furthermore, in order to be able to at
least test the correctness of the implementation, we would be required
to provide machine code for programs and to also compute their control-flow
graph, which would be tedious and time consuming without the appropriate tools.

As noted in \cite{popl2015} it would make more sense to go
through the effort of writing and verifying machine code for a more
realistic architecture.  In a standard RISC architecture setting
(\EG ARM) we could write the policy monitor in a higher-level
language (even C) and use a (verified) compiler (\EG CompCert
\cite{leroy09:compcert}) to obtain the machine code. Furthermore, we
could leverage existing verification frameworks, either for low-level
code \cite{Chlipala2013,JensenBK13} or for the high-level language we
used to code the policy monitor (\EG
\cite{Appel:2011:VST:1987211.1987212} in the case of C code), in order
to verify the correctness of our implementation.


% due to the
% general scope of the micro-policies framework it makes more sense to
% devise an expressive DSL for micro-policies and build a compiler that
% would automatically emit efficient machine code. This approach was
% taken in \cite{PicoCoq2013}, that can be considered as a predecessor
% to the micro-policies project, where the machine code for an IFC
% monitor was obtained using structured code generators and a verified
% DSL compiler. Although their approach lacked generality and
% expressiveness to encode a wide-range of micro-policies it strengthens
% our intuition about the feasibility of such a project. Furthermore
% as the authors of \cite{pump_popl2015} mention it would make
% more sense to write the machine code for the policy monitors for
% a more realistic architecture. This would even allow us
% to use a high-level language to code the policy monitor and perhaps
% enable the re-use of existing verification frameworks in order
% to verify its correctness.

\subsection{Call-Stack Protection}

\CFI enforces that the execution path of a program follows a
pre-computed, \textit{static} control-flow graph. Thus it cannot
enforce that a function returns to the original callsite it was called
from. We can increase the precision of \CFI on returns, by using a
protected call-stack. This is the approach taken in \cite{AbadiBEL09} in
order to increase precision on returns.

We believe that we can use the micro-policies mechanism to enforce a
calling convention and increase the precision of the \CFI
micro-policy. This would certainly include reserving a part of the
memory as a call-stack and protecting it in a fashion similar (but
stronger) to the \NWC micro-policy. We then have to populate this
protected call-stack in a meaningful way. We have not yet concluded on
an efficient and effective way to do this although we have studied a
few options. One rather crude approach to this would be to use tags
and rules to enforce that suitable book-keeping instructions,
manipulating the call-stack, are executed before and after each
call. This would most probably have the desired effectiveness, however
it may be too restrictive in some contexts. A more elegant solution
would be to use the tag on the \pc, the tag on the ra register and the
tags on the protected call-stack part of the memory, to store suitable
meta-data (\EG call depth) in order to determine whether a return
should be allowed or not.

Concerning the formal verification of such a micro-policy, an
ambitious goal would be to prove refinement between the concrete
machine running a dynamic monitor for call-stack protection and an
abstract machine with a separate protected-call stack. While this
abstract machine provides an intuitive specification for call-stack
protection, it would result in a complex refinement relation due to
the fact that the concrete machine would have to execute some
book-keeping instructions which the abstract machine would not.






% \backmatter
\cleardoublepage % start at the next odd page
\phantomsection % correct hyperlinking
\addcontentsline{toc}{chapter}{\bibname} % add bibliography section to toc
\bibliography{bibliography,safe,local}
\bibliographystyle{abbrv} % plain/abbrv/alpha/abstract/apalike/...
% \include{glossary}

%\appendix
%\chapter{Stuff}
%
\section{Control-Flow Integrity Micro-Policy}

We begin with a micro-policy targeting control-flow hijacking attacks,
in which an attacker exploits a low-level vulnerability (e.g. a buffer
or integer overflow) to gain full control of a target program~\cite{
  ShellcoderHandbook, Szekeres2013, Smashing1996, SkyLined, PincusB04,
  Sotirov07, DanielHM08, AfekS07, Dobrovitski03}.
%
As a first line of defense, we can use tags to make code non-writable
(NWC) and data non-executable (NXD), preventing the injection and
execution of an attacker payload.
%
This useful defense appears in various forms in existing systems.
However, it does not prevent code-reuse attacks~\cite{Newsham1997,
  SolarDesigner1997, McDonald1999, Shacham07, Checkoway2010,
  Buchanan2008, SnowMDDLS13, outofcontrol_ieeesp2014} such as return- or
jump-oriented programming~\cite{Shacham07, Checkoway2010}, where the
attacker chains together existing code snippets (``gadgets'') to induce
arbitrary malicious behavior.
%
We therefore use tags to enforce fine-grained {\em control-flow integrity
  (CFI)}~\cite{AbadiBEL09, ZhaoLSR11, Zhang2013, CriswellDA14, NiuT14,
  ZhaoLSR11, CriswellDA14} on top of basic NWC and NXD protection.
%
This ensures that all indirect control flows (computed jumps) adhere
to a fixed control flow graph (CFG).

\newcommand{\CODEname}[0]{\ii{Code}}
\newcommand{\CODE}[1]{\CODEname~#1}

\makeatletter
\newdimen\OPCODEwidth
\OPCODEwidth .5in
\newdimen\RULEwidth
\newcommand{\TRUE}{\text{\tt true}}
\newcommand{\RULE}[9]{
\gdef\RULEARROW{\ifthenelse{\equal{#7}{}}{\Rightarrow}{\rightarrow}}  % HACK!
\gdef\RULEINPUT{(#2,#3,#4,#5,#6)}
\gdef\RULEOUTPUT{(#8,#9)}
\gdef\RULECOND{%
  \ifthenelse{\equal{#7}{}}{}%
             {\ifthenelse{\equal{#7}{\TRUE}}{}%
                         {\mathrm{\;if\;}#7}}%
}
&& \hspace*{-18.5em}
  \hbox to 1in {
      \hbox to \OPCODEwidth {\ifx&#1&\else$#1$\ : \fi}
      % See how big it is on one line
      \setbox \@tempboxa \hbox{$\RULEINPUT \RULEARROW \RULEOUTPUT \RULECOND$}
      \RULEwidth \wd\@tempboxa
      % Does it fit?
      \ifdim \RULEwidth < 2.4in
        % Use it
        \box\@tempboxa
      \else\ifdim \RULEwidth < 4.8in
        % Put it on two lines
        $ \hspace*{-1.3em}
        \begin{array}[t]{@{}l@{\ }l}
            & \RULEINPUT \\
            \RULEARROW
            & \RULEOUTPUT\RULECOND
        \end{array}
       $
      \else
        % Put it on three lines
        $ \hspace*{-1.3em}
        \begin{array}[t]{@{}l@{}l}
            & \ \RULEINPUT \\
            \RULEARROW
            & \ \RULEOUTPUT \\
            & \RULECOND
        \end{array}
       $
     \fi\fi
  }
}

\newcommand{\RULEWITHPREMISE}[9]{
\typicallabel{}
  \infrule{#7}{\ii{#1} : (#2, #3, #4, #5, #6) \to (#8,#9)}
\typicallabel{MkKey}
}

We use tags to distinguish between code and data.
%
Tags on memory and the PC are drawn from the set
%
$
\DATA \;|\; \CODE{\ii{addr}} \;|\; \CODE{\bot}
$
(registers are always tagged $\DATA$).
%
To simplify the CFG conformance checks, instructions that are the
source or target of indirect control flows are tagged with
$\CODE{\ii{addr}}$, where $\ii{addr}$ is the address of that
instruction in memory.
%
For example, a $\ii{Jump}$ instruction stored at address $500$ is
tagged $\CODE{500}$.
%
All other instructions are tagged $\CODE{\bot}$.
%
\aaa{Actually, we can't use the instruction's address on the tag if we
  are to have the same number of bits on words and tags. Maybe change
  to ``id''?}

We write transfer functions as a collection of {\em
  symbolic rules}~\cite{popl2015, pump_hasp2014}.
% \ch{Should we remove all references to \cite{pump_ccs2014}?}
% BCP: yes
%
(The PUMP hardware uses a lower-level {\em concrete rule} format, 
described in \autoref{implementation}.)
%
Each symbolic rule has the form
%
% \begin{eqnarray*}
% \RULE
%   {\mathit{opcode}}
%   {\mathit{PC}}{\mathit{CI}}{\mathit{OP_1}}{\mathit{OP_2}}{\mathit{OP_3}}
%   {\TRUE}
%   {\mathit{PC'}}{\mathit{R'}}
% \end{eqnarray*}
%
\newcommand{\INLINERULE}[8]{\mathit{#1}
\mathrel{:}\allowbreak
 ({\mathit{#2}},\allowbreak{\mathit{#3}},\allowbreak{\mathit{#4}},\allowbreak{\mathit{#5}},\allowbreak{\mathit{#6}})
 \rightarrow\allowbreak ({\mathit{#7}},\allowbreak{\mathit{#8}})
}%
``$\INLINERULE{opcode}{PC}{CI}{OP_1}{OP_2}{OP_3}{PC'}{R'}$,''
%
which says that the rule matches on the given {\it opcode} together
with the metadata tags on the program counter ($\mathit{PC}$), the
current instruction ($\mathit{CI}$), and on up to three operands
($\mathit{OP_1}$ to $\mathit{OP_3}$).
%
If the rule applies, the right-hand side determines how to update the
tags on the PC ($\mathit{PC'}$) and on the result of the operation
($\mathit{R'}$).
%
We write ``$-$'' to indicate input or output fields that are ignored
(``wildcard'').
%
\chrev{All instructions that are not explicitly allowed by the
  symbolic rules are disallowed.}%
\aaa{We should choose only one of $-$ or $\_$ for our wildcard and use
  it consistently (cf. the ``Store'' rule below)}%

The CFI transfer function enforces that only memory locations tagged
$\DATA$ can be modified (NWC) and only instructions fetched from
locations tagged $\CODEname$ can be executed (NXD).
%
The symbolic rule for the $\ii{Store}$ instruction illustrates both
these points:
%
\begin{eqnarray*}
\INLINERULE
  {\ii{Store}}
  {\DATA}{\CODE{\text{\textunderscore}\,}}{-}{-}{\DATA}
  {\DATA}{-}
\end{eqnarray*}
%
It requires the fetched $\ii{Store}$ instruction to be tagged
$\CODEname$ and the written location to be tagged $\DATA$.
%
This rule only applies when the PC is also tagged $\DATA$, which is
the case when the $\ii{Store}$ instruction was reached by direct
control flow (not a computed jump).
%
The rule preserves the $\DATA$ tag on the PC, since $\ii{Store}$ is
not a computed jump.
%
Performing a computed jump (\EG using \ii{Jal}, a
jump-and-link instruction) requires that the current instruction be
tagged $\CODE{\ii{src}}$ for some address $\ii{src}$.
%
\begin{eqnarray*}
\INLINERULE
  {\ii{Jal}}
  {\DATA}{\CODE{\ii{src}}}{-}{-}{-}
  {\CODE{\ii{src}}}{-}
\end{eqnarray*}
%
This rule copies $\CODE{\ii{src}}$ to the PC tag to indicate
that a jump from \ii{src} has just occurred.
%
Only on the next instruction do we have enough information about the
destination in the tags to check that the jump is indeed allowed by
the CFG.
%
For this we add a second rule for \ii{Store}, dealing with the case
where it is the target of a jump and thus the PC is tagged
$\CODE{\ii{src}}$.
%
\RULEWITHPREMISE
  {\ii{Store}}
  {\CODE{\ii{src}}}{\CODE{\ii{tgt}}}{-}{-}{\DATA}
  {(\ii{src},\ii{tgt}) \in \ii{CFG}}
  {\DATA}{-}
%
\aaa{Maybe we could discuss here a little bit why we verify the jump
  on the next instruction, as opposed to when the jump is
  performed. This might get some people confused, since this is not
  very natural and fundamentally driven by our current design of the
  PUMP. Even Nick wanted to know if we couldn't do it differently.}%
  The premise of this rule ensures that the source and target of the
  just-performed jump are allowed by the CFG.
%
  We add a similar rule for each instruction, including jumps (since
  the target of a computed jump can itself be another computed
  jump):
%
\RULEWITHPREMISE
{\ii{Jal}}
{\CODE{\ii{src}}}{\CODE{\ii{tgt-src}}}{-}{-}{-}
{(\ii{src},\ii{tgt-src}) \in \ii{CFG}}
{\CODE{\ii{tgt-src}}}{-}

This micro-policy enforces fine-grained CFI~\cite{NiuT14,
  outofcontrol_ieeesp2014, CriswellDA14}, not coarse-grained
approximations~\cite{AbadiBEL09, Zhang2013} that are potentially vulnerable
to attack~\cite{outofcontrol_ieeesp2014}.  Indeed, we recently
proved~\cite{popl2015} that this micro-policy enforces a variant of the CFI
property introduced by Abadi~\ETAL\cite{AbadiBEL09}, ensuring that all
indirect control flows adhere to a fixed CFG.
%
Recent simulations of an optimized PUMP architecture~\cite{pump_asplos2015}
show that the CFI policy can be enforced with around 3\% average runtime
overhead.


% \printindex

\end{document}
