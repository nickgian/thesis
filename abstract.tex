\chapter*{Περίληψη}
\label{absgr}
\addcontentsline{toc}{chapter}{Περίληψη}
  Μια ευρεία γκάμα επιθέσεων λογισμικού προσπαθούν να ανακτήσουν τον
  έλεγχο ροής του προγραμμάτος με σκοπό να τροποποιήσουν τη
  συμπεριφορά του. Η Ακεραιότητα Ελέγχου-Ροής είναι μία αποτελεσματική
  πολιτική ασφαλείας, που μπορεί να αποτρέψει όλες τις επιθέσεις που
  επιχειρούν να παρακάμψουν την αρχική ροή ελέγχου του προγράμματος. 

  Σε αυτή τη διπλωματική εργασία, χρησιμοποιούμε το εργαλείο
  διαδραστικών αποδείξεων Coq για να αιτιολογήσουμε τυπικά την
  ορθότητα και την αποτελεσματικότητα ενός δυναμικού ελεγκτή που
  επιβάλλει Ακεραιότητα Ελέγχου-Ροής, βασιζόμενος σε ένα καινοτόμο
  μηχανισμό ασφαλείας που χρησιμοποιεί λογισμικί και υλικό.
  Συγκεκριμένα, αποδεικνύομε οτι ο μηχανισμός επιβάλλει Ακεραιότητα
  Ελέγχου-Ροής ακόμα και υπό την παρουσία ενός ισχυρού κακόβουλου
  χρήστη. Επιπλέον αποδεικνύουμε μέσω εκκαθάρισης ότι ένα μηχάνημα στο
  οποίο τρέχει ο δυναμικός ελεγκτής για την Ακεραιότητα Ελέγχου-Ροής,
  επακριβώς εξομοιώνει όλες τις συμπεριφορές ενός αφηρημένου
  μηχανήματος που έχει Ακεραιότητα Ελέγχου-Ροής εκ κατασκευής.

\begin{keywords}
ροή-ελέγχου, ασφάλεια, επαλήθευση, αρχιτεκτονικές με ετικέτες
\end{keywords}


\chapter*{Abstract}
%% English Abstract
\label{Abstract}
\addcontentsline{toc}{chapter}{Abstract}

  A wide-range of software attacks attempt to hijack the control-flow
  of the program in order to alter its behavior. Control-Flow
  Integrity is an effective security policy, able to thwart all
  attacks that attempt to circumvent the original control-flow of a
  program.

  In this thesis, we use the Coq proof assistant to formally reason
  about the correctness and the effectiveness of a dynamic monitor
  enforcing \CFI, based on a novel software-hardware security
  mechanism. In particular, we prove that the mechanism enforces \CFI
  even in the presence of a powerful attacker. Furthermore, we prove
  by refinement that a machine running the dynamic monitor for \CFI,
  precisely emulates all behaviors of an abstract machine that has
  \CFI by construction.

\begin{keywordseng}
control-flow, security, verification, tagged architectures
\end{keywordseng}




%%Greek Acknowledgements
\begin{acknowledgements}

Θα ήθελα να ευχαριστήσω τον C\u{a}t\u{a}lin Hri\c{t}cu για την εμπιστοσύνη που μου έδειξε, την ευκαιρία να εργαστώ σε
ένα κορυφαίο ερευνητικό κέντρο και την καθοδήγηση του κατα την εκπόνηση αυτής της διπλωματικής εργασίας. 

Θα ήθελα επίσης να ευχαριστήσω τους καθηγητές μου κ.Νίκο Παπασπύρου και κ.Κωστή Σαγώνα για τη διδασκαλία τους μέσω
της οποίας μου μετέφεραν το ενδιαφέρον τους για τις γλώσσες προγραμματισμού αλλά και τη βοήθεια που μου
προσέφεραν όποτε τη χρειάστηκα στη μέχρι τώρα ακαδημαϊκή μου πορεία.

Τέλος, θα ήθελα να ευχαριστήσω την οικογένεια μου και τη σύντροφο μου
Ζωή Παρασκευοπούλου για την αστείρευτη τους στήριξη και αγάπη.
\end{acknowledgements}
