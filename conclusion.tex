\chapter{Conclusion}

\section{Future Work}

There are many directions still left to explore before we can consider
our work done. Some of them include writing the \CFI monitor code and
verifying it, increasing precision by enforcing call-stack protection,
scaling to more complex architectures and looking for ways to enforce
\CFI-like policies on self modifying programs.

\subsection{Writing and Verifying Monitor Code}

In this thesis, we described the \CFI micro-policy and reasoned about
its security properties by using a high-level specification of the
policy monitor, expressed in terms of a \TRANSFER function written in
Coq. In reality, when we leveraged the micro-policies framework we
\emph{assumed} the existence of machine code implementing the \CFI
policy monitor and its correctness as specified by the high-level
\TRANSFER function.

Although we have not written the machine code for the policy monitor -
and consequently not verified it - we consider the existence of
correct code implementing the policy monitor as a realistic
assumption. Azevedo \ETAL provided code for a dynamic sealing
micro-policy in \cite{pump_popl2015}, although they did not verify it.
Furthermore in \cite{PicoCoq2013}, that can be considered as a predecessor
to the micro-policies project, machine code for an IFC
monitor was obtained using structured code generators and a verified
DSL compiler.
\nick{shrink references to IFC and sealing? I want them as a
witness to my claims about the possibility of writing/verifying the cfi
monitor code}

Arguably the code for a dynamic sealing monitor is simpler than the
code for a \CFI monitor, but even an efficient implementation of a
\CFI monitor would probably resemble a compiled switch statement/match
expression, for which there are plenty of resources on efficient
compilation strategies. One could even write the \CFI policy monitor
by hand, however we decided not to attempt this, as it seemed that
without verifying it, there was little added value considering the
amount of effort required. Furthermore, in order to be able to at
least test the correctness of the implementation, we would be required
to provide machine code for programs and to also compute their control-flow
graph, which would be tedious and time consuming without the appropriate tools.

As noted in \cite{pump_popl2015} it would make more sense to go
through the effort of writing and verifying machine code for a more
realistic architecture.  In a standard RISC architecture setting
(\EG ARM) we could write the policy monitor in a higher-level
language (even C) and use a (verified) compiler (\EG CompCert
\cite{leroy09:compcert}) to obtain the machine code. Furthermore, we
could leverage existing verification frameworks, either for low-level
code \cite{Chlipala2013,JensenBK13} or for the high-level language we
used to code the policy monitor (\EG
\cite{Appel:2011:VST:1987211.1987212} in the case of C code), in order
to verify the correctness of our implementation.

\nick{probably rephrase that and somehow unify the citation to
VST and compcert?}

% due to the
% general scope of the micro-policies framework it makes more sense to
% devise an expressive DSL for micro-policies and build a compiler that
% would automatically emit efficient machine code. This approach was
% taken in \cite{PicoCoq2013}, that can be considered as a predecessor
% to the micro-policies project, where the machine code for an IFC
% monitor was obtained using structured code generators and a verified
% DSL compiler. Although their approach lacked generality and
% expressiveness to encode a wide-range of micro-policies it strengthens
% our intuition about the feasibility of such a project. Furthermore
% as the authors of \cite{pump_popl2015} mention it would make
% more sense to write the machine code for the policy monitors for
% a more realistic architecture. This would even allow us
% to use a high-level language to code the policy monitor and perhaps
% enable the re-use of existing verification frameworks in order
% to verify its correctness.

\subsection{Call-Stack Protection/ XFI}
