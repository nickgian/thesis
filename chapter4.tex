\chapter{Formally Verified Control-Flow Integrity Micro-Policy}
\label{ch:verified_cfi}

Using the micro-policies framework described in \ref{sec:framework} we
proved that the concrete machine instantiated with a \CFI micro-policy
like the one described in \ref{sec:cfi_fine} \ii{simulates} an
abstract machine that has \CFI by construction. We do this proof by
using the symbolic machine as an intermediate step, to prove backwards
simulation between the symbolic and the abstract machine and
afterwards by leveraging the framework of section \ref{sec:framework}
we obtain a backwards refinement between the concrete and the abstract
machine.

In addition, we provide an attacker model for all the machines used
and we prove that a property capturing the notion of \CFI holds even
when the attacker tampers with the machine, similarly to what is
proposed in \cite{AbadiBEL09}, but adapted to the setting of our
machines. We do this by first proving this property for the
abstract machine and then by using a generic preservation theorem
we developed we prove that this property is \emph{preserved} by
backwards refinement and thus transferring the property to the
symbolic and consequently to the concrete machine.

\section{Representing control-flow graphs}\label{sec:cfi_tags}

Our approach for enforcing \CFI, as explained in \ref{sec:cfi_fine},
requires that we encode the nodes in the control-flow graph in terms
of identifiers, which in turn are used to tag all sources and targets
of indirect control-flows.

At this point we take a detour, to point out an important design point
of the micro-policies framework and our \CFI micro-policy.  Throughout
both developments, a heavily parametric and modular approach was
taken. This parametric design is enabled by the use of the
\emph{Section} and \emph{Type Classes} mechanisms of Coq. As an
example, the node identifiers, along with a number of properties we
require of them are expressed by the following interface (defined in
terms of a type class):

\begin{figure}[!htpb] 
\begin{coqdoccode}
  \coqdocnoindent \coqdockw{Context} \{\coqdocvar{t} :
  \coqdocvar{machine\_types}\}.\coqdoceol \coqdocemptyline
  \coqdocnoindent \coqdockw{Class} \coqdocvar{cfi\_id} := \{\coqdoceol
  \coqdocindent{1.00em} \coqdocvar{id} : \coqdocvar{eqType};\coqdoceol
  \coqdocnoindent \coqdoceol \coqdocindent{1.00em}
  \coqdocvar{word\_to\_id} : \coqdocvar{word} \coqdocvar{t}
  \ensuremath{\rightarrow} \coqdocvar{option}
  \coqdocvar{id};\coqdoceol \coqdocindent{1.00em}
  \coqdocvar{id\_to\_word} : \coqdocvar{id} \ensuremath{\rightarrow}
  \coqdocvar{word} \coqdocvar{t};\coqdoceol \coqdocindent{1.00em}
  \coqdoceol \coqdocindent{1.00em} \coqdocvar{id\_to\_wordK} :
  \coqdockw{\ensuremath{\forall}} \coqdocvar{x},
  \coqdocvar{word\_to\_id} (\coqdocvar{id\_to\_word} \coqdocvar{x}) =
  \coqdocvar{Some} \coqdocvar{x};\coqdoceol \coqdocindent{1.00em}
  \coqdocvar{word\_to\_idK} : \coqdockw{\ensuremath{\forall}}
  \coqdocvar{w} \coqdocvar{x}, \coqdocvar{word\_to\_id} \coqdocvar{w}
  = \coqdocvar{Some} \coqdocvar{x} \ensuremath{\rightarrow}
  \coqdocvar{id\_to\_word} \coqdocvar{x} = \coqdocvar{w}\coqdoceol
  \coqdocindent{14.50em} \coqdoceol \coqdocnoindent \}.\coqdoceol
\end{coqdoccode}
\caption{Interface of node identifiers}
\label{fig:cfi_id}
\end{figure}

The \emph{Context} command on the top of the code above, allows us to
\textit{assume} that there exists an instance of this interface.
In fact, the \emph{machine\_types} argument is just another type class,
serving as a specification of the various types of the machine (\EG the word 
size). This approach allowed us to abstract away from various details and
structure our proofs in a clean way.
In addition, we can easily instantiate a different machine with minimal changes
in our proofs and definitions (\EG instantiate the machine with a different word
size).

However, one drawback is that one wrong specification in a type class would
disallow us to instantiate it and would require that we go back and change
all parts that used this wrong specification (\EG in our case, the 
\emph{machine\_types} class was widely used). Therefore one should be careful
when doing heavy use of such mechanisms.

Returning to the identifiers, looking at the definition in \ref{fig:id_class},
we require that the type of the identifiers \id is an Eqtype (has decidable
boolean equality) and that there exists conversion functions between elements
of type \word and \id, satisfying some constraints. 

\QUESTION{Should I give some intuition, as to why word\_to\_id is partial? Is it
obvious?}

As mentioned in \ref{sec:cfi_fine}, we check for violations of the
control-flow with respect to a binary relation (on the identifiers)
\CFG which represents the set of allowed (indirect) jumps. We can
extend this relation to precisely describe the control-flow of a
program, by lifting \CFG to a relation \SUCC{} on machine states, that
includes the set of allowed targets for the rest of the
instructions. Throughout this thesis we use \CFG to refer to the set
of allowed jumps. In our Coq development we assumed a translation of
the allowed jumps in form of a function on two identifiers.

\begin{figure}[H]
  \coqdockw{Variable} \coqdocvar{cfg} : \coqdocvar{id}
  \ensuremath{\rightarrow} \coqdocvar{id} \ensuremath{\rightarrow}
  \coqdocvar{bool}.\coqdoceol
\caption{Function on ids representing the set of allowed jumps}
\label{fig:cfg}
\end{figure}

In addition we defined a function valid\_jmp (referred to with the
notation \J) that expresses the set of allowed jumps between words, by
using the \emph{word\_to\_id} function.

\begin{figure}[H]
  \coqdockw{Definition} \coqdocvar{valid\_jmp} \coqdocvar{w1}
  \coqdocvar{w2} :=\coqdoceol \coqdocindent{1.00em} \coqdockw{match}
  \coqdocvar{word\_to\_id} \coqdocvar{w1}, \coqdocvar{word\_to\_id}
  \coqdocvar{w2} \coqdockw{with}\coqdoceol \coqdocindent{2.00em}
  \ensuremath{|} \coqdocvar{Some} \coqdocvar{id1}, \coqdocvar{Some}
  \coqdocvar{id2} \ensuremath{\Rightarrow} \coqdocvar{cfg}
  \coqdocvar{id1} \coqdocvar{id2}\coqdoceol \coqdocindent{2.00em}
  \ensuremath{|} \coqdocvar{\_}, \coqdocvar{\_}
  \ensuremath{\Rightarrow} \coqdocvar{false}\coqdoceol
  \coqdocindent{1.00em} \coqdockw{end}.\coqdoceol
\caption{Function on words representing the set of allowed jumps}
\label{fig:valid_jmp}
\end{figure}


\section{Control-Flow Integrity Property}\label{sec:cfi_property}

Our formalization includes a definition of \CFI, similar to the one found in
\cite{AbadiBEL09}, which we prove to be true of all our machines. The need for a
new definition arises from fundamental differences between our enforcement
mechanism on the concrete mechanism and the one used by Abadi \ETAL. In
particular, our enforcement-mechanism does not prevent a violation, instead
it can detect it after it has occurred by taking an arbitrary number of
``protected'' (monitor mode) steps before eventually bringing the machine to a
halt. This does not have any impact on the security effectiveness of our
mechanism, it does however lead to a more complex definition and therefore
more complex proofs.

The definition of \CFI is further parameterized by an attacker model. We
model the attacker as a step relation (\stepa{}{}). Intuitively the attacker is
allowed to change any \emph{user-level} data but not the code of the program and
the \pc, as well as the tags in the case of a tagged machine. 
This limitations ensures that an attacker cannot directly circumvent the monitor
protection mechanism and our user-level policies (\NWC , \NXD and \CFI). To 
account for attacker steps, the stepping relation is extended as the union of 
the normal step relation (\stepn{}{}), as defined by the machine semantics, and
the attacker step relation (\stepa{}{}), as defined by the attacker model.

\begin{figure}[ht]
\centering
\begin{minipage}[b]{0.25\linewidth}
\centering
\infrule[]{\stepn{s}{s'}
  }{\step{s}{s'}}
\label{fig:step_stepn}
\end{minipage}
\hspace{0.5cm}
\begin{minipage}[b]{0.15\linewidth}
\centering
\infrule[]{\stepa{s}{s'}{}
  }{\step{s}{s'}}
\label{fig:step_stepa}
\end{minipage}
\caption{Step relation definition}
\end{figure}

We define a predicate \INITIAL{s}, where s is a machine state, that
states that s is an initial state. We use this predicate to express some
invariants that are preserved through execution (\EG the initial tagging scheme
for the memory). Finally we define a stopping predicate on an execution trace
that states that the machine is coming to a halt with respect to normal steps.

Since we want to instantiate the above parameters in a different way for
each of our machines, it makes sense to wrap them in a type class which
we will instantiate for each machine to get the corresponding definition
of \CFI.

\begin{figure}[ht]
\begin{coqdoccode}
  \coqdocnoindent \coqdockw{Class} \coqdocvar{cfi\_machine} :=
  \{\coqdoceol \coqdocindent{1.00em} \coqdocvar{state} :
  \coqdockw{Type};\coqdoceol \coqdocindent{1.00em} \coqdocvar{initial}
  : \coqdocvar{state} \ensuremath{\rightarrow}
  \coqdockw{Prop};\coqdoceol \coqdocindent{1.00em} \coqdoceol
  \coqdocindent{1.00em} \coqdocvar{step\_n} : \coqdocvar{state}
  \ensuremath{\rightarrow} \coqdocvar{state} \ensuremath{\rightarrow}
  \coqdockw{Prop};\coqdoceol \coqdocindent{1.00em} \coqdocvar{step\_a}
  : \coqdocvar{state} \ensuremath{\rightarrow} \coqdocvar{state}
  \ensuremath{\rightarrow} \coqdockw{Prop};\coqdoceol \coqdocnoindent
  \coqdoceol \coqdocindent{1.00em} \coqdocvar{succ} :
  \coqdocvar{state} \ensuremath{\rightarrow} \coqdocvar{state}
  \ensuremath{\rightarrow} \coqdocvar{bool};\coqdoceol
  \coqdocindent{1.00em} \coqdocvar{stopping} : \coqdocvar{list}
  \coqdocvar{state} \ensuremath{\rightarrow} \coqdockw{Prop}\coqdoceol
  \coqdocnoindent \}.\coqdoceol
\end{coqdoccode}
\caption{Interface of a cfi\_machine}
\label{fig:cfi_machine}
\end{figure}

For a machine of type cfi\_machine we give the following definitions:

\begin{definition}\label{definition:traceHasCfi}
  We say that an execution trace $s_0 \to s_1 \to \ldots \to s_n$ {\em has CFI}
  if for all $ i \in [0,\ldots,n)$ if \stepn{s_i}{s_{i+1}} then
  $(s_i,s_{i+1}) \in$ \SUCC .
\end{definition}

\QUESTION{The word relation for succ and cfg is strange since they are
booleans, is it ok, or does it confuse you, making you believe they are Props?}

The above definition corresponds to the one found in \cite{abadi2005}, however
it is stronger in the sense that it requires that steps that are in the
intersection of normal and attacker steps respect the control-flow. If we did
not allow for any violations then the above definition would be enough, but
since our enforcement mechanism allows for one violation we have to resort to a
weaker definition.

\begin{definition}[CFI]\label{definition:CFI}
  We say that the machine
  $(\ii{State}, \ii{initial}, \to_n,\allowbreak \to_a, \SUCCm{}, \ii{stopping})$
  has \CFI with respect to the set of allowed indirect jumps \CFG
  if, for any execution starting from initial state $s_0$
  and producing a trace $s_0 \to \ldots \to s_n$, either
  \begin{enumerate}
  \item The whole trace has \CFI according to
    \ref{definition:traceHasCfi}, or else
  \item There is some $i$ such that $s_i \to_n s_{i+1}$,
  and $(s_i, s_{i+1}) \not \in$ \SUCC{}, where
  the sub-traces $s_0 \to \ldots \to s_i$ and
  $s_{i+1} \to \ldots \to s_n$ both have CFI
  and the sub-trace $s_{i+1} \to \ldots \to s_n$ is stopping.
  \end{enumerate}
\end{definition}

\section{The Abstract Machine}\label{sec:abstract_cfi}

The abstract machine
has \CFI, \NXD and \NWC by construction and will serve as a 
specification for the symbolic and eventually the concrete machine that
implement \CFI through the tag-based system explained in the previous chapter.

Unlike the symbolic and the concrete machine, this abstract machine splits the 
memory into two disjoint memories, an instruction memory and a data memory. The
instruction memory is fixed (non-writable) and the machine uses this memory to
fetch instructions to execute, so \NWC and \NXD are enforced by construction.

In addition the state of the machine includes an \ok bit, indicating 
whether a control-flow violation has occurred or not. The rest of the machine
state is completed by a set of registers and a \pc register. We use a 5-tuple
notation for the state \acfistat{\imem}{\dmem}{\reg}{\pc}{\ok}, where the first
field is the instruction memory, the second the data memory, the third the
registers, the fourth is the pc register and the fifth is the \ok bit.

\subsection{Operational semantics}\label{abstract_semantics}
Below is the step rule for the Store instruction, illustrating both \NWC and 
\NXD. Notice that the instruction is fetched by the instruction memory and
the store is done on the data memory.

\begin{figure}[!htpb]
\infrule[Store]{
  \imem[\pc] = i \andalso \ii{decode}~i = \ii{Store}~r_p~r_s \andalso
  \rd{\reg}{r_p} = p \\
  \rd{\reg}{r_s} = w \andalso
  \dmem' = \upd{\dmem}{p}{w}
  }{\step{\acfistat{\imem}{\dmem}{\reg}{\pc}{\ii{true}}}{
    \acfistat{\imem}{\dmem'}{\reg'}{\pc+1}{\ii{true}}}}
\caption{Step rule for Store instruction of abstract machine}
\end{figure}

In the above rule, the \ok bit is true for both the starting and the
resulting state. In fact, the machine can take a step only when the
\ok bit is set to true. In the above rule, the \ok bit is set to true
in the resulting state, indicating that no control-flow violation has
happened, as expected by the execution of a Store
instruction. Control-flow violations in the \NWC setting our machine
is executing, can only occur from \emph{indirect} jump instructions,
in our case the Jump and Jal instructions. Upon execution of a Jump or
Jal instruction, we consult \J (see \ref{fig:valid_jmp}) to check
whether the change of control-flow is legal. If the jump is not
allowed according to \J then the jump is taken but the \ok bit is set
to false, which will halt the machine in the next step, as it is only
allowed to step when the \ok bit is set to true. Otherwise the \ok bit
will remain true.

\begin{figure}[!htpb]
\infrule[Jal]{
  \imem[\pc] = i \andalso \ii{decode}~i = \ii{Jal}~r \andalso
  \rd{\reg}{r} = \pc' \\
  \reg' = \upd{\reg}{\ra}{pc+1} \andalso
  \ii{ok} = (\pc,\pc') \in \Jm
  }{\step{\acfistat{\imem}{\dmem}{\reg}{\pc}{\ii{true}}}{
    \acfistat{\imem}{\dmem}{\reg'}{\pc'}{\ii{ok}}}}
\bigskip

\infrule[Jump]{
  \imem[\pc] = i \andalso \ii{decode}~i = \ii{Jump}~r \andalso
  \rd{\reg}{r} = \pc' \andalso
  \ii{ok} = (\pc,\pc') \in \Jm
  }{\step{\acfistat{\imem}{\dmem}{\reg}{\pc}{\ii{true}}}{
    \acfistat{\imem}{\dmem}{\reg'}{\pc'}{\ii{ok}}}}
\caption{Step rule for Jump and Jal instruction of abstract machine}
\end{figure}

As the abstract machine serves as a specification to a machine with
\CFI, a more intuitive definition of it would not include the \ok bit
and would only allow the Jump and Jal instructions to step if they do
not violate the control-flow graph. However, this abstract machine
would not allow for any violations to occur unlike our enforcement
mechanism for the symbolic and the concrete machine and would lead to
more complex simulation proofs, therefore we do not favor it.

The abstract machine also allows for monitor services to be included,
although the \CFI enforcement mechanism does not require any. We
assume that a monitor service is a privileged action and that it's
execution does not violate the control-flow of the program. Execution
of a monitor service is done simply by jumping to it's address, there
is no separate instruction. As with all other instructions, execution
of the monitor service is only allowed if the \ok bit is set to true.

\begin{figure}[H]
\infrule[Service]{
  \pc \not\in \ii{dom(\imem)} \andalso \pc \not\in \ii{dom(\dmem)} \andalso
  \ii{get\_service}~\pc = (addr,f) \\
  f~\acfistat{\imem}{\dmem}{\reg}{\pc}{\ii{true}}=
    \acfistat{\imem}{\dmem'}{\reg'}{\pc'}{\ii{true}}
  }{\stepn{\acfistat{\imem}{\dmem}{\reg}{\pc}{\ii{true}}}{
    \acfistat{\imem}{\dmem'}{\reg'}{\pc'}{\ii{true}}}}
\caption{Step rule for monitor services of abstract machine}
\end{figure}

\TODO{Put all rules in appendix?}

\subsection{Attacker model}\label{sec:abstract_attacker}

The attacker for the abstract machine is allowed to change the
contents of the data memory and the registers but not the rest of the state.

\begin{figure}[ht]
\infrule{
  \ii{dom}~\dmem = \ii{dom}~\dmem' \andalso
  \ii{dom}~\reg = \ii{dom}~\reg'
}{
  \acfistat{\imem}{\dmem}{\reg}{\pc}{\ok} \to_a^A
  \acfistat{\imem}{\dmem'}{\reg'}{\pc}{\ok}
}
\caption{Attacker model for the abstract machine}
\end{figure}

\subsection{Allowed control-flows for the abstract machine}
\label{sec:abstract_flow}

We can construct a function \SUCC{A} for the abstract machine
that represents the set of allowed control-flows for all
instructions, by extending the set of allowed jumps \CFG we
introduced earlier.

Below we give a specification of the \SUCC{A} function for the abstract machine,
in form of inference rules. A function is defined in the actual Coq development.

\begin{figure}[ht]
\infrule[IndirectFlows]{
  \imem[\pc] = i 
  \andalso \ii{decode}~i \in \lbrace \ii{Jal~r}, \ii{Jump~r} \rbrace
  \andalso (\pc,\pc') \in \Jm
  }{
  (\acfistat{\imem}{\dmem}{\reg}{\pc}{\ok} ,
  \acfistat{\imem}{\dmem'}{\reg'}{\pc'}{\ok}) \in \SUCCm{A}}
\bigskip

\infrule[ConditionalFlows]{
  \imem[\pc] = i 
  \andalso \ii{decode}~i = \ii{Bnz~r~imm}\\
   (\pc' = \pc + 1) \vee (\pc' = \pc + imm)
  }{
  (\acfistat{\imem}{\dmem}{\reg}{\pc}{\ok} ,
  \acfistat{\imem}{\dmem'}{\reg'}{\pc'}{\ok}) \in \SUCCm{A}}
\bigskip

\infrule[NormalFlows]{
  \imem[\pc] = i 
  \andalso 
  \ii{decode}~i \not\in \lbrace \ii{Jal~r}, \ii{Jump~r}, \ii{Bnz~r~imm}, 
  \varnothing \rbrace
  \\ \pc' = \pc + 1
  }{
  (\acfistat{\imem}{\dmem}{\reg}{\pc}{\ok} ,
  \acfistat{\imem}{\dmem'}{\reg'}{\pc'}{\ok}) \in \SUCCm{A}}
\bigskip

\infrule[ServiceFlows]{
  \imem[\pc] = \varnothing \andalso \dmem[\pc] = \varnothing\\
  \ii{get\_service}~\pc = (addr,f)
  }{
  (\acfistat{\imem}{\dmem}{\reg}{\pc}{\ok} ,
  \acfistat{\imem}{\dmem'}{\reg'}{\pc'}{\ok}) \in \SUCCm{A}}
\caption{Allowed control-flows for instructions of the abstract machine}
\end{figure}

Notice that a monitor service is allowed to return anywhere. As we
mentioned before, monitor services, execute in a protected by the
monitor environment where we assume that an attacker who can only
tamper the machine at the user level cannot interfere.

\subsection{Stopping predicate for the abstract machine}
\label{sec:abstract_stopping}

Finally, we define what it means for the abstract machine to be ``stopping'' by
defining a predicate on execution traces:
\begin{enumerate}
\item All states in the trace are stuck with respect to normal steps
  (\stepn{}{})
\item All steps in the trace are attacker steps (\stepa{}{}{})
\end{enumerate}

\subsection{CFI proof for the Abstract Machine}\label{abstract_proof}

Regarding initial states, we only require that the \ok bit is set to true.
We can now instantiate the class of the machines defined in 
\ref{fig:cfi_machine}, with the abstract machine and that the abstract machine
has \CFI according to definition \ref{definition:CFI}.
We first prove a helpful lemma.

\begin{lemma}[Step Intersection]
\label{lemma:attacker_no_v}
For all states \textit{st st'} such that \stepa{st}{st'}{A}
and \stepn{st}{st'}, $(\mathit{st},\mathit{st'}) \in \SUCCm{A}$.
\end{lemma}

\begin{proof}
~
\begin{itemize}
\item By the relation \stepn{st}{st'} we know that the \ok bit
of \textit{st} is set to true. 
\item The relation \stepa{st}{st'}{A} retains the \ok bit of
\textit{st}, therefore \textit{st'} has the \ok bit set to true.
\item It trivially follows from the definition of \SUCC{A} that
$(\mathit{st},\mathit{st'}) \in \SUCCm{A}$.
\end{itemize}
\end{proof}

\begin{thm}[Abstract CFI]\label{thm:CFIabstract}
The abstract machine has the \CFI property defined by \ref{definition:CFI}.
\end{thm}

\begin{proof} 
  ~ The proof proceeds by induction on the execution trace.
  \begin{itemize}
  \item \textbf{Base Case} In this case the execution trace is made up
    of a single step \step{st}{st'}. We proceed with case analysis on
    the step.
    \begin{itemize}
    \item \textbf{Attacker Step} By lemma \ref{lemma:attacker_no_v} we
      note that an attacker step cannot be a normal step outside the
      \SUCC{A} relation. Thus in this case the whole trace has \CFI
      according to \ref{definition:traceHasCfi}.
    \item \textbf{Normal Step} By case analysis, if
      $(\mathit{st},\mathit{st'}) \in \SUCCm{A}$ then trivially the
      whole trace has \CFI. Otherwise $(\mathit{st},\mathit{st'}) \not
      \in \SUCCm{A}$ and the sub-traces \textit{st} and \textit{st'}
      vacuously have \CFI. In addition the sub-trace \textit{st'} is
      stopping, as the \ok bit of \textit{st'} is set to false and the
      state is stuck with respect to normal steps.
    \end{itemize}
  \item \textbf{Inductive Case} In this case the execution trace is
    extended by an additional step at it's beginning
    $\mathbf{s_0 \to s_1} \to s_2 \to \ldots \to s_n$. 
    By the induction hypothesis either:
    \begin{itemize}
    \item The trace $s_1 \to s_2 \to \ldots \to s_n$ has \CFI, by case
      analysis if $(\mathit{s_0},\mathit{s_1}) \in \SUCCm{A}$ the
      whole trace has \CFI. Otherwise $(\mathit{s_0},\mathit{s_1})
      \not \in \SUCCm{A}$, the sub-trace $s_0$ vacuously has \CFI and
      the sub-trace $s_1 \to \ldots \to s_n$ has \CFI by the induction
      hypothesis. Additionally, the sub-trace $s_1 \to \ldots \to s_n$
      is stopping because:
      \begin{itemize}
        \item The whole trace is made up of attacker steps.
           Since $(\mathit{s_0},\mathit{s_1}) \not \in \SUCCm{A}$
           the \ok bit of $s_1$ will be set to false and a normal
           step is not allowed by the operational semantics,
           while attacker steps retain the \ok bit.
         \item The whole trace is stuck with respect to normal steps.
           Trivial from the above.
       \end{itemize}
     \item There exists a step $\stepn{s_{v1}}{s_{v2}}$ such that
       $(\mathit{s_{v1}},\mathit{s_{v2}}) \not \in \SUCCm{A}$ and
       the sub-traces $s_1 \to \ldots \to s_{v1}$ and
       $s_{v2} \to \ldots \to s_n$ both have \CFI and the later
       is also a stopping trace. 
       \begin{itemize}
         \item If $(\mathit{s_0},\mathit{s_1}) \in \SUCCm{A}$ then
           \ref{definition:CFI} still holds and the sub-trace $s_1 \to \ldots 
           \to s_{v1}$ is extended by one step to $s_0 \to \ldots \to s_{v1}$.
         \item Otherwise the \ok bit for $s_1$ is set to false and the
           rest of the trace is stuck with respect to normal steps. However from
           the induction hypothesis we know that $\stepn{s_{v1}}{s_{v2}}$, which
           is a contradiction.
         \end{itemize}
       \end{itemize}
  \end{itemize}
\end{proof}

\section{The Symbolic  Machine}\label{sec:symbolic_cfi}

The symbolic machine was described in \ref{sec:symbolic}. Unlike the
abstract machine, the symbolic machine has one memory and the
distinction between data and executable instructions is made through
tags, in a fashion similar to what was shown in \ref{sec:nwc_nxd} and
\ref{sec:cfi_fine}. We instantiate the symbolic machine, according to
the aforementioned sections, with a set of tags
\TAGS{\DATA,\INSTR{\ii{id}}, \INSTR{\bot}} where id is drawn from the
class of identifiers \ref{fig:cfi_id}.

\begin{figure}[!htpb]
  \coqdockw{Context} \{\coqdocvar{ids} : @\coqdocvar{cfi\_id}
  \coqdocvar{t}\}.\coqdoceol \coqdocemptyline \coqdocnoindent
  \coqdockw{Inductive} \coqdocvar{cfi\_tag} : \coqdockw{Type}
  :=\coqdoceol \coqdocnoindent \ensuremath{|} \coqdocvar{INSTR} :
  \coqdocvar{option} \coqdocvar{id} \ensuremath{\rightarrow}
  \coqdocvar{cfi\_tag}\coqdoceol \coqdocnoindent \ensuremath{|}
  \coqdocvar{DATA} : \coqdocvar{cfi\_tag}.\coqdoceol
\caption{Coq definition of Symbolic tags}
\end{figure}

Although enforcement of \CFI does not require any monitor services we
expose the monitor services mechanism and we check whether calls to
each monitor service are allowed or not according to the control-flow
graph. This is done by assuming a lookup-table of monitor services
where each entry has a tag that is used to check for control-flow
violations and a semantic function from symbolic state to symbolic
state which produces the new machine state after execution of the
system call, as shown in \ref{fig:symbolic_step}.

We do not need any internal state for this micro-policy therefore,
only the transfer function is left to implement.

\subsection{Transfer Function}\label{sec:transfer_fun}

We implement the \TRANSFER function based on the rules found in
\ref{sec:cfi_fine}, using Gallina to define a function mapping
input vectors (mvector) to output vectors (rvector).

\begin{figure}[htbp!]
\begin{coqdoccode}
  \coqdocnoindent \coqdockw{Definition} \coqdocvar{cfi\_handler}
  \coqdocvar{umvec} :=\coqdoceol \coqdocindent{1.00em}
  \coqdockw{match} \coqdocvar{umvec} \coqdockw{with}\coqdoceol
  \coqdocindent{1.00em} \ensuremath{|} \coqdocvar{mkMVec}
  \coqdocvar{JUMP} (\coqdocvar{INSTR} (\coqdocvar{Some}
  \coqdocvar{n})) (\coqdocvar{INSTR} (\coqdocvar{Some} \coqdocvar{m}))
  \coqdocvar{\_}\coqdoceol \coqdocindent{1.00em} \ensuremath{|}
  \coqdocvar{mkMVec} \coqdocvar{JAL} (\coqdocvar{INSTR}
  (\coqdocvar{Some} \coqdocvar{n})) (\coqdocvar{INSTR}
  (\coqdocvar{Some} \coqdocvar{m})) \coqdocvar{\_}
  \ensuremath{\Rightarrow}\coqdoceol \coqdocindent{2.00em}
  \coqdockw{if} \coqdocvar{cfg} \coqdocvar{n} \coqdocvar{m}
  \coqdockw{then} \coqdocvar{Some} (\coqdocvar{mkRVec}
  (\coqdocvar{INSTR} (\coqdocvar{Some} \coqdocvar{m}))
  \coqdocvar{DATA})\coqdoceol \coqdocindent{2.00em} \coqdockw{else}
  \coqdocvar{None}\coqdoceol \coqdocindent{1.00em} \ensuremath{|}
  \coqdocvar{mkMVec} \coqdocvar{JUMP} \coqdocvar{DATA}
  (\coqdocvar{INSTR} (\coqdocvar{Some} \coqdocvar{n}))
  \coqdocvar{\_}\coqdoceol \coqdocindent{1.00em} \ensuremath{|}
  \coqdocvar{mkMVec} \coqdocvar{JAL} \coqdocvar{DATA}
  (\coqdocvar{INSTR} (\coqdocvar{Some} \coqdocvar{n})) \coqdocvar{\_}
  \ensuremath{\Rightarrow} \coqdoceol \coqdocindent{2.00em}
  \coqdocvar{Some} (\coqdocvar{mkRVec} (\coqdocvar{INSTR}
  (\coqdocvar{Some} \coqdocvar{n})) \coqdocvar{DATA})\coqdoceol
  \coqdocindent{1.00em} \ensuremath{|} \coqdocvar{mkMVec}
  \coqdocvar{JUMP} \coqdocvar{DATA} (\coqdocvar{INSTR}
  \coqdocvar{None}) \coqdocvar{\_}\coqdoceol \coqdocindent{1.00em}
  \ensuremath{|} \coqdocvar{mkMVec} \coqdocvar{JAL} \coqdocvar{DATA}
  (\coqdocvar{INSTR} \coqdocvar{None}) \coqdocvar{\_}
  \ensuremath{\Rightarrow}\coqdoceol \coqdocindent{2.00em}
  \coqdocvar{None}\coqdoceol \coqdocindent{1.00em} \ensuremath{|}
  \coqdocvar{mkMVec} \coqdocvar{STORE} (\coqdocvar{INSTR}
  (\coqdocvar{Some} \coqdocvar{n})) (\coqdocvar{INSTR}
  (\coqdocvar{Some} \coqdocvar{m})) [\coqdocvar{\_} ; \coqdocvar{\_} ;
  \coqdocvar{DATA}] \ensuremath{\Rightarrow}\coqdoceol
  \coqdocindent{2.00em} \coqdockw{if} \coqdocvar{cfg} \coqdocvar{n}
  \coqdocvar{m} \coqdockw{then} \coqdocvar{Some} (\coqdocvar{mkRVec}
  \coqdocvar{DATA} \coqdocvar{DATA}) \coqdockw{else}
  \coqdocvar{None}\coqdoceol \coqdocindent{1.00em} \ensuremath{|}
  \coqdocvar{mkMVec} \coqdocvar{STORE} \coqdocvar{DATA}
  (\coqdocvar{INSTR} \coqdocvar{\_}) [\coqdocvar{\_} ; \coqdocvar{\_}
  ; \coqdocvar{DATA}] \ensuremath{\Rightarrow} \coqdoceol
  \coqdocindent{2.00em} \coqdocvar{Some} (\coqdocvar{mkRVec}
  \coqdocvar{DATA} \coqdocvar{DATA})\coqdoceol \coqdocindent{1.00em}
  \ensuremath{|} \coqdocvar{mkMVec} \coqdocvar{STORE} \coqdocvar{\_}
  \coqdocvar{\_} \coqdocvar{\_} \ensuremath{\Rightarrow}
  \coqdocvar{None}\coqdoceol \coqdocindent{1.00em} \ensuremath{|}
  \coqdocvar{mkMVec} \coqdocvar{\_} (\coqdocvar{INSTR}
  (\coqdocvar{Some} \coqdocvar{n})) (\coqdocvar{INSTR}
  (\coqdocvar{Some} \coqdocvar{m})) \coqdocvar{\_}
  \ensuremath{\Rightarrow} \coqdoceol \coqdocindent{2.00em}
  \coqdockw{if} \coqdocvar{cfg} \coqdocvar{n} \coqdocvar{m}
  \coqdockw{then} \coqdocvar{Some} (\coqdocvar{mkRVec}
  \coqdocvar{DATA} \coqdocvar{DATA}) \coqdockw{else}
  \coqdocvar{None}\coqdoceol \coqdocindent{1.00em} \ensuremath{|}
  \coqdocvar{mkMVec} \coqdocvar{\_} \coqdocvar{DATA}
  (\coqdocvar{INSTR} \coqdocvar{\_}) \coqdocvar{\_}
  \ensuremath{\Rightarrow} \coqdoceol \coqdocindent{2.00em}
  \coqdocvar{Some} (\coqdocvar{mkRVec} \coqdocvar{DATA}
  \coqdocvar{DATA})\coqdoceol \coqdocindent{1.00em} \ensuremath{|}
  \coqdocvar{mkMVec} \coqdocvar{\_} \coqdocvar{\_} \coqdocvar{\_}
  \coqdocvar{\_} \ensuremath{\Rightarrow} \coqdocvar{None}\coqdoceol
  \coqdocindent{1.00em} \coqdockw{end}.\coqdoceol
\end{coqdoccode}
\caption{Transfer function for symbolic machine in Gallina}
\label{fig:transfer_coq}
\end{figure}
\TODO{Should I remove the aggressive capitilization above?
It may make it less painful on the eye...}


Although, the rules in \ref{sec:cfi_fine} were fairly simply,
expressing them using Gallina's pattern matching increased their
size. We also experimented, with different ways of writing the
transfer function but we decided to stick with the definition above as
it's the most straightforward. It's worth to note that bugs in the
above definition were easily made apparent when proving theorems
involving the transfer function. In fact, an ``interesting''
experiment was to re-define the above function in a different way and
prove the two equivalent. It took two iterations before getting both
functions to agree and although for small definitions like the one
above, testing or manually reviewing the code will reveal most if not
all bugs, the importance of formal verification in software
engineering and critical software is made obvious even for definitions
that may seem trivial at first. The correctness of the transfer
function will come from simulation proofs between the abstract and the
symbolic machine.

\subsection{Attacker model}\label{sec:symbolic_attacker}

Similar to the abstract attacker, the symbolic attacker can change all words
tagged as \DATAname but not the ones tagged as \INSTRname. This is expressed by
the following relations:

\begin{figure}[htbp]
\infrule[AttackData]{
  }{\stepa{\atom{w_1}{\DATA}}{\atom{w_2}{\DATA}}{S}}
\bigskip
\infrule[AttackInstr]{
  }{\stepa{\atom{w_1}{\INSTR{id}}}{\atom{w_1}{\INSTR{id}}}{S}}
\caption{Attacker capabilities}
\label{fig:symbolic_attacker_atom}
\end{figure}

These attacker capabilities on symbolic atoms are lifted to
the memory and registers by a pointwise extension.
\TODO{be more specific about pointwise?}

\begin{figure}[htbp]
\infrule{
  \stepa{\mem}{\mem'}{S} \andalso
  \stepa{\reg}{\reg'}{S}
}{
  \stepa{\astat{\mem}{\reg}{\atom{\pc}{t_\pc}}{\extra}{}}
  {\astat{\mem'}{\reg'}{\atom{\pc}{t_\pc}}{\extra}{}}
  {S}
}
\caption{Attacker model for the Symbolic machine}
\label{fig:symbolic_attacker}
\end{figure}

\subsection{Allowed control-flows for the Symbolic Machine}
\label{sec:symbolic_flow}

Similar to the abstract machine \ref{sec:abstract_flow}, we construct
\SUCC{S} for the symbolic machine (figure \ref{fig:symbolic_succ}) by
extending the set of allowed jumps \CFG.

\begin{figure}[htbp!]
\small
\infrule[IndirectFlows]{
  \mem[\pc] = \atom{\textit{i}}{\ti} 
  \andalso \textit{decode~i} \in \lbrace \ii{Jal~r}, \ii{Jump~r} \rbrace\\
  \mem[\pc'] = \atom{\textit{i}'}{\ti'} 
    \andalso \ti = \INSTR{src}
    \andalso \ti' = \INSTR{dst}\\
    (\textit{src},\textit{dst}) \in \CFGm
  }{
  (\acfistat{\mem}{\reg}{\pc}{\extra}{},
  \acfistat{\mem}{\reg}{\pc'}{\extra}{}) \in \SUCCm{S}}
\bigskip

\infrule[IndirectFlows2]{
  \mem[\pc] = \atom{\textit{i}}{\ti}
  \andalso \textit{decode~i} \in \lbrace \ii{Jal~r}, \ii{Jump~r} \rbrace\\
  \mem[\pc'] = \varnothing 
    \andalso \textit{get\_service}~\pc = (\ti',f)\\
    \ti = \INSTR{src} \andalso \ti' = \INSTR{dst}
    \\ (\textit{src},\textit{dst}) \in \CFGm
  }{
  (\acfistat{\mem}{\reg}{\pc}{\extra}{},
  \acfistat{\mem}{\reg}{\pc'}{\extra}{}) \in \SUCCm{S}}
\bigskip

\infrule[ConditionalFlows]{
  \mem[\pc] = \atom{\textit{i}}{\ti}
  \andalso \textit{decode~i} = \ii{Bnz~r~imm}\\
   (\pc' = \pc + 1) \vee (\pc' = \pc + imm)
  }{
  (\acfistat{\mem}{\reg}{\pc}{\extra}{},
  \acfistat{\mem}{\reg}{\pc'}{\extra}{}) \in \SUCCm{S}}
\bigskip

\infrule[NormalFlows]{
  \mem[\pc] = \atom{\textit{i}}{\ti}
  \andalso 
  \ii{decode}~i \not\in \lbrace \ii{Jal~r}, \ii{Jump~r}, \ii{Bnz~r~imm}, 
  \varnothing \rbrace
  \\ \pc' = \pc + 1
  }{
  (\acfistat{\mem}{\reg}{\pc}{\extra}{},
  \acfistat{\mem'}{\reg'}{\pc'}{\extra}{}) \in \SUCCm{S}}
\bigskip

\infrule[ServiceFlows]{
  \mem[\pc] = \varnothing \andalso \textit{get\_service}~\pc = (\ti',f)\\
  }{
  (\acfistat{\mem}{\reg}{\pc}{\extra}{},
  \acfistat{\mem'}{\reg'}{\pc'}{\extra'}{}) \in \SUCCm{S}}
\caption{Allowed control-flows for instructions of the symbolic machine}
\label{fig:symbolic_succ}
\end{figure}

\subsection{Initial states of the Symbolic Machine}
\label{sec:symbolic_initial}

For the symbolic machine, we do require that certain tagging
conventions are respected initially. Additionally we prove that
these initial conditions are invariants of the machine and they
are preserved at every (normal or attacker) step.

These invariants are required for backward simulation between
the symbolic and the abstract machine. More invariants may
be required for forward simulation.

\TODO{hopefully try to get to forward simulation and
see what's the situation there.}

\begin{definition}[Instructions Tagged]\label{definition:instructions_tagged}
  ~ For all addresses \textit{addr} in the memory such that
  $$\mem[\mathit{addr}] = \atom{\mathit{i}}{\INSTR{id}}$$ \textit{addr}
  is in the domain of \emph{word\_to\_id} and additionally
  $$word\_to\_id ~\mathit{addr} = \mathit{id}$$
\end{definition}

\begin{definition}[Entry Points Tagged]\label{definition:entry_tagged}
  ~ For all addresses \textit{addr} such that 
  \begin{align*}
  & \mem[addr] = \varnothing\\
  & get\_service~addr = (\mathit{it},f)\\
  & \mathit{it} = \INSTR{id}
  \end{align*}
  \textit{addr} is in the domain of \emph{word\_to\_id} and additionally
  $$word\_to\_id ~\mathit{addr} = \mathit{id}$$
\end{definition}

\begin{definition}[Valid Jumps Tagged]\label{definition:valid_jmp_tagged}
  For all addresses \textit{saddr, taddr} such that
  $$(\mathit{saddr},\mathit{taddr}) \in \Jm$$ it holds that
  $$\exists \mathit{i}, \mem[saddr] = \atom{\mathit{i}}{\INSTR{(word\_to\_id~saddr)}} $$
  and either $$\exists \mathit{i'}, \mem[taddr]
  =\atom{\mathit{i'}}{\INSTR{word\_to\_id~taddr}}$$ or
  \begin{align*}
  & \mem[taddr] = \varnothing\\ 
  & \exists (\mathit{it},f), get\_service~addr = (\mathit{it},f)\\
  & \mathit{it} = \INSTR{(word\_to\_id~taddr)}
  \end{align*}
\end{definition}

We define a predicate \emph{initial} on symbolic states as
the conjunction of the above invariants and the
proposition that the tag on the \pc is set to \DATAname.

\begin{figure}[ht]
  \coqdockw{Definition} \coqdocvar{invariants} \coqdocvar{st}
  :=\coqdoceol \coqdocindent{1.00em} \coqdocvar{instructions\_tagged}
  (\coqdocvar{Symbolic.mem} \coqdocvar{st})
  \ensuremath{\land}\coqdoceol \coqdocindent{1.00em}
  \coqdocvar{valid\_jmp\_tagged} (\coqdocvar{Symbolic.mem}
  \coqdocvar{st}) \ensuremath{\land}\coqdoceol \coqdocindent{1.00em}
  \coqdocvar{entry\_points\_tagged} (\coqdocvar{Symbolic.mem}
  \coqdocvar{st}).\coqdoceol \coqdocemptyline \coqdocnoindent

  \coqdockw{Definition} \coqdocvar{initial} (\coqdocvar{s} :
  \coqdocvar{Symbolic.state} \coqdocvar{t}) :=\coqdoceol
  \coqdocindent{1.00em} (\coqdocvar{common.tag}
  (\coqdocvar{Symbolic.pc} \coqdocvar{s})) = \coqdocvar{DATA}
  \ensuremath{\land}\coqdoceol \coqdocindent{1.00em}
  \coqdocvar{invariants} \coqdocvar{s}.\coqdoceol
\caption{Initial symbolic states predicate}
\end{figure}

It's straightforward by the semantics of the step relations to prove
that both normal and attacker steps preserve each of the invariants.
We only need to assume that this holds for monitor services (\IE if we
were to provide some monitor services they would have to preserve
these invariants).

\begin{figure}[ht]
  \coqdockw{Lemma} \coqdocvar{invariants\_preserved\_by\_step}
  \coqdocvar{st} \coqdocvar{st'} :\coqdoceol \coqdocindent{1.00em}
  \coqdocvar{invariants} \coqdocvar{st}
  \ensuremath{\rightarrow}\coqdoceol \coqdocindent{1.00em}
  \coqdocvar{step\_n} \coqdocvar{st}
  \coqdocvar{st'} \ensuremath{\rightarrow}\coqdoceol
  \coqdocindent{1.00em} \coqdocvar{invariants}
  \coqdocvar{st'}.\coqdoceol

  \coqdocemptyline \coqdocnoindent \coqdockw{Lemma}
  \coqdocvar{invariants\_preserved\_by\_step\_a} \coqdocvar{st}
  \coqdocvar{st'} :\coqdoceol \coqdocindent{1.00em}
  \coqdocvar{invariants} \coqdocvar{st}
  \ensuremath{\rightarrow}\coqdoceol \coqdocindent{1.00em}
  \coqdocvar{step\_a} \coqdocvar{st} \coqdocvar{st'}
  \ensuremath{\rightarrow}\coqdoceol \coqdocindent{1.00em}
  \coqdocvar{invariants} \coqdocvar{st'}.\coqdoceol
\caption{Symbolic invariants preservation}
\end{figure}

\subsection{Stopping predicate for the Symbolic Machine}
\label{sec:symbolic_stopping}

Similar to the abstract machine, we say that an execution trace of the
symbolic machine is stopping if:
\begin{enumerate}
\item All states in the trace are stuck with respect to normal steps
  (\stepn{}{})
\item All steps in the trace are attacker steps (\stepa{}{}{})
\end{enumerate}