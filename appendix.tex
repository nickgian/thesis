
\section{Control-Flow Integrity Micro-Policy}

We begin with a micro-policy targeting control-flow hijacking attacks,
in which an attacker exploits a low-level vulnerability (e.g. a buffer
or integer overflow) to gain full control of a target program~\cite{
  ShellcoderHandbook, Szekeres2013, Smashing1996, SkyLined, PincusB04,
  Sotirov07, DanielHM08, AfekS07, Dobrovitski03}.
%
As a first line of defense, we can use tags to make code non-writable
(NWC) and data non-executable (NXD), preventing the injection and
execution of an attacker payload.
%
This useful defense appears in various forms in existing systems.
However, it does not prevent code-reuse attacks~\cite{Newsham1997,
  SolarDesigner1997, McDonald1999, Shacham07, Checkoway2010,
  Buchanan2008, SnowMDDLS13, outofcontrol_ieeesp2014} such as return- or
jump-oriented programming~\cite{Shacham07, Checkoway2010}, where the
attacker chains together existing code snippets (``gadgets'') to induce
arbitrary malicious behavior.
%
We therefore use tags to enforce fine-grained {\em control-flow integrity
  (CFI)}~\cite{AbadiBEL09, ZhaoLSR11, Zhang2013, CriswellDA14, NiuT14,
  ZhaoLSR11, CriswellDA14} on top of basic NWC and NXD protection.
%
This ensures that all indirect control flows (computed jumps) adhere
to a fixed control flow graph (CFG).

\newcommand{\CODEname}[0]{\ii{Code}}
\newcommand{\CODE}[1]{\CODEname~#1}

\makeatletter
\newdimen\OPCODEwidth
\OPCODEwidth .5in
\newdimen\RULEwidth
\newcommand{\TRUE}{\text{\tt true}}
\newcommand{\RULE}[9]{
\gdef\RULEARROW{\ifthenelse{\equal{#7}{}}{\Rightarrow}{\rightarrow}}  % HACK!
\gdef\RULEINPUT{(#2,#3,#4,#5,#6)}
\gdef\RULEOUTPUT{(#8,#9)}
\gdef\RULECOND{%
  \ifthenelse{\equal{#7}{}}{}%
             {\ifthenelse{\equal{#7}{\TRUE}}{}%
                         {\mathrm{\;if\;}#7}}%
}
&& \hspace*{-18.5em}
  \hbox to 1in {
      \hbox to \OPCODEwidth {\ifx&#1&\else$#1$\ : \fi}
      % See how big it is on one line
      \setbox \@tempboxa \hbox{$\RULEINPUT \RULEARROW \RULEOUTPUT \RULECOND$}
      \RULEwidth \wd\@tempboxa
      % Does it fit?
      \ifdim \RULEwidth < 2.4in
        % Use it
        \box\@tempboxa
      \else\ifdim \RULEwidth < 4.8in
        % Put it on two lines
        $ \hspace*{-1.3em}
        \begin{array}[t]{@{}l@{\ }l}
            & \RULEINPUT \\
            \RULEARROW
            & \RULEOUTPUT\RULECOND
        \end{array}
       $
      \else
        % Put it on three lines
        $ \hspace*{-1.3em}
        \begin{array}[t]{@{}l@{}l}
            & \ \RULEINPUT \\
            \RULEARROW
            & \ \RULEOUTPUT \\
            & \RULECOND
        \end{array}
       $
     \fi\fi
  }
}

\newcommand{\RULEWITHPREMISE}[9]{
\typicallabel{}
  \infrule{#7}{\ii{#1} : (#2, #3, #4, #5, #6) \to (#8,#9)}
\typicallabel{MkKey}
}

We use tags to distinguish between code and data.
%
Tags on memory and the PC are drawn from the set
%
$
\DATA \;|\; \CODE{\ii{addr}} \;|\; \CODE{\bot}
$
(registers are always tagged $\DATA$).
%
To simplify the CFG conformance checks, instructions that are the
source or target of indirect control flows are tagged with
$\CODE{\ii{addr}}$, where $\ii{addr}$ is the address of that
instruction in memory.
%
For example, a $\ii{Jump}$ instruction stored at address $500$ is
tagged $\CODE{500}$.
%
All other instructions are tagged $\CODE{\bot}$.
%
\aaa{Actually, we can't use the instruction's address on the tag if we
  are to have the same number of bits on words and tags. Maybe change
  to ``id''?}

We write transfer functions as a collection of {\em
  symbolic rules}~\cite{popl2015, pump_hasp2014}.
% \ch{Should we remove all references to \cite{pump_ccs2014}?}
% BCP: yes
%
(The PUMP hardware uses a lower-level {\em concrete rule} format, 
described in \autoref{implementation}.)
%
Each symbolic rule has the form
%
% \begin{eqnarray*}
% \RULE
%   {\mathit{opcode}}
%   {\mathit{PC}}{\mathit{CI}}{\mathit{OP_1}}{\mathit{OP_2}}{\mathit{OP_3}}
%   {\TRUE}
%   {\mathit{PC'}}{\mathit{R'}}
% \end{eqnarray*}
%
\newcommand{\INLINERULE}[8]{\mathit{#1}
\mathrel{:}\allowbreak
 ({\mathit{#2}},\allowbreak{\mathit{#3}},\allowbreak{\mathit{#4}},\allowbreak{\mathit{#5}},\allowbreak{\mathit{#6}})
 \rightarrow\allowbreak ({\mathit{#7}},\allowbreak{\mathit{#8}})
}%
``$\INLINERULE{opcode}{PC}{CI}{OP_1}{OP_2}{OP_3}{PC'}{R'}$,''
%
which says that the rule matches on the given {\it opcode} together
with the metadata tags on the program counter ($\mathit{PC}$), the
current instruction ($\mathit{CI}$), and on up to three operands
($\mathit{OP_1}$ to $\mathit{OP_3}$).
%
If the rule applies, the right-hand side determines how to update the
tags on the PC ($\mathit{PC'}$) and on the result of the operation
($\mathit{R'}$).
%
We write ``$-$'' to indicate input or output fields that are ignored
(``wildcard'').
%
\chrev{All instructions that are not explicitly allowed by the
  symbolic rules are disallowed.}%
\aaa{We should choose only one of $-$ or $\_$ for our wildcard and use
  it consistently (cf. the ``Store'' rule below)}%

The CFI transfer function enforces that only memory locations tagged
$\DATA$ can be modified (NWC) and only instructions fetched from
locations tagged $\CODEname$ can be executed (NXD).
%
The symbolic rule for the $\ii{Store}$ instruction illustrates both
these points:
%
\begin{eqnarray*}
\INLINERULE
  {\ii{Store}}
  {\DATA}{\CODE{\text{\textunderscore}\,}}{-}{-}{\DATA}
  {\DATA}{-}
\end{eqnarray*}
%
It requires the fetched $\ii{Store}$ instruction to be tagged
$\CODEname$ and the written location to be tagged $\DATA$.
%
This rule only applies when the PC is also tagged $\DATA$, which is
the case when the $\ii{Store}$ instruction was reached by direct
control flow (not a computed jump).
%
The rule preserves the $\DATA$ tag on the PC, since $\ii{Store}$ is
not a computed jump.
%
Performing a computed jump (\EG using \ii{Jal}, a
jump-and-link instruction) requires that the current instruction be
tagged $\CODE{\ii{src}}$ for some address $\ii{src}$.
%
\begin{eqnarray*}
\INLINERULE
  {\ii{Jal}}
  {\DATA}{\CODE{\ii{src}}}{-}{-}{-}
  {\CODE{\ii{src}}}{-}
\end{eqnarray*}
%
This rule copies $\CODE{\ii{src}}$ to the PC tag to indicate
that a jump from \ii{src} has just occurred.
%
Only on the next instruction do we have enough information about the
destination in the tags to check that the jump is indeed allowed by
the CFG.
%
For this we add a second rule for \ii{Store}, dealing with the case
where it is the target of a jump and thus the PC is tagged
$\CODE{\ii{src}}$.
%
\RULEWITHPREMISE
  {\ii{Store}}
  {\CODE{\ii{src}}}{\CODE{\ii{tgt}}}{-}{-}{\DATA}
  {(\ii{src},\ii{tgt}) \in \ii{CFG}}
  {\DATA}{-}
%
\aaa{Maybe we could discuss here a little bit why we verify the jump
  on the next instruction, as opposed to when the jump is
  performed. This might get some people confused, since this is not
  very natural and fundamentally driven by our current design of the
  PUMP. Even Nick wanted to know if we couldn't do it differently.}%
  The premise of this rule ensures that the source and target of the
  just-performed jump are allowed by the CFG.
%
  We add a similar rule for each instruction, including jumps (since
  the target of a computed jump can itself be another computed
  jump):
%
\RULEWITHPREMISE
{\ii{Jal}}
{\CODE{\ii{src}}}{\CODE{\ii{tgt-src}}}{-}{-}{-}
{(\ii{src},\ii{tgt-src}) \in \ii{CFG}}
{\CODE{\ii{tgt-src}}}{-}

This micro-policy enforces fine-grained CFI~\cite{NiuT14,
  outofcontrol_ieeesp2014, CriswellDA14}, not coarse-grained
approximations~\cite{AbadiBEL09, Zhang2013} that are potentially vulnerable
to attack~\cite{outofcontrol_ieeesp2014}.  Indeed, we recently
proved~\cite{popl2015} that this micro-policy enforces a variant of the CFI
property introduced by Abadi~\ETAL\cite{AbadiBEL09}, ensuring that all
indirect control flows adhere to a fixed CFG.
%
Recent simulations of an optimized PUMP architecture~\cite{pump_asplos2015}
show that the CFI policy can be enforced with around 3\% average runtime
overhead.
