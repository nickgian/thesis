\chapter{Micro-policies: Verified, Hardware-Assisted Monitors}\label{ch:policies}
Current hardware provides vary limited (coarse-grained)
security mechanisms \TODO{name some;
  4 protection rings, page-level memory protection via virtual memory},
leaving most of the work to the software. This requires that the software
performs various sanity-checks during an execution and that it carefully
maintains various safety and security invariants, a tedious and error-prone task
that results in security holes and often in high runtime performance penalties.

Many potentially effective mitigation techniques are not deployed because of the
performance overhead they incur. Another requirement for deployment of a
protection mechanism is the compatibility with existing executables and
the degree of intervention required by a human.
Usually even making slight changes to a code and redistributing has high cost
and the protection mechanism is likely to see very low adoption.

The lack of efficient and effective generic ways to enforce security policies,
forces programmers to protect their own code, a task which is not trivial even
for the small and simply programs. As a result most, if not all, software
carries weaknesses which can be exploited by an attacker. ``Safe'' languages,
automate some of the checks required and eases the work of the programmer,
for example by implementing array bounds checking or by disallowing
pointer-arithmetic. However these solutions only reduce the chance of
introducing exploitable bugs in a program and do not enforce stricter,
more effective policies such as Control Flow Integrity
or complete Memory Safety (spatial/temporal protection for heap and stack).
In addition, we still need effective and efficient protection mechanisms for a
plethora of software written in unsafe languages such as C.

\section{Micro-Policies}\label{sec:micropolicies}

A wide range of security policies can be enforced by associating metadata
to the data being processed (\EG this is an instruction, this is from the
network, this is private, \ETC), propagating the metadata as
instructions are executed and using a set of rules on the metadata to check
whether a policy is violated and how the tags should be propagated.

Abstractly, these rules form a partial function from a set of
input tags to a set of output tags\ch{unclear whether this
  is supposed to be a type or a term; if type then it doesn't
  match our concrete implementation (just bits) or symbolic abstraction 
  (where we have a complex dependent type now). It would be better
  though if this was clearly a (pseudo-Coq) term, and if you used the
  same notation as in the symbolic rules below: CI, PC, etc. So
  $\ii{transfer}(\ii{opcode}, \ii{PC}, \ii{CI}, \ldots) = \ii{Some}~\ldots$}
\nick{then this would not match the syntax we use when defining the rules.
I think it just captures the type in an (as mentioned) abstract way.
We could match the notation by using the record notation like
$$opcode:{PC,CI,...} \rightarrow {PC',RES}$$ but it wouldn't disambiguate
match in terms of what you are saying. It would still give one notation
less to worry about and would be something familiar for the examples later}
$$(opcode, tag_{pc},tag_{instr}, tag_{arg1}, tag_{arg2}, tag_{arg3})
\rightarrow (tag_{pc'},tag_{result})$$
informally read as, ``if the next instruction to be executed is opcode, the
current tag of the program counter is $tag_{pc}$, the current tag on the
instruction location is $tag_{instr}$ and the tags on the operands of the
instruction are $tag_{arg1}, tag_{arg2}$ and $tag_{arg3}$ then if execution of
the instruction is allowed the tag on the program counter should be set
to $tag_{pc'}$ and any new data created by the instruction should be tagged
$tag_{result}$''.

More specific, a micro-policy is made up of the following elements:
\begin{enumerate}
\item a set of {\em metadata tags} used to tag the contents of the memory and
all the registers as well as the pc.
\item a {\em transfer function} that implements the checks on the tags and
the tag propagation as described above.
\item a tagging scheme for the initial state of the machine.
\item for some micro-policies, a set of {\em monitor services} (\IE privileged
code) that can be invoked by user code.
\end{enumerate}

Furthermore, as we will see in \cref{sec:pump}, a software-hardware
mechanism that enables the efficient implementation of micro-policies
without sacrificing flexibility (in terms of the policies that can be
enforced) has already been designed. Simulations and benchmarks show
that the runtime overhead is very low compared to dedicated software
solutions thus making it a realistic and appealing way to deploy a
wide range of security policies in future computing systems.

\section{Example:
  Non-Writable Code \& Non-Executable Data}
\label{sec:nwc_nxd}

% \ch{Make it clearer that this is informal and
%   you will return to the formalization later on}
% \nick{I think I did now}

% \ch{Symbolic vs concrete rules ... should introduce symbolic rules first,
%   although this is a quite trivial example; ALT: write these as pseudo-Coq
%   functions?}

% \nick{Saying something explicitly about concrete rules here seems hard
%   because it comes out of the blue without the PUMP. Writing a
%   function would omit introduction of syntax for rules and I am not
%   sure if this is what we want.}

In order to demonstrate the mechanism explained above we sketch a

simply micro-policy that enforces that code is not writable(\NWC) and
data are not executable (\NXD), omitting the formalization to which we
will return in \cref{ch:verified_cfi}.

We use the set of tags \TAGS{\DATA,\INSTRname}. If we initially tag
all executable regions in memory as \INSTRname and all non-executable
as \DATAname then we can enforce \NWC and \NXD by two rules of the form

\begin{figure}[htb!]
\infrule[Store/Data]{
  }{
  \frule{Store}{}{\INSTRname}{}{}{\DATA} {-}{\DATA}
  }
\bigskip

\infrule[Rest]{
  \textit{opcode} \not \in \lbrace \ii{Store} \rbrace
  }{
  \frule{\textit{opcode}}{}{\INSTRname}{}{}{} {-}{-}
  }
\bigskip

\caption{Rules enforcing \NWC and \NXD}
\end{figure}

The dashes in the result vector, represent \textit{don't care} values,
meaning we will not use their values for anything, so any tag (usually
a default tag set by the policy designer) can be used.\ch{I don't
really buy your Rest rule then. If I choose the default to be Code
then this policy is completely broken; while in general the intuition
of the don't care default tag is that any value would do. So you
shouldn't be using a don't care for RES I think, but just Data.}
\nick{but isn't this the case you are describing? you can use Code
the tag will just get thrown away. Only Store needs to set the result
tag to DATA. A mov can set the register tag to Code, a load too,
nobody will care}
Furthermore, we are omitting from the input vector the fields that are
unused by the transfer function. For this simple policy, the transfer
function only uses the tag on the current instruction (\ii{CI}) and in
the case of a Store instruction the tag on the memory (\ii{MR}), \IE
the tag on the memory location we are trying to write. If no rule
applies, the execution of the instruction is disallowed. Informally
the above rules can be read as ``Execution is allowed only if the tag
on the current instruction is \INSTRname; if the opcode of the
instruction is Store, we additionally require that the tag of the
overwritten memory location is \DATA. In that case the tag on the new
data on the memory should remain \DATA.''

\section{Generic Verification Framework for Micro-Policies}
\label{sec:framework}

Unsurprisingly, designing a security policy, reasoning about its
effectiveness against potential attackers and encoding it as a
micro-policy can become a complex task. Azevedo \ETAL
\cite{popl2015} built a generic framework for defining and
verifying micro-policies on top of a machine modeling a tagged RISC
processor (referred to as concrete machine), formalized this framework
in Coq and used it to define and formally verify micro-policies for
dynamic sealing, control-flow integrity, memory safety,
compartmentalization and protecting the enforcement mechanism
(referred to as policy monitor) itself.

The framework offers a higher-level machine, called the symbolic
machine, that abstracts away from various - insignificant to security
policies - implementation details. The symbolic machine can be used as
an interface to the concrete machine, simplifying the work of the
micro-policy designer and allowing him to use structured objects in
order to define and reason about the micro-policy, avoiding the
added complexity of working on machine words.

\nick{still unsure about the right way to put this}

In order to implement the micro-policy at the concrete machine level, one needs
to additionally provide machine code that implements the transfer function, an
encoding of tags to words and machine code for any monitor services that the
micro-policy may use. The relation between the symbolic and the concrete machine
is formally defined as a two-way refinement (forward and backward). This is a
generic refinement proof, parameterized by the encoding of the symbolic tags to
words and a proof of correctness of the monitor code for a micro-policy.
The designer of a micro-policy can use this two-way refinement simply by
providing these two parameters.

\subsection{Correctness of micro-policies}\label{sec:verification}

For each micro-policy the policy designer should define an abstract
machine, which serves as a specification to the desired invariants.
The abstract machine is ``correct'' by construction, meaning that it's
designed to respect those invariants. Using the symbolic machine as an
intermediate step to simplify the proofs, by proving a refinement
between the symbolic and the abstract machine and by utilizing the
generic refinement between the symbolic and the concrete machine, we
can prove a refinement between the abstract and the concrete machine,
thus showing that every step of the concrete machine adheres to the
specification expressed by the abstract machine.

All the machines introduced in the original paper by Azevedo \ETAL
\cite{popl2015}, as well as this thesis, have a similar
structure. In particular, they share a common RISC-based instruction
set (with a few - uninteresting for the scope of this thesis -
exceptions) and they have a fixed number of general-purpose registers,
along with a pc register. Of course the abstract machine defined by
the policy designer can differ in various ways, but more similarities
with the symbolic machine implies easier proofs of correctness.

\ch{Introduce the (names of the) various machines and
  how they relate to each-other. Nice diagram?}

\TODO{Write instruction set? maybe not}\ch{Yes, the instruction syntax
  and informal semantics would be very helpful for understanding rules
  later on; not clear if this is the right place, or more likely in
  the Symbolic machine section (where you already have the semantic
  rules)}

\subsection{Symbolic Machine}\label{sec:symbolic}

As mentioned above, the symbolic machine enables us to abstract away from
various low-level details of the concrete machine. We can express and reason
about policies in terms of mathematical objects written in Gallina rather than
machine code and the corresponding proofs for the concrete machine comes for
free under some assumptions.
In essence, the symbolic machine is parameterized by a micro-policy as it was
defined in \ref{sec:micropolicies}, with the addition of an internal state
that can be used by monitor services.

The states of the symbolic machine consists of the memory, the
registers, the \pc register and the internal state. The memory and
register contents, as well as the \pc, are all tagged with a symbolic
tag drawn from the set of meta-data tags of the micro-policy. We name
their contents \textit{symbolic atoms} referred to with the notation
\atom{\ii{w}}{\ii{t}}, where \ii{w} is the value (word) and \ii{t} is
the tag.

At each step, a record named \emph{mvector} is formed. It consists of
the current opcode, the tag on the \pc, the tag on the current
instruction and optionally up to three tags depending on the opcode of
the instruction. The \emph{mvector} is passed to the transfer function
which decides whether the step violated the enforced policy. In the
case of a violation the machine is halted, otherwise if no violation
occurred the \TRANSFER function returns a tag for the new \pc and a
tag for any results the execution of the instruction produced.

In \cref{symbolic_step} we give, in form of inference rules, the
stepping relation for the Symbolic machine, demonstrating how the
transfer function and the tag propagation works at each step.

\begin{figure}[htbp]
\small
%Nop
\infrule[Nop]{
  \mem[\pc] = \atom{i}{\ti} \andalso \ii{decode}~i = \ii{Nop} \\
  \frule{\ii{Nop}}{\tpc}{\ti}{}{}{} {\tpc'}{-} \\
  }{\step{\astat{\mem}{\reg}{\atom{\pc}{\tpc}}{int}}
  {\astat{\mem}{\reg}{\atom{\pc + 1}{\tpc'}}{int}}
  }
\bigskip

%Const
\infrule[Const]{
  \mem[\pc] = \atom{i}{\ti} \andalso \ii{decode}~i = \ii{Const}~n~r \andalso
  \rd{\reg}{r} {=} \atom{w\old}{t\old} \\
  \frule{\ii{Const}}{\tpc}{\ti}{t\old}{}{} {\tpc'}{\targ{res}} \\
  \reg' = \upd{\reg}{r}{n@\targ{res}}
  }{\step{\astat{\mem}{\reg}{\atom{\pc}{\tpc}}{int}}
  {\astat{\mem}{\reg'}{\atom{\pc + 1}{\tpc'}}{int}}
  }
\bigskip

%Mov
\infrule[Mov]{
  \mem[\pc] = \atom{i}{\ti} \andalso \ii{decode}~i = \ii{Mov}~r_p~r_s \\
  \rd{\reg}{r_p} {=} \atom{w}{t_p} \andalso
  \rd{\reg}{r_s} {=} \atom{w\old}{t\old} \\
  \frule{\ii{Mov}}{\tpc}{\ti}{t_p}{t\old}{} {\tpc'}{\targ{res}} \\
  \reg' = \upd{\reg}{r_s}{w@\targ{res}}
  }{\step{\astat{\mem}{\reg}{\atom{\pc}{\tpc}}{int}}
  {\astat{\mem}{\reg'}{\atom{\pc + 1}{\tpc'}}{int}}
  }
\bigskip

%Binop
\infrule[Binop]{
  \mem[\pc] = \atom{i}{\ti} \andalso \ii{decode}~i = \ii{Binop}~op~r_p~r_s~r_t\\
  \rd{\reg}{r_p} {=} \atom{w_p}{t_p} \andalso
  \rd{\reg}{r_s} {=} \atom{w_s}{t_s} \andalso
  \rd{\reg}{r_t} {=} \atom{w\old}{t\old} \\
  \frule{\ii{Binop op}}{\tpc}{\ti}{t_p}{t_s}{t\old} {\tpc'}{\targ{res}} \\
  \reg' = \upd{\reg}{r_t}{(w_p{op}w_s@\targ{res}}
  }{\step{\astat{\mem}{\reg}{\atom{\pc}{\tpc}}{int}}
  {\astat{\mem}{\reg'}{\atom{\pc + 1}{\tpc'}}{int}}
  }
\bigskip

%Load
\infrule[Load]{
  \mem[\pc] = \atom{i}{\ti} \andalso \ii{decode}~i = \ii{Load}~r_p~r_s\\
  \rd{\reg}{r_p} {=} \atom{w_p}{t_p} \andalso
  \rd{\mem}{w_p} {=} \atom{w}{\targ{mem}} \andalso
  \rd{\reg}{r_s} {=} \atom{w\old}{t\old} \\
  \frule{\ii{Load}}{\tpc}{\ti}{t_p}{\targ{mem}}{t\old} {\tpc'}{\targ{res}} \\
  \reg' = \upd{\reg}{r_s}{(w@\targ{res}}
  }{\step{\astat{\mem}{\reg}{\atom{\pc}{\tpc}}{int}}
  {\astat{\mem}{\reg'}{\atom{\pc + 1}{\tpc'}}{int}}
  }
\bigskip

%Store
\infrule[Store]{
  \mem[\pc] = \atom{i}{\ti} \andalso \ii{decode}~i = \ii{Store}~r_p~r_s  \\
  \rd{\reg}{r_p} {=} \atom{w_p}{t_p} \andalso
  \rd{\reg}{r_s} {=} \atom{w_s}{t_s} \andalso
  \rd{\mem}{w_p} {=} \atom{w\old}{t\old} \\
  \frule{\ii{Store}}{\tpc}{\ti}{t_p}{t_s}{t\old} {\tpc'}{t_d'} \\
  \mem' = \upd{\mem}{w_p}{\atom{w_s}{t_d'}}
  }{\step{\astat{\mem}{\reg}{\atom{\pc}{\tpc}}{int}}
  {\astat{\mem'}{\reg}{\atom{\pc + 1}{\tpc'}}{int}}
  }
\bigskip

%Jump
\infrule[Jump]{
  \mem[\pc] = \atom{i}{\ti} \andalso \ii{decode}~i = \ii{Jump}~r \andalso
  \rd{\reg}{r} {=} \atom{w}{t_w} \\
  \frule{\ii{Jump}}{\tpc}{\ti}{t_w}{}{} {\tpc'}{-}
  }{\step{\astat{\mem}{\reg}{\atom{\pc}{\tpc}}{int}}
  {\astat{\mem}{\reg}{\atom{w}{\tpc'}}{int}}
  }
\bigskip

%Bnz
\infrule[Bnz]{
  \mem[\pc] = \atom{i}{\ti} \andalso \ii{decode}~i = \ii{Bnz}~r~n \andalso
  \rd{\reg}{r} {=} \atom{w}{t_w} \\
  \frule{\ii{Bnz}}{\tpc}{\ti}{t_w}{}{} {\tpc'}{-}
  }{\step{\astat{\mem}{\reg}{\atom{\pc}{\tpc}}{int}}
  {\astat{\mem}{\reg}{\atom{w}{\tpc'}}{int}}
  }
\bigskip

%Jump
\infrule[Jal]{
  \mem[\pc] = \atom{i}{\ti} \andalso \ii{decode}~i = \ii{Jal}~r \\
  \rd{\reg}{r} {=} \atom{w}{t_w} \andalso
  \rd{\reg}{ra} {=} \atom{w\old}{t\old} \\
  \frule{\ii{Jal}}{\tpc}{\ti}{t_w}{t\old}{} {\tpc'}{-} \\
  \reg' = \upd{\reg}{ra}{\atom{pc + 1}{\targ{res}}}
  }{\step{\astat{\mem}{\reg}{\atom{\pc}{\tpc}}{int}}
  {\astat{\mem}{\reg'}{\atom{w}{\tpc'}}{int}}
  }
\bigskip

%Monitor Service
\infrule[Service]{
  \mem[\pc] = \varnothing \andalso
  \ii{get\_service}~\pc = (\ti,f) \\
  \frule{\ii{Service}}{\tpc}{\ti}{}{}{} {\tpc'}{-} \\
  f~\acfistat{\mem}{\reg}{\pc}{\extra}{}=
    \acfistat{\mem'}{\reg'}{\pc'}{\extra}{}
  }{\step{\astat{\mem}{\reg}{\atom{\pc}{\tpc}}{\extra}}
  {\astat{\mem'}{\reg'}{\atom{\pc'}{\tpc'}}{\extra'}}
  }

\caption{Stepping relation for the symbolic machine}
\label{symbolic_step}
\end{figure}

Notice for example, that when a store instruction is executed, the tag on
the memory location to be overwritten is fetched, allowing the
\TRANSFER function to know what kind of data we are trying to
overwrite. Returning to the example micro-policy in \ref{sec:nwc_nxd}
we can define the transfer function that is used by the symbolic machine
as a Coq function.

\lstheader{transfer_coq_nwc_nxd}{Transfer function for NWC and NXD in pseudo-code}
Definition transfer ivec : option ovec :=
  match ivec with
  | mkIVec   Store  _ Code  [_ ; _ ; Data] =>
    Some (mkOVec _ Data)
  | mkIVec   Store  _  _  _  => None
  | mkIVec   _      _  Code _   =>
    Some (mkOVec _ _)
  | mkIVec _ _ _ _ => None
  end.
\end{lstlisting}
\nick{heavily abusing notation with \_ on result vectors}

\section{A Programmable Unit for Metadata Processing}\label{sec:pump}

%\ch{Could consider moving this one level up (turn it into chapter)}

\subsection{Hardware Architecture}

The Programmable Unit for Metadata Processing (PUMP) architecture
\cite{pump_hasp2014} allows us to efficiently implement a wide range
of micro-policies, using software to describe
the micro-policy, while the hardware provides efficiency
by undertaking the propagation of the tags and by using a cache for
the rules.

On the hardware level, the PUMP is an extension to a conventional RISC
architecture. Every word of data in the machine - whether in memory or
a register, is extended with a word-sized metadata tag.  These tags
are not interpreted by hardware, instead the interpretation of the
tags is left to the software, thus making it easy to implement new
policies on the metadata. Since tags are word-sized, they can be
pointers to complex data-structures of tags, such as tuples of tags,
allowing for complex policies to be expressed and multiple orthogonal
policies to be enforced in parallel.

The hardware undertakes the correct propagation of tags from operands
to results according to the rules defined by the software. A hardware
rule cache mapping sets of input tags to sets of output tags is used
for common case efficiency. On each instruction dispatch, in parallel
with the usual behavior of an instruction (\EG execution of an
addition in the ALU), the hardware forms the set of input tags and a
lookup is performed on the rule cache. If the lookup is successful a
set of output tags is returned and combined with the results of the
normal execution of the instruction a new state is produced. On the
other hand, if the lookup failed, the hardware invokes a trusted piece
of system software - the fault handler - which checks the input tags
and decides whether the execution should be allowed or not. In the
first case, the fault handler returns a set of result tags, a pair of
set of input and output tags is formed and inserted into the rules
cache, while the faulting instruction is restarted and will now hit
the cache. Otherwise, execution of this instruction violated some
rules of the enforced policy and execution should not continue
normally (\EG should be halted).

As described in the original PUMP paper by Dehon \ETAL
in \cite{pump_hasp2014} a rich set of
effective security policies can be efficiently implemented using the
architecture mentioned above. In particular, implementations of
dynamic typing, memory safety for heap-based data, control flow
integrity and taint tracking are described, evaluated against a
specific threat model and benchmarked. The benchmarks are done using a
simulation of the described hardware and the two papers claim low
overhead (~3\% on average) for each of the policies named above.

Compared to other software solutions for enforcing security policies, the PUMP
offers significantly lower overhead, thanks to dedicated hardware assistance,
while the fact that interpretation of the metadata is done by software offers
flexibility with regard to the policies that can be implemented, compared to
hardware solutions implementing a specific policy.

While the PUMP offers flexibility at a low runtime performance
overhead, there are more overheads associated to such a mechanism. For
example adding metadata to all the data in the machine, would result
in a 100\% memory overhead.  In addition, the extra hardware and the
rule cache along with potentially larger memories could result into a
400\% overhead on energy usage. The authors
claim that a careful and well-optimized implementation can reduce
these numbers, resulting in a 50\% energy overhead.
\nick{I can use optimized numbers but they are taken out of asplos paper,
so I am now just taking numbers out of nowhere}
%
\ch{use optimized numbers}

\subsection{Concrete Machine Modeling PUMP Architecture}\label{sec:concrete}

The concrete machine is a model of the PUMP architecture, modeling a
RISC machine with a rules \cache and a software \emph{miss handler}.
The instruction set has been extended with four additional
instructions that are meant to be used by monitor code only, a
restriction that is enforced by the monitor self-protection
micro-policy.

The state of the concrete machine consists of the memory, the
registers, the \pc register, the \epc register - a special purpose
register that holds the address of the faulting instruction so the miss handler
can return to it - and a rules cache. The cache works as a key-value store
where a key is an \emph{input vector} that contains an instruction
opcode, the tag of the current instruction, the tag of the pc and up
to three operand tags, and a value is an \emph{output vector} which
contains a tag for the new pc and a tag for any results from the
execution of the instruction. In the context of the concrete machine a
tag is the encoding into a word of a symbolic tag. Lifting this
encoding relation to vectors, we get that a concrete vector is the
encoding of a symbolic vector. Similar to the
symbolic machine the contents of the memory, the registers, the pc and
the epc are concrete atoms \atom{w}{t} where w is a word and t is the
encoding of a tag into a word.

The stepping relation for the concrete machine is a bit more
complicated than the one for the symbolic machine. In particular, on
each step the machine forms the \emph{input vector} and looks it up in
the cache. If the lookup succeeds then the instruction is allowed, an
\emph{output vector} is returned by the cache and the next state is
tagged according to it.  If the lookup fails, then the \emph{input
  vector} is saved in memory, the current \pc is stored in the special
register \epc and the machine traps to the \emph{miss handler}.  The
above are demonstrated in the two example rules in \cref{cstep_store}.

\begin{figure}[htb]
\infrule[Store]{
  \mem[\pc] = \atom{i}{t_i} \andalso \ii{decode}~i = \ii{Store}~r_p~r_s \\
  \rd{\reg}{r_p} {=} \atom{w_p}{t_p} \andalso
  \rd{\reg}{r_s} {=} \atom{w_s}{t_s} \andalso
  \rd{\mem}{w_p} {=} \atom{w\old}{t\old} \\
  \rulelookup{\ii{Store}}{\tpc}{\ti}{t_p}{t_s}{t\old} {\tpc'}{t_d'} \\
  \mem' = \upd{\mem}{w_p}{\atom{w_s}{t_d'}}
  }{\step{\cstat{\mem}{\reg}{\atom{\pc}{\tpc}}{\epc}{\cache}}{
              \cstat{\mem'}{\reg}{\atom{(\pc{+}1)}{\tpc'}}}{\epc}{\cache}}
\bigskip
\infrule[Store-Miss]{
  \mem[\pc] = \atom{i}{t_i} \andalso \ii{decode}~i = \ii{Store}~r_p~r_s  \\
  \rd{\reg}{r_p} {=} \atom{w_p}{t_p} \andalso
  \rd{\reg}{r_s} {=} \atom{w_s}{t_s} \andalso
  \rd{\mem}{w_p} {=} \atom{w\old}{t\old} \\
  \rulelookupmiss{\ii{Store}}{\tpc}{\ti}{t_p}{t_s}{t\old} \\
  \mem' = \savestate{\ii{Store}}{\tpc}{\ti}{t_p}{t_s}{t\old} {\mem}
  }{\step{\cstat{\mem}{\reg}{\atom{\pc}{\tpc}}{\epc}{\cache}}{
              \cstat{\mem'}{\reg}{\atom{\trapaddr}{\ii{Monitor}}}{\atom{\pc}{\tpc}}{\cache}}}
\caption{Concrete step rules for Store instruction}
\label{cstep_store}
\end{figure}

Addresses 0 to 5 are used to store the \emph{input vector} and 6 to 7 are used
by the miss handler to store the \emph{output vector}. As a side-note, cache
eviction is not modeled (an infinite cache is assumed).

\subsection{Concrete Policy Monitor}\label{sec:concrete_policy}

Unlike the symbolic machine, where the user cannot cannot change the
\TRANSFER function, enforcing a micro-policy on the concrete machine requires
that we are able to protect the policy monitor itself and that privileged
instructions are not executed by user code. This self-protection policy can be
easily composed with another micro-policy and enforced by the infrastructure
described above.

Using tags of the form, \USER{\ii{st}}, \ENTRY{\ii{st}}, \MONITOR we can
distinguish between user-level data, the monitor and monitor services.
In particular \USER{\ii{st}} is used to tag a user-level atom, where \ii{st} is
the word-encoding of a symbolic tag. \MONITOR is used to tag the monitor memory
and registers. The \pc is tagged with \MONITOR when a monitor
execution takes place and \USER{\ii{st}} when user-code is executed. The tag
\ENTRY{\ii{st}} is used to tag the first instruction of a monitor service and
serves as an indication that execution will continue under the privileged
\MONITOR mode. \nick{especially the last phrase can be tweaked}

The miss handler is a composed policy monitor that protects itself
from \USERname code and that enforces a desired micro-policy.  One
important thing to note is that the miss handler for the concrete
machine can take an arbitrary number of steps before deciding whether
no violation occurred and returning to \USERname mode, unlike the
symbolic \TRANSFER function that does not need to take any steps.
\nick{mention here what happens if a violation happened?}

